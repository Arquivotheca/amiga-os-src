%
%	The ARexx User's Manual
%
% Copyright � 1987 by William S. Hawes.  All Rights Reserved.
%
\ifx\radeye\fmtversion\subdoc\else\input pmac\fi

\chapter{Appendix}{C}{The \AR\ Systems Library}%
The \AR\ interpreter is supplied as a shared library named
|rexxsyslib.library| and should reside in the system |LIBS:| directory.
While many of the library routines are highly specific to the interpreter,
some of the functions will be useful to applications that use \AR.
The library is opened when the \AR\ resident process is first loaded and will 
always be available while it remains active.

The system library routines were designed to be called from assembly-language
programs and,
unless otherwise noted, save all registers except for A0/A1 and D0/D1.  
Many routines return values in more than one register to help reduce
code size.
In addition,
the routines will set the condition-code register (CCR) wherever appropriate.
In most cases the CCR reflects the value returned in D0.

The library offsets are defined in the file |rxslib.i|, 
which should be INCLUDEd in the program source code.
Calls may be made from ``C'' programs if suitable binding routines are 
provided when the program is linked.
The definitions for the constants and data structures used in \AR\ are 
provided as INCLUDE files on the program distribution disk.  
These should be reviewed carefully before attempting to use the library 
functions.

Some of the library functions are not documented here.
These private entry points are reserved for the internal use of the 
interpreter and should not be called from external programs.

\section{\chapterno-}{Functional Groups}%
The library functions can be grouped into \hp{Conversion}, \hp{Input/Output},
\hp{Resource Management}, and \hp{String Manipulation} functions.

\subsub{Data Conversion}.
These functions provide many of the common data-conversion requirements.

\subsub{Input/Output}.
Two levels of I/O support are provided.
The low level functions use DOS filehandles directly,
while the higher-level functions use linked lists of {\bf IoBuff} structures 
and support logical file names.

\subsub{Resource}.
These functions allocate, release,
or otherwise manage the data structures used with \AR.

\subsub{String Functions}.
All data in \AR\ are managed as strings.  
These functions provide some of the more common string-manipulation operations.

$$\vbox{\halign{\tt #\hfil&&\qquad#\hfil\cr
\multispan3\hfil Table \chapterno .1 \AR\ Systems Library Functions\hfil\cr
\bf Name&\bf Functional Group&\bf Description\cr
AddClipNode&Resource&Allocate a Clip node\cr
ClearMem&Resource&Clear a block of memory\cr
ClearRexxMsg&Resource&Release argstrings from message\cr
CloseF&Input/Output&Close a file buffer\cr
ClosePublicPort&Resource&Close a port resource node\cr
CmpString&String&Compare string structures for equality\cr
CreateArgstring&Resource&Create an argstring structure\cr
CreateDOSPkt&Input/Output&Create a DOS StandardPacket\cr
CreateRexxMsg&Resource&Create a message packet\cr
CurrentEnv&Resource&Get current storage environment\cr
CVa2i&Conversion&ASCII to integer\cr
CVc2x&Conversion&Character to Hex or Binary digits\cr
CVi2a&Conversion&Integer to ASCII\cr
CVi2arg&Conversion&Integer to ASCII argstring\cr
CVi2az&Conversion&Integer to ASCII, leading zeroes\cr
CVs2i&Conversion&String structure to integer\cr
CVx2c&Conversion&Hex or binary digits to binary\cr
DeleteArgstring&Resource&Release an argstring structure\cr
DeleteDOSPkt&Input/Output&Release a DOS StandardPacket\cr
DeleteRexxMsg&Resource&Release a message packet\cr
DOSRead&Input/Output&Read from a DOS filehandle\cr
DOSWrite&Input/Output&Write to a DOS filehandle\cr
ErrorMsg&Conversion&Get error message from error code\cr
ExistF&Input/Output&Check whether a DOS file exists\cr
FillRexxMsg&Resource&Convert and install argstrings\cr
FindDevice&Input/Output&Locate a DOS device node\cr
FindRsrcNode&Resource&Locate a resource node\cr
FreePort&Resource&Close a message port\cr
FreeSpace&Resource&Release internal memory\cr
GetSpace&Resource&Allocate internal memory\cr
InitList&Resource&Initialize a list header\cr
InitPort&Resource&Initialize a message port\cr
IsRexxMsg&Resource&Test a message packet\cr
LengthArgstring&Resource&Get length of argstring\cr
ListNames&Resource&Copy node names to an argstring\cr
OpenF&Input/Output&Open a file buffer\cr
OpenPublicPort&Resource&Allocate and open a port resource node\cr
QueueF&Input/Output&Queue a line in a file buffer\cr
ReadF&Input/Output&Read from a file buffer\cr
ReadStr&Input/Output&Read a string from a file buffer\cr
RemClipNode&Resource&Release a Clip node\cr
RemRsrcList&Resource&Release a resource list\cr
RemRsrcNode&Resource&Release a resource node\cr}}$$

$$\vbox{\halign{\tt #\hfil&&\qquad#\hfil\cr
\multispan3\hfil Table \chapterno .1  Library Functions (cont.)\hfil\cr
\bf Name&\bf Functional Group&\bf Description\cr
SeekF&Input/Output&Reposition a file buffer\cr
StackF&Input/Output&Stack a line in a file buffer\cr
StcToken&String&Break out a token\cr
StrcmpN&String&Compare strings\cr
StrcpyA&String&Copy a string, converting to ASCII\cr
StrcpyN&String&Copy a string\cr
StrcpyU&String&Copy a string, converting to uppercase\cr
StrflipN&String&Reverse characters in a string\cr
Strlen&String&Find length of a string\cr
ToUpper&Conversion&ASCII to uppercase\cr
WriteF&Input/Output&Write to a file buffer\cr}}$$

\section{\chapterno-}{Library Functions}%
The following descriptions of the \AR\ Systems Library functions are listed
alphabetically.  
The required arguments and register assignments are shown in parentheses 
after the function name.
Multiple returns are shown in parentheses on the left-hand side of the call. 

\subsub{AddClipNode()}%
-- allocate and link a Clip node
\begverb
Usage: node = AddClipNode(list,name,length,value)
        D0                 A0   A1    D0     D1
        A0
       (CCR)
#endverb%
Allocates and links a Clip node into the specified list.
Clip nodes are resource nodes containing a name and value string,
and include an ``auto-delete'' function for simple maintenance.
The {\it list} argument must point to a properly-initialized EXEC list header.
The {\it name} argument points to a null-terminated name string,
the {\it value} argument is a pointer to a storage area, 
and the {\it length} argument is its length in bytes.
The returned value is a pointer to the allocated node, 
or 0 if the allocation failed.

The {\bf RemClipNode()} function is installed as the ``auto-delete'' function
for each node.
Clip nodes can be intermixed with other resource nodes in a list and then
released with a single call to {\bf RemRsrcList()}.
\idxline{AddClipNode(),}{Systems Library function}
\seealso{RemClipNode(), RemRsrcList(), RemRsrcNode()}

\subsub{AddRsrcNode()}%
-- allocate and link a resource node
\begverb
Usage: node = AddRsrcNode(list,name,length)
        D0                 A0   A1    D0
        A0
       (CCR)
#endverb%
Allocates and links a resource node (a {\bf RexxRsrc} structure) to the 
specified {\it list}.
The {\it name} argument is a pointer to a null-terminated string, 
a copy of which is installed in the node structure.  
The length argument is the total length for the node; 
this length is saved within the node so that it may be released later.  
The returned value is a pointer to the allocated node, 
or 0 if the allocation failed.
\idxline{AddRsrcNode(),}{Systems Library function}
\seealso{RemRsrcList(), RemRsrcNode()}

\subsub{ClearMem()}%
-- clear a block of memory
\begverb
Usage: ClearMem(address,length)
                  A0      D0
#endverb
Clears a block of memory beginning at the given {\it address} for the 
specified {\it length} in bytes.
The address must be word-aligned and the length must be a multiple of 4 bytes;
all structures allocated by \AR\ meet these restrictions.
Register A0 is preserved.
\idxline{ClearMem(),}{Systems Library function}

\subsub{ClearRexxMsg()}%
-- release argument strings
\begverb
Usage: ClearRexxMsg(msgptr,count)
                      A0    D0
#endverb
Releases one or more argstrings from a message packet and clears the
corresponding slots.
The {\it count} argument specifies the number of argument slots to clear,
and can be set to less than 16 to reserve some to the slots for private use.
No action is taken if the slot already contains a zero value.
\idxline{ClearRexxMsg(),}{Systems Library function}
\seealso{FillRexxMsg()}

\subsub{CloseF()}%
-- close a file buffer
\begverb
Usage: boolean = CloseF(IoBuff)
         D0               A0
#endverb
Releases the {\bf IoBuff} structure and closes the associated DOS file.
|CloseF()| is the ``auto-delete'' function for the {\bf IoBuff} structure,
so an entire list of file buffers can be closed with a single call to
|RemRsrcList()|.
\idxline{CloseF(),}{Systems Library function}

\subsub{ClosePublicPort()} 
-- close a port resource node
\begverb
Usage: ClosePublicPort(node)
                        A0
#endverb%
Unlinks and closes the message port and releases the resource node structure.
The node must have been allocated by the |OpenPublicPort()| function.
\idxline{ClosePublicPort(),}{Systems Library function}

\seealso{OpenPublicPort()}%

\subsub{CmpString()}%
-- compare string structures for equality
\begverb
Usage: test = CmpString(ss1,ss2)
        D0               A0  A1
       (CCR)
#endverb%
The arguments {\it ss1} and {\it ss2} must be pointers to \AR\ 
string structures and are compared for equality.
String structures include the length and hash code of the string, 
so the actual strings are not compared unless the lengths and hash codes match.
The return value sets the CCR and will be -1 ({\bf True}) if the strings match
and 0 ({\bf False}) otherwise.
\idxline{CmpString(),}{Systems Library function}

\subsub{CreateArgstring()}%
-- create an argument string structure
\begverb
Usage: argstring = CreateArgstring(string,length)
          D0                         A0     D0
          A0
         (CCR)
#endverb%
Allocates a {\bf RexxArg} structure and copies the supplied string into it.
The {\it argstring} return is a pointer to the string buffer of the structure, 
and can be treated like an ordinary string pointer.  
The {\bf RexxArg} structure stores the structure size and string length at 
negative offsets to the string pointer.
The string pointer can be set to NULL if only an uninitialized structure is 
required.
\idxline{CreateArgstring(),}{Systems Library function}
\seealso{DeleteArgstring()}%

\subsub{CreateDOSPkt()}%
-- allocate and initialize a DOS standardPacket.
\begverb
Usage: packet = CreateDOSPkt()
         D0
         A0
        (CCR)
#endverb%
Allocates a DOS {\bf StandardPacket} structure and initializes it by 
interlinking the EXEC message and the DOS packet substructures.
No replyport is installed in either the message or the packet,
as these fields are generally filled in just before the message is sent.
\idxline{CreateDOSPkt(),}{Systems Library function}
\seealso{DeleteDOSPkt()}%

\subsub{CreateRexxMsg()}%
-- allocate an \AR\ message packet
\begverb
Usage: msgptr = CreateRexxMsg(replyport,extension,host)
         D0                     A0         A1      D0
         A0
        (CCR)
#endverb%
This functions allocates an \AR\ message packet from the system free memory 
list.
The message packet consists of a standard EXEC message structure extended 
to include space for function arguments, returned results, 
and internal defaults.
The {\it replyport} argument points to a public or private message port and
must be supplied, 
as it is required to return the message packet to the sender.
The {\it extension} and {\it host} arguments are pointers to null-terminated
strings that provide values for the default file extension and the initial 
host address, respectively.
Additional override fields in the extended packet may be filled in
after the packet has been allocated.

The interpreter preserves all of the fields in the message packet except 
for the primary and secondary result fields |rm\_Result1| and |rm\_Result2|.
\idxline{CreateRexxMsg(),}{Systems Library function}
\seealso{DeleteRexxMsg()}%

\subsub{CVa2i()}%
-- convert from ASCII to integer
\begverb
Usage: (digits,value) = CVa2i(buffer)
          D0     D1             A0
#endverb%
Converts the {\it buffer} of ASCII characters to a signed long integer value.
The scan proceeds until a non-digit character is found or until an overflow 
is detected.  
The function returns both the number of digits scanned and the converted value.
\idxline{CVa2i(),}{Systems Library function}

\subsub{CVc2x()}%
-- convert (unpack) from character string to hex or binary digits.
\begverb
Usage: error = CVc2x(outbuff,string,length,mode)
         D0             A0     A1     D0    D1 
#endverb%
Converts the {\it string} argument to a string of hex (0-9, A-F) or 
binary (0,1) digits.
\idxline{CVc2x(),}{Systems Library function}

\subsub{CVi2a()}%
-- convert from integer to ASCII
\begverb
Usage: (length,pointer) = CVi2a(buffer,value,digits)
          D0     A0               A0     D0    D1
#endverb%
Converts the signed integer {\it value} argument to ASCII characters using 
the supplied {\it buffer} pointer.
The {\it digits} argument specifies the maximum number of characters 
that will be copied to the buffer.  
The returned {\it length} is the actual number of characters copied.
The {\it pointer} return is the new buffer pointer.
\idxline{CVi2a(),}{Systems Library function}

\seealso{CVi2az()}%

\subsub{CVi2arg()}%
-- convert from integer to argstring
\begverb
Usage: argstring = CVi2arg(value,digits)
          D0                 D0    D1
          A0
         (CCR)
#endverb%
Converts the signed long integer {\it value} argument to ASCII characters, 
and installs them in an argstring (a {\bf RexxArg} structure).
The returned value is an argstring pointer or 0 if the allocation failed.
The allocated structure can be released using |DeleteArgstring()|.
\idxline{CV2i2arg(),}{Systems Library function}

\subsub{CVi2az()}%
-- convert from integer to ASCII with leading zeroes
\begverb
Usage: (length,pointer) = CVi2az(buffer,value,digits)
         D0     A0                 A0     D0    D1
#endverb%
Converts the signed long integer {\it value} argument to ASCII characters
in the supplied {\it buffer}, 
with leading zeroes to fill out the requested number of {\it digits}.
This function is identical to CVi2a except that leading zeroes are supplied.
\idxline{CVi2az(),}{Systems Library function}

\seealso{CVi2a()}%

\subsub{CVs2i()}%
-- convert from string structure to integer
\begverb
Usage: (error,value) = CVs2i(ss)
          D0    D1           A0
#endverb%
The {\it ss} argument must be a pointer to a string structure.
It is converted to a signed long integer {\it value} return.  
The {\it error} return code is 47 ("Arithmetic conversion error") if the
string is not a valid number.
\idxline{CVs2i(),}{Systems Library function}

\subsub{CVx2c()}%
-- convert from hex or binary digits to (packed) string
\begverb
Usage: error = CVx2c(outbuff,string,length,mode)
         D0             A0     A1     D0    D1 
#endverb%
Converts the {\it string} argument of hex (0-9,A-F) or binary (0,1) digits
to the packed binary representation.  
The {\it mode} argument specifies the (hex or binary) conversion mode,
and must be set to -1 for hex strings or 0 for binary strings.  
Blank characters may be embedded in the string for readability, 
but only at byte boundaries.  
The {\it error} return code is 47 if the string is not a valid hex 
or binary string.
\idxline{CVx2c(),}{Systems Library function}

\subsub{CurrentEnv()}%
-- return the current storage environment
\begverb
Usage: envptr = CurrentEnv(rxtptr)
         D0                  A0
#endverb%
Returns a pointer to the current storage environment associated with an
executing \AR\ program.
The {\it rxtptr} argument is a pointer to the {\bf RexxTask} structure,
and may be obtained from the message packet sent to an external application.
\idxline{CurrentEnv(),}{Systems Library function}

\subsub{DeleteArgstring()}%
-- delete (release) an argstring structure
\begverb
Usage: DeleteArgstring(argstring)
                          A0
#endverb%
Releases an argstring ({\bf RexxArg}) structure.  
The {\bf RexxArg} structure contains the total allocated length at a negative
offset from the {\it argstring} pointer.
\idxline{DeleteArgstring(),}{Systems Library function}
\seealso{CreateArgstring()}%

\subsub{DeleteDOSPkt()}%
-- release a DOS StandardPacket structure.
\begverb
Usage: DeleteDOSPkt(message)
                      A0
#endverb%
Releases a DOS {\bf StandardPacket} structure,
typically obtained by a prior call to |CreateDOSPkt()|.
\idxline{DeleteDOSPkt(),}{Systems Library function}

\seealso{CreateDOSPkt()}%

\subsub{DeleteRexxMsg()}%
-- delete (release) an \AR\ message packet
\begverb
Usage: DeleteRexxMsg(packet)
                       A0
#endverb%
Releases an \AR\ message packet to the system free-memory list.
The internal |MN\_LENGTH| field is used as the total size of the memory block 
to be released,
so this function can be used to release any message packet that contains the
total length in this field.
Any embedded argument strings must be released before calling 
|DeleteRexxMsg()|.
\idxline{DeleteRexxMsg(),}{Systems Library function}
\seealso{CreateRexxMsg()}%

\subsub{DOSRead()}%
-- read from a DOS file
\begverb
Usage: count = DOSRead(filehandle,buffer,length)
         D0                A0       A1    D0
        (CCR)
#endverb%
Reads one or more characters from a DOS filehandle into the supplied buffer.
The {\it length} argument specifies the maximum number of characters that will 
be read.  
The returned {\it count} is the actual number of bytes transferred,
or -1 if an error occurred.
\idxline{DOSRead(),}{Systems Library function}

\subsub{DOSWrite()}%
-- write to a DOS file
\begverb
Usage: count = DOSWrite(filehandle,buffer,length)
         D0                 A0       A1     D0
        (CCR)
#endverb%
Writes a buffer of the specified length to a DOS filehandle.  
The returned {\it count} is the actual number of bytes written,
or -1 if an error occurred.
\idxline{DOSWrite(),}{Systems Library function}

\subsub{ErrorMsg()}%
-- find the message associated with an error code
\begverb
Usage: (boolean,ss) = ErrorMsg(code)
          D0    A0              D0
		
#endverb%
Returns the error message (as a pointer to a string structure) associated
with the specified \AR\ error code.
The boolean return will be -1 if the supplied code was a valid \AR\ error code,
and 0 otherwise.
\idxline{ErrorMsg(),}{Systems Library function}

\subsub{ExistF()}%
-- check whether an external file exists
\begverb
Usage: boolean = ExistF(filename)
         D0                A0
        (CCR)
#endverb%
Tests whether an external file currently exists by attempting to obtain a
read lock on the file.
The boolean return indicates whether the operation succeeded, 
and the lock is released.
\idxline{ExistF(),}{Systems Library function}

\subsub{FillRexxMsg()}%
-- convert and install arguments in message packet.
\begverb
Usage: boolean = FillRexxMsg(msgptr,count,mask)
         D0                    A0    D0    D1
        (CCR)
#endverb%
This function can be used to convert and install up to 16 argument strings 
in a {\bf RexxMsg} structure.
The message packet must be allocated and the argument fields of interest
set to either a pointer to a null-terminated string or an integer value.
The {\it count} argument specifies the number of fields, beginning with |ARG0|,
to be converted into argstrings and installed into the argument slot.
Bits 0-15 of the {\it mask} argument specify whether the corresponding 
argument is a string pointer (bit clear) or an integer value (bit set).

The {\it count} argument is normally set to the exact number of strings 
to be passed.
By setting this count to less than 16,
a number of the slots can be reserved for private uses.

The returned value is -1 ({\bf True}) if all of the arguments were
successfully converted.
In the event of an allocation failure, 
all of the partial results are released and a value of 0 is returned.
\idxline{FillRexxMsg(),}{Systems Library function}

\seealso{ClearRexxMsg()}%

\subsub{FindDevice()}%
-- check whether a DOS device exists.
\begverb
Usage: device = FindDevice(devicename,type)
         D0                   A0       D0
         A0
        (CCR)
#endverb%
Scans the DOS DeviceList for a device node of the specified type matching
the null-terminated name string.
The acceptable values for the {\it type} argument are the constants 
|DLT\_DEVICE|, |DLT\_DIRECTORY|, or |DLT\_VOLUME| defined in the DOS INCLUDE
files.
Device names are converted to uppercase before checking for a match.  
The returned value is a pointer to the matched device node, 
or 0 if the device was not found.
\idxline{FindDevice(),}{Systems Library function}

\subsub{FindRsrcNode()}%
-- locate a resource node with the given name.
\begverb
Usage: node = FindRsrcNode(list,name,type)
        D0                  A0   A1   D0
        A0
       (CCR)
#endverb%
Searchs the specified list for the first node of the selected type with the
given name.
The {\it list} argument must be a pointer to a properly-initialized EXEC list
header.
The {\it name} argument is a pointer to a null-terminated string.
If the {\it type} argument is 0, all nodes are selected;
otherwise, the supplied type must match the |LN\_TYPE| field of the node.
The returned value is a pointer to the node or 0 if no matching node was found.
\idxline{FindRsrcNode(),}{Systems Library function}

\subsub{FreePort()}%
-- release resources associated with a message port
\begverb
Usage: FreePort(port)
                 A0
#endverb%
This function deallocates the signal bit associated with a message port
and marks the port as ``closed.''  
The task calling |FreePort()| must be the same one that initialized the port,
since signal bit allocations are specific to a task.
The memory space associated with the port is not released.  
\idxline{FreePort(),}{Systems Library function}
\seealso{InitPort()}%

\subsub{FreeSpace()}%
-- releases space to the internal memory allocator.
\begverb
Usage: FreeSpace(envptr,block,length)
                   A0     A1    D0
#endverb%
Returns a block of memory to the internal allocator, 
which must have been obtained from a call to |GetSpace()|.
The {\it envptr} argument is a pointer to the base or current storage 
environment.
\idxline{FreeSpace(),}{Systems Library function}
\seealso{CurrentEnv(), GetSpace()}%

\subsub{GetSpace()}%
-- allocate memory using the internal allocator.
\begverb
Usage:	block = GetSpace(envptr,length)
          D0               A0     D0
          A0
         (CCR)
#endverb%
Allocates a block of memory using the internal allocator.  
The memory is obtained from an internal pool managed by the interpreter 
and is returned to the operating system when the \AR\ program terminates.
The {\it envptr} argument is a pointer to the base or current storage
environment for the program.

The internal allocator must be used to allocate strings for use as 
values for symbols, 
and is convenient for obtaining small blocks of memory whose lifetime 
will not exceed that of the \AR\ program.
\idxline{GetSpace(),}{Systems Library function}

\seealso{CurrentEnv(), FreeSpace()}%

\subsub{InitList()}%
-- initialize a list header
\begverb
Usage:	InitList(list)
                  A0
#endverb%
Initializes an EXEC list header structure.
\idxline{InitList(),}{Systems Library function}

\subsub{InitPort()}%
-- initialize a previously-allocated message port.
\begverb
Usage:	(signal,port) = InitPort(port,name)
          D0     A1               A0   A1
#endverb%
Initializes a message port structure for which memory space has been 
previously allocated, 
typically as part of a larger structure or as static storage in a program.  
It installs the task ID (of the task calling the function) into the 
|MP\_SIGTASK| field and allocates a signal bit.  
The {\it name} parameter must be a pointer to a null-terminated string.  
The {\it signal} return is the signal bit that was allocated for the port.  
In the event that a signal could not be assigned, a value of -1 is returned.

Note that the port is not linked into the system Ports List.
If the port is to be made public,
this can be done after the function returns.
The port address is returned in the correct register (A1) for a subsequent
call to the EXEC function |AddPort()|.
\idxline{InitPort(),}{Systems Library function}

\seealso{FreePort()}%

\subsub{IsRexxMsg()}%
-- check whether a message came from \AR.
\begverb
Usage: boolean = IsRexxMsg(msgptr)
         D0                  A0
#endverb%
Tests whether the message packet specified by the {\it msgptr} argument
came from an \AR\ program.  
\AR\ marks its messages with a pointer to a static string ``REXX'' in 
the |LN\_NAME| field.
The returned value is either -1 ({\bf True}) if the message came from \AR\
or 0 ({\bf False}) otherwise.
\idxline{IsRexxMsg(),}{Systems Library function}

\subsub{IsSymbol()}%
-- check whether a string is a valid symbol.
\begverb
Usage: (code,length) = IsSymbol(string)
         D0    D1                 A0
#endverb%
Scans the supplied string pointer for \AR\ symbol characters.  
The {\it code} return is the symbol type if a symbol was found, 
or 0 if the string did not start with a symbol character.  
The {\it length} return is the total length of the symbol.
\idxline{IsSymbol(),}{Systems Library function}

\subsub{ListNames()}%
-- build a string of names from a list.
\begverb
Usage:	argstring = ListNames(list,separator)
           D0                  A0   D0[0:7]
           A0
          (CCR)
#endverb%
Scans the specified list and copies the name strings into an argstring.
The {\it list} argument must be a pointer to an initialized EXEC list header.
The {\it separator} argument is the character, possibly a null, 
to be placed as a delimiter between the node names.

The list is traversed inside a Forbid() exclusion and so may be used 
with shared or system lists.
The returned argstring can be released using |DeleteArgstring()| after the
names are no longer needed.
\idxline{ListNames(),}{Systems Library function}
\seealso{DeleteArgstring()}%

\subsub{LockRexxBase()}%
-- lock a shared resource.
\begverb
Usage:	LockRexxBase(resource)
                        D0
#endverb%
Secures the specified resource in the \AR\ Systems Library base for read 
access.
The {\it resource} argument is a manifest constant for the required resource,
or zero to lock the entire structure.

Note that write access to shared resources is normally mediated by the \AR\
resident process,
which operates at an elevated priority to gain exclusive access.
Locking a resource should not be attempted from a process operating at a
priority higher than the resident process.
\idxline{LockRexxBase(),}{Systems Library function}

\seealso{UnlockRexxBase()}%

\subsub{OpenF()}%
-- open a file buffer
\begverb
Usage: IoBuff = OpenF(list,filename,mode,logical)
          D0           A0     A1     D0    D1
          A0
         (CCR)
#endverb%
Attempts to open an external file in the specified mode, 
which should be one of the constants |RXIO\_READ|, |RXIO\_WRITE|, 
or |RXIO\_APPEND| defined in the \AR\ INCLUDE files.
If successful,
an {\bf IoBuff} structure is allocated and linked into the specified list.
The {\it list} argument must be a pointer to a properly-initialized EXEC list 
header.

The optional {\it logical} argument is the logical name for the file,
and must be either a pointer to a null-terminated string or zero (|NULL|) if 
a name is not required.
\idxline{OpenF(),}{Systems Library function}
\idxline{IoBuff structure}{}

\seealso{CloseF()}%

\subsub{OpenPublicPort()}%
-- open a public message port
\begverb
Usage: node = OpenPublicPort(list,name)
        D0                    A0   A1
        A0
       (CCR)
#endverb%
Allocates a message port as an ``auto-delete'' resource node and links it 
into the specified list.  
The {\it list} argument must point to a properly initialized EXEC list header.
The message port is initialized with the given name and linked into 
the system Ports List.
\idxline{OpenPublicPort(),}{Systems Library function}
\seealso{ClosePublicPort()}%

\subsub{QueueF()}%
-- queue a line to a file buffer.
\begverb
Usage: count = QueueF(IoBuff,buffer,length)
         D0             A0     A1     D0
#endverb%
Queues a buffer of characters in the stream associated with the {\bf IoBuff}
structure.
The stream must be managed by a DOS handler that supports the |ACTION\_QUEUE|
packet.

Queued lines are placed in ``firtst-in, first-out'' order and are immediately
available to be read from the stream.
The {\it buffer} argument is a pointer to a string of characters, 
and the {\it length} specifies the number of characters to be queued.
The return value is the actual count of characters or -1 if an error occurred.
\idxline{QueueF(),}{Systems Library function}
\idxline{IoBuff structure}{}

\seealso{StackF()}%

\subsub{ReadF()}%
-- read characters from a file buffer
\begverb
Usage: count = ReadF(IoBuff,buffer,length)
         D0            A0     A1     D0
        (CCR)
#endverb%
Reads one or more characters from the file specified by the {\bf IoBuff}
pointer.
The {\it buffer} argument is a pointer to a storage area, 
and the {\it length} argument specifies the maximum number of characters 
to be read.
The return value is the actual number of characters read,
or -1 if an error occurred.
\idxline{QueueF(),}{Systems Library function}
\idxline{IoBuff structure}{}

\subsub{ReadStr()}%
-- read a string from a file
\begverb
Usage: (count,pointer) = ReadStr(IoBuff,buffer,length)
         D0    A1                  A0     A1     D0
#endverb%
Reads characters from the file specified by the {\bf IoBuff} pointer until
a ``newline'' character is found.
The ``newline'' is not included in the returned string.  
The return value is the actual number of characters read, 
or -1 if an error occurred.
\idxline{ReadStr(),}{Systems Library function}
\idxline{IoBuff structure}{}

\seealso{ReadF()}%

\subsub{RemClipNode()}%
-- unlink and deallocate a list Clip node.
\begverb
Usage: RemClipNode(node)
                    A0
#endverb%
Unlinks and releases the specified Clip node.
The function is the ``auto-delete'' function for Clip nodes,
and will be called automatically by |RemRsrcNode()| or |RemRsrcList()|.
\idxline{RemClipNode(),}{Systems Library function}

\seealso{AddClipNode(), RemRsrcList(), RemRsrcNode()}%

\subsub{RemRsrcList()}%
-- unlink and deallocate a list of resource nodes
\begverb
Usage: RemRsrcList(list)
                    A0
#endverb%
Scans the supplied list and releases any nodes found.
The list must consist of resource nodes ({\bf RexxRsrc} structures), 
which contain information to allow automatic cleanup and deletion.
\idxline{RemRsrcList(),}{Systems Library function}

\seealso{RemRsrcNode()}%

\subsub{RemRsrcNode()}%
-- unlink and deallocate a resource node
\begverb
Usage: RemRsrcNode(node)
                    A0
#endverb%
Unlinks and releases the specified resource node, 
including the name string if one is present.  
If an ``auto-delete'' function has been specified in the node, 
it is called to perform any required resource deallocation before the 
node is released.
\idxline{RemRsrcNode(),}{Systems Library function}

\seealso{RemRsrcList()}%

\subsub{SeekF()}%
-- seek to the specified position in a file.
\begverb
Usage: position = SeekF(IoBuff,offset,anchor)
          D0              A0     D0     D1
#endverb%
Seeks to a new position in the file is specified by the \hp{IoBuff} pointer.
The position is given by the \hp{offset} argument,
a byte offset relative to the supplied \hp{anchor} argument.
The anchor may specify the beginning (-1), the current position (0), 
or the end of the file (1).
The return value is the new position relative to the beginning of the file.
\idxline{SeekF(),}{Systems Library function}
\idxline{IoBuff structure}{}

\subsub{StackF()}%
-- stack a line to a file buffer.
\begverb
Usage: count = StackF(IoBuff,buffer,length)
         D0             A0     A1     D0
#endverb%
Stacks a buffer of characters in the stream associated with the {\bf IoBuff}
structure.
The {\it buffer} argument is a pointer to a string of characters, 
and the {\it length} specifies the number of characters to be stacked.
The return value is the actual count of characters or -1 if an error occurred.

Stacked lines are placed in ``last-in, first-out'' order and are immediately
available to be read from the stream.
The stream must be managed by a DOS handler that supports the |ACTION\_STACK|
packet.
\idxline{StackF(),}{Systems Library function}
\idxline{IoBuff structure}{}
\seealso{QueueF()}%

\subsub{StcToken()}%
-- break out the next token from a string
\begverb
Usage: (quote,length,scan,token) = StcToken(string)
         D0     D1    A0   A1                 A0
#endverb%
Scans a null-terminated string to select the next token delimited by 
``white space,'' 
and returns a pointer to the start of the token.  
The {\it quote} return will be an ASCII single or double quote if 
the token was quoted and 0 otherwise; 
white space characters are ignored within quoted strings.  
The {\it length} return is the total length of the token,
including any quote characters.  
The {\it scan} return is advanced beyond the current token to prepare for the 
next call.
\idxline{StcToken(),}{Systems Library function}

\subsub{StrcpyA()}%
-- copy a string, converting to ASCII
\begverb
Usage: hash = StrcpyA(destination,source,length)
        D0                A0        A1     D0
#endverb%
Copies the {\it source} string to the {\it destination} area, 
converting the characters to ASCII by clearing the high-order bit of each byte.
The {\it length} of the string (which may include embedded nulls) is 
considered as a 2-byte unsigned integer.
so the string is limited in length to 65,535 bytes.  
The {\it hash} return is the internal hash byte for the copied string.
\idxline{StrcpyA(),}{Systems Library function}
\seealso{StrcpyN(), StrcpyU}%

\subsub{StrcpyN()}%
-- copy a string
\begverb
Usage: hash = StrcpyN(destination,source,length)
        D0                A0        A1     D0
#endverb%
Copies the {\it source} string to the {\it destination} area.
The {\it length} of the string (which may include embedded nulls) is 
considered as a 2-byte unsigned integer. 
The {\it hash} return is the internal hash byte for the copied string.
\idxline{StrcpyN(),}{Systems Library function}
\seealso{StrcpyA(), StrcpyU}%

\subsub{StrcpyU()}%
-- copy a string, converting to uppercase
\begverb
Usage: hash = StrcpyU(destination,source,length)
        D0                A0        A1     D0
#endverb%
Copies the {\it source} string to the {\it destination} area,
converting to uppercase alphabetics.
The {\it length} of the string (which may include embedded nulls) is 
considered as a 2-byte unsigned integer.
The {\it hash} return is the internal hash byte for the copied string.
\idxline{StrcpyU(),}{Systems Library function}
\seealso{StrcpyA(), StrcpyN}%

\subsub{StrflipN()}%
-- reverse the characters in a string
\begverb
Usage: StrflipN(string,length)
                  A0     D0
#endverb%
Reverses the sequence of characters in a string.
The conversion is performed in place.
\idxline{StrflipN(),}{Systems Library function}

\subsub{Strlen()}%
-- find the length of a null-terminated string
\begverb
Usage: length = Strlen(string)
         D0              A0
        (CCR)
#endverb%
Returns the number of characters in a null-terminated string.  
Register A0 is preserved,
and the CCR is set for the returned length.
\idxline{Strlen(),}{Systems Library function}

\subsub{StrcmpN()}%
-- compare the values of strings
\begverb
Usage: test = StrcmpN(string1,string2,length)
        D0               A0      A1     D0
       (CCR)
#endverb%
The {\it string1} and {\it string2} arguments are compared for the specified 
number of characters.
The comparison proceeds character-by-character until a difference is found 
or the maximum number of characters have been examined.
The returned value is -1 if the first string was less, 
1 if the first string was greater,
and 0 if the strings match exactly.  
The CCR register is set for the returned value.
\idxline{StrcmpN(),}{Systems Library function}

\subsub{ToUpper()}%
-- translate an ASCII character to uppercase
\begverb
Usage: upper = ToUpper(character)
        D0                D0
#endverb%
Converts an ASCII character to uppercase.  Only register D0 is affected.
\idxline{ToUpper(),}{Systems Library function}

\subsub{UnlockRexxBase()}%
-- unlock a shared resource.
\begverb
Usage:	UnlockRexxBase(resource)
                          D0
#endverb%
Releases the specified resource, or all resources if the argument is zero.
Every call to |LockRexxBase()| should be followed eventually by a call to 
|UnlockRexxBase()| for the same resource.
\idxline{UnlockRexxBase(),}{Systems Library function}
\seealso{LockRexxBaseF()}%

\subsub{WriteF()}%
-- write characters to a file buffer
\begverb
Usage: count = WriteF(IoBuff,buffer,length)
         D0             A0     A1     D0
        (CCR)
#endverb%
Writes a buffer of characters of the specified length to the file associated
with the {\bf IoBuff} pointer.
The {\it buffer} argument is a pointer to a storage area, 
and the {\it length} argument specifies the number of characters to be written.
The returned value is the actual number of characters written or -1 if an 
error occurred.
\idxline{WriteF(),}{Systems Library function}
\idxline{IoBuff structure}{}
\seealso{CloseF(), OpenF(), ReadF()}%

\thatsit
