%
%	The ARexx User's Manual
%
% Copyright � 1987 by William S. Hawes.  All Rights Reserved.
%
\ifx\radeye\fmtversion\subdoc\else\input pmac \fi

\chapter{Appendix}{B}{Limits and Compatibility}%
\AR\ was designed to adhere closely to the REXX language standard.
This appendix discusses those areas where \AR\ departs from the standard.

\section{\chapterno-}{Limits}%
Language definitions seldom include predefined limits to the program
structures that can be created.  
Only a few such restrictions were imposed in implementing \AR,
and most of the internal structures are limited only by the total amount 
of memory available.
The current implementation limits are listed below.

\startlist
\item{$\bullet$} {\it Length of Strings}.
Strings, symbol names, and value strings are limited to a maximum length 
of 65,535 bytes.

\item{$\bullet$} {\it Length of Clauses}.
Clauses are limited to a maximum of 800 characters after removing comments 
and multiple blanks.

\item{$\bullet$} {\it Nodes in Compound Names}.
Compound symbol names may include a maximum of 50 nodes, including the stem.

\item{$\bullet$} {\it Arguments to Functions}.
Built-In and external functions are limited to a maximum of 15 arguments.
There is no limit to the number of arguments that may be passed to an 
internal function.

\item{$\bullet$} {\it Subexpression Nesting}.
The maximum nesting level for subexpressions is 32.
\idxline{nesting,}{subexpression limit}
\endlist

\section{\chapterno-}{Compatibility}%
\AR\ departs in a few ways from the language definition.  
The differences can be classified as omissions or extensions, 
and are described below.

\subsub{Omissions}.
The only significant specification of the language standard omitted from this
implementation is the arbitrary-precision arithmetic facility.
Arithmetic operations are limited to about 14 digits of precision, 
and the |FUZZ| option is not implemented at all.
Only the |SCIENTIFIC| format is used for exponential notation.
The full numeric capabilities will be provided in a later release.
\idxline{omissions,}{from REXX standard}

\subsub{Extensions}.
The following extensions to the language standard have been included in 
this implementation:
\idxline{extensions,}{to REXX standard}

\item{$\bullet$}{\it |BREAK| Instruction}.
A new instruction called |BREAK| has been implemented.  
It is used to exit from the scope of any |DO| or |INTERPRET| instruction.
\idxline{BREAK instruction}{}

\item{$\bullet$}{\it |ECHO| Instruction}.
The |ECHO| instruction has been included as a synonym for |SAY|.
\idxline{ECHO instruction}{}

\item{$\bullet$}{\it |SHELL| Instruction}.
The |SHELL| instruction has been included as a synonym for |ADDRESS|.
\idxline{SHELL instruction}{}

\item{$\bullet$}{\it |SIGNAL| Options}.
Several additional |SIGNAL| keywords have been implemented.
|BREAK\_C|, |BREAK\_D|, |BREAK\_E|, and |BREAK\_F| will detect and trap the 
control-C through control-F signals passed by AmigaDOS.
The |IOERR| keyword traps errors detected by the I/O system.
\idxline{SIGNAL instruction}{}

\item{$\bullet$}{\it Stem Symbols}.
A stem symbol is valid anywhere that a simple symbol could be employed.

\item{$\bullet$}{\it Template Processing}.
Templates have been generalized in several ways.
Variable symbols may be used as positional tokens if preceded by an operator;
the ``|=|'' operator is used to denote an absolute position.  
Multiple templates can be used with all source forms of the |PARSE|
instruction.

\thatsit
