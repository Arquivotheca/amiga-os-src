%
%	The ARexx User's Manual
%
% Copyright � 1987 by William S. Hawes.  All Rights Reserved.
%
\ifx\radeye\fmtversion\subdoc\else\input pmac \fi

\chapter{Chapter}{9}{The Resident Process}%
This chapter describes some of the capabilities of the \AR\ 
\hp{resident process},
a global communications and resources manager.  
The material presented here is directed to the general user; 
Chapter 10 covers these topics in greater depth for software developers 
who wish to integrate \AR\ with other applications programs.
\idxline{resident process}{,capabilities}

The resident process must be active before any \AR\ programs can be run.
It announces its presence to the system by opening a public message port 
named ``REXX,''  
so applications programs that use \AR\ should check for the presence of 
this port.
If the port is not open, 
the user can either be informed that the macro processor is not available, 
or else the applications program can start up the resident process.
The latter option can be done using the |rexxmast| command.
\idxline{message ports,}{REXX}

The primary function of the resident process is to launch \AR\ programs.  
When an applications program sends a ``command'' or ``function'' message to 
the ``REXX'' port,
the resident process creates a new DOS process to execute the program,
and forwards the invocation message to newly created process.  
It also creates a new instance of the \AR\ global data structure,
which links together all of the structures manipulated by the program.

In addition to launching programs,
the resident process manages various resources used by \AR.  
These resources include a list of available function libraries called the 
Library List,
a list of (name,value) pairs called the Clip List,
and a list of the currently active \AR\ programs.
Built-In functions are available to manipulate the Library List and Clip List 
from within an \AR\ program. 
Applications programs can modify a resource list either by sending a request 
packet to the resident process or by direct manipulation of the list.
\idxline{Clip List}{}
\idxline{Library List}{}
\idxline{resident process}{,resources managed}

\section{\chapterno-}{Command Utilities}%
\AR\ is supplied with a number of command utilities to provide various 
control functions.
These are executable modules that can be run from the CLI, 
and should reside in the system command (|C:|) directory for convenience.
These commands are relevant only when the \AR\ resident process is active.

The functions performed by these utilities may also be available from
within an applications program.
All of the utilities are implemented by sending message packets to the 
resident process, 
so an application designed to work closely with \AR\ could easily provide 
these functions as part of its control menu.

\subsection{HI}%
\usage{|HI|}
Sets the global halt flag,
which causes all active programs to receive an external halt request.
Each program will exit immediately unless its |HALT| interrupt has been 
enabled.
The halt flag does not remain set, 
but is cleared automatically after all current programs have received 
the request.
\idxline{HI command}{}
\idxline{HALT}{,global flag}

\subsection{RX}%
\usage{|RX| name [arguments]} 
This command launches an \AR\ program.  
If the specified {\it name} includes an explicit path,
only that directory is searched for the program;
otherwise, the current directory and the system |REXX:| device are checked
for a program with the given name.  
The optional argument string is passed to the program.
\idxline{RX command}{}

\subsection{RXSET}%
\usage{|RXSET| name value}
Adds a (name,value) pair to the Clip List.
Name strings are assumed to be in mixed case.
If a pair with the same name already exists, 
its value is replaced with the current string.
If a name without a value string is given, 
the entry is removed from the Clip List.
\idxline{RXSET command}{}
\idxline{Clip List,}{adding entries}

\subsection{RXC}%
\usage{|RXC|}
Closes the resident process.
The ``REXX'' public port is withdrawn immediately,
and the resident process exits as soon as the last \AR\ program finishes.
No new programs can be launched after a ``close'' request.
\idxline{RXC command}{}
\idxline{resident process,}{closing}

\subsection{TCC}%
\usage{|TCC|}
Closes the global tracing console as soon as all active programs are no 
longer using it.  
All read requests queued to the console must be satisfied before it can be 
closed.
\idxline{TCC command}{}
\idxline{trace console,}{closing}

\subsection{TCO}%
\usage{|TCO|}
Opens the global tracing console.
The tracing output from all active programs is diverted automatically to
the new console.  
The console window can be moved and resized by the user,
and can be closed with the ``|TCC|'' command.
\idxline{TCO command}{}
\idxline{trace console,}{opening}

\subsection{TE}%
\usage{|TE|}
Clears the global tracing flag,
which forces the tracing mode to |OFF| for all active \AR\ programs.
\idxline{TE command}{}
\idxline{TRACE}{,global flag}

\subsection{TS}%
\usage{|TS|} 
Starts interactive tracing by setting the external trace flag,
which forces all active \AR\ programs into interactive tracing mode.  
Programs will start producing trace output and will pause after the next 
statement.
This command is useful for regaining control over programs caught in 
infinite loops or otherwise misbehaving.  
The trace flag remains set until cleared by the |TE| command,
so subsequent program invocations will begin executing in interactive 
tracing mode.
\idxline{TS command}{}

\section{\chapterno-}{Resource Management}%
Individual \AR\ programs manage their internal memory allocation and I/O 
file resources,
but some resources need to be available to all programs.  
The management of these global resources is one of the major functions 
of the resident process.  
Global resources are maintained as doubly-linked lists, 
in keeping with the general design principles of the EXEC operating system.
Linked lists provide a flexible and open mechanism for resource management, 
and help avoid the built-in limits common with other approaches.

\subsection{The Global Tracing Console}%
The tracing output from an \AR\ program usually goes to the standard
output stream |STDOUT|,
and is therefore interleaved with the normal output of the program.
Since this may be confusing at times, 
a global trace console can be opened to display only tracing output.  
The console can be opened using the |tco| command utility or by sending 
an |RXTCOPN| request packet to the resident process.  
\AR\ programs will automatically divert their tracing output to the new window,
which is opened as a standard AmigaDOS console. 
The user can move it and resize it as required.
\idxline{TRACE}{,opening console}

The tracing console also serves as the input stream for programs during 
interactive tracing. 
When a program pauses for tracing input,
the input line must be entered at the trace console.  
Any number of programs may use the tracing console simultaneously, 
although it is generally recommended that only one program at a time be traced.

The tracing console can be closed using the |tcc| command or by sending 
an |RXTCCLS| request packet to the resident process.  
The closing is delayed until all read requests to the console have 
been satisfied.  
Only when all of the active programs indicate that they are no longer 
using the console will it actually be closed.
\idxline{TCC command}{}
\idxline{TRACE}{,closing console}

\subsection{The Library List}%
The resident process maintains a Library List of the \hp{function libraries}
and \hp{function hosts} currently available to \AR\ programs.
This list is used to resolve all references to external functions.
Each entry has an associated search priority in the range 100 to -100,
with the higher-valued entries being searched first until the requested
function is found.
The list is searched by calling each entry, 
using the appropriate protocol,
until the return code indicates that the function was found.
\idxline{Library List}{}
\idxline{function libraries,}{in Library List}
\idxline{function hosts,}{in Library List}

The two types of entities maintained by the list are quite different in some
respects,
but the ultimate way in which a function call is resolved is transparent
to the calling program.
A function library is a collection of functions organized as an Amiga
shared library,
while a function host is a separate task that manages a message port.
Function libraries are called as part of the \AR\ interpreter's task context,
but calls to function hosts are mediated by passing a message packet.
The \AR\ resident process is itself a function host, 
and is installed in the Library List at a priority of -60.

The resident process provides addition and deletion operations for maintaining
the Library List; 
these operations are performed by sending an appropriate message packet.
The Library List is always maintained in priority order.
Within a given priority level any new entries are added to the end of the 
chain,
so that entries added first will be searched first.
The priority levels are significant if any of the libraries have duplicate 
function name definitions,
since the function located further down the search chain could never be 
called.
\vskip-\lastskip
\idxline{Library List,}{adding entries}
\idxline{Library List,}{deleting entries}
\idxline{priority,}{Library List}

\subsub{Function Libraries}.
Each function library entry in the Library List contains a library name, 
a search priority, an entry point offset, and a version number.
The library name must refer to a standard Amiga shared library residing in
the system |LIBS:| directory so that it can be loaded when needed.
Function libraries can be created and maintained by users or applications 
developers;
Chapter 10 has information on their design and implementation.

The ``query'' function is the library entry point that is actually called 
by the interpreter.
It must be specified as an integer offset (e.g. ``-30'') from the library base.
The return code from the query call then indicates whether the desired 
function was found;
if it was, the function is called with the parameters passed by the 
interpreter and the function result is returned to the caller.  
Otherwise, the search continues with the next entry in the list.  
In either event the library is closed to await the next call. 
\idxline{function libraries,}{query function}

A note of caution:  not every Amiga shared library can be used as a function 
library.
Function libraries must have a special entry point to perform the dynamic
linking required to access the functions from within \AR.  
Each library should include documentation providing its version number 
and the integer offset to its ``query'' entry point.

\subsub{Function Hosts}.
The name associated with a function host is the name of its public message
port.
Function calls are passed to the host as a message packet;
it is then up to the individual host to determine whether the specified
function name is one that it recognizes.
The name resolution is completely internal to the host,
so function hosts provide a natural gateway mechanism for implementing 
remote procedure calls to other machines in a network.

\subsection{The Clip List}%
The Clip List maintains a set of (name,value) pairs that may be used for a
variety of purposes.
Each entry in the list consists of a name and a value string,
and may be located by name.
Since the Clip List is publicly accessible,
it may be used as a general clipboard-like mechanism for intertask
communication.
In general, the names used should be chosen to be unique to an application
to prevent collisions with other programs.
Any number of entries may be posted to the list.
\idxline{Clip List}{}

One potential application for the Clip List is as a mechanism for loading 
predefined constants into an \AR\ program.  
The language definition does not include a facility comparable to the 
``header file'' preprocessor in the ``C'' language.
However, consider a string in the Clip List of the form
\begprog
pi=3.14159; e=2.718; sqrt2=1.414 ...
#endverb
i.e., a series of assignments separated by semicolons.   
In use, such a string could be retrieved by name using the Built-In function 
|GETCLIP()| and then |INTERPRET|ed within the program.  
The assignment statements within the string would then create the required 
constant definitions.
The following program fragment illustrates the process:

\begprog
/* assume a string called "numbers" is available*/
numbers = getclip('numbers')  /* case-sensitive */ 
interpret numbers             /* ... assignments*/
...
#endverb%
More generally,
the strings would not be restricted to contain only assignment statements,
but could include any valid \AR\ statements.  
The Clip List could thus provide a series of programs for initializations 
or other processing tasks.

The resident process supports addition and deletion operations for 
maintaining the Clip List.  
The names in the (name,value) pairs are assumed to be in mixed case,
and are maintained to be unique in the list.  
An attempt to add a string with an existing name will simply update the 
value string.
The name and value strings are copied when an entry is posted to the list, 
so the program that adds an entry is not required to maintain the strings.

Entries posted to the Clip List remain available until explicitly removed.
The Clip List is automatically released when the resident process exits.

\thatsit
