%
%	The ARexx User's Manual
%
% Copyright � 1987 by William S. Hawes.  All Rights Reserved.
%
\ifx\radeye\fmtversion\subdoc\else\input pmac \fi

\chapter{Chapter}{7}{Tracing and Interrupts}%
\AR\ provides tracing and source-level debugging facilities that are unusual
in a high-level language.  
\hp{Tracing} refers to the ability to display selected statements in a 
program as the program executes.  
When a clause is traced, its line number, source text,
and related information are displayed on the console.  
The tracing action of the interpreter is determined by a \hp{trace option}
that selects which source clauses will be traced,
and two modifier flags that control \hp{command inhibition} and 
\hp{interactive tracing}.

The internal \hp{interrupt system} enables an \AR\ program to detect certain
synchronous or asynchronous events and to take special actions when they occur.
Events such as a syntax error or an external halt request that would normally 
cause the program to exit can instead be trapped so that corrective actions
can be taken.

\section{\chapterno-}{Tracing Options}%
Trace options are sometimes called an alphabetic options,
since the keywords that select an option can be shortened to one letter for 
convenience.  
The alphabetic options are:
\idxline{tracing,}{alphabetic options}
\idxline{alphabetic option}{}

\startlist
\item{$\bullet$} |ALL|.
All clauses are traced.  
\idxline{ALL,}{trace option}

\item{$\bullet$} |COMMANDS|.  
All command clauses are traced before being sent to the external host.
Non-zero return codes are displayed on the console.
\idxline{COMMANDS,}{trace option}

\item{$\bullet$} |ERRORS|.
Commands that generate a non-zero return code are traced after the 
clause is executed.
\idxline{ERRORS,}{trace option}

\item{$\bullet$} |INTERMEDIATES|.
All clauses are traced, 
and intermediate results are displayed during expression evaluation.
These include the values retrieved for variables, 
expanded compound names, and the results of function calls.
\idxline{INTERMEDIATES,}{trace option}

\item{$\bullet$} |LABELS|.
All label clauses are traced as they are executed.
A label will be displayed each time a transfer of control takes place.
\idxline{INTERMEDIATES,}{trace option}

\item{$\bullet$} |NORMAL|.
Command clauses with return codes that exceed the current error failure level
are traced after execution,
and an error message is displayed.
This is the default trace option. 
\idxline{NORMAL,}{trace option}

\item{$\bullet$} |RESULTS|.
All clauses are traced before execution,
and the final result of each expression is displayed.
Values assigned to variables by |ARG|, |PARSE|, or |PULL| instructions 
are also displayed.
\idxline{RESULTS,}{trace option}

\item{$\bullet$} |SCAN|.
This is a special option that traces all clauses and checks for errors,
but suppresses the actual execution of the statements.  
It is helpful as a preliminary screening step for a newly-created program.
\idxline{SCAN,}{trace option}
\endlist

The tracing mode can be set using either the |TRACE| instruction or the 
|TRACE()| Built-In function.
The |RESULTS| trace option is recommended for general-purpose testing.
Tracing can be selectively disabled from within a program so that 
previously-tested parts of a program can be skipped.
\idxline{TRACE(),}{built-in function}

\section{\chapterno-}{Display Formatting}%
Each trace line displayed on the console is indented to show the effective 
control (nesting) level at that clause,
and is identified by a special three-character code, 
as shown in Table 7.1 below.
The source for each clause is preceded by its line number in the program.
Expression results or intermediates are enclosed in double quotes so that 
leading and trailing blanks will be apparent.
\idxline{display formatting,}{during tracing}

$$\vbox{\halign{\tt #\hfil&&\qquad#\hfil\cr
\multispan2\hfil Table \number\chapterno .1  Tracing Prefix Codes\hfil\cr
\bf Code&\bf Displayed Values\cr
+++&Command or syntax error\cr
>C>&Expanded compound name\cr
>F>&Result of a function call\cr
>L>&Label clause\cr
>O>&Result of a dyadic operation\cr
>P>&Result of a prefix operation\cr
>U>&Uninitialized variable\cr
>V>&Value of a variable\cr
>>>&Expression or template result\cr
>.>&``Placeholder'' token value\cr}}$$

\subsection{Tracing Output}%
The tracing output from a program is always directed to one of two logical 
streams.
The interpreter first checks for a stream named |STDERR|, 
and directs the output there if the stream exists.
Otherwise the trace output goes to the standard output stream |STDOUT| and
will be interleaved with the normal console output of the program.
The |STDERR| and |STDOUT| streams can be opened and closed under 
program control,
so the programmer has complete control over the destination of tracing output.
\idxline{output,}{tracing}
\idxline{Stream,}{STDERR}
\idxline{Stream,}{STDOUT}

In some cases a program may not have a predefined output stream.
For example, a program invoked from a host application that did not provide
input and output streams would not have an output console.
To provide a tracing facility for such programs,
the resident process can open a special \hp{global tracing console} for use 
by any active program.
When this console opens,
the interpreter automatically opens a stream named |STDERR| for each \AR\
program in which |STDERR| is not currently defined,
and the program then diverts its tracing output to the new stream.
\idxline{Tracing,}{global console}

The global console can be opened and closed using the command utilities |tco| 
and |tcc|, respectively.
The console may not close immediately upon request, however.
The resident process waits until all active programs have diverted their
tracing streams back to the default state before actually closing the console.
Applications programs may provide direct control over the tracing console
by sending request packets to the resident process, 
which is discussed in Chapter 10.
\idxline{TCO command}{}
\idxline{TCC command}{}

The trace stream (|STDERR| or |STDOUT|) is also used for trace input,
so a program in interactive tracing mode will wait for user input from 
this console.
The global tracing console is always shared among all currently active 
programs.
Since it may be confusing to have several programs simultaneously writing
writing to the same console,
it is recommended that only one program at a time be traced using the 
global console. 

\subsection{Command Inhibition}%
\AR\ provides a tracing mode called \hp{command inhibition} that suppresses
host commands.
In this mode command clauses are evaluated in the normal manner,
but the command is not actually sent to the external host,
and the return code is set to zero.  
This provides a way to test programs that issue potentially destructive 
commands,
such as erasing files or formatting disks.
Command inhibition does not apply to command clauses that are entered
interactively.
These commands are always performed,
but the value of the special variable |RC| is left unchanged.
\idxline{command inhibition}{}
\idxline{RC special variable,}{with command inhibition}

Command inhibition may be used in conjunction with any trace option.
It is controlled by the ``|!|'' character,
which may appear by itself or may precede any of the alphabetic options 
in a |TRACE| instruction.  
Each occurrence of the ``|!|'' character ``toggles'' the inhibition mode
currently in effect.
Command inhibition is cleared when tracing is set to |OFF|.

\section{\chapterno-}{Interactive Tracing}%
Interactive tracing is a debugging facility that allows the user to enter 
source statements while a program is executing.
These statements may be used to examine or modify variable values,
issue commands, or otherwise interact with the program.
Any valid language statements can be entered interactively, 
with the same rules and restrictions that apply to the |INTERPRET| instruction.
In particular,
compound statements such as |DO| and |SELECT| must be complete within the 
entered line.
\idxline{Tracing,}{interactive}
\idxline{INTERPRET instruction,}{with interactive tracing}

Interactive tracing can be used with any of the trace options.
While in interactive tracing mode,
the interpreter pauses after each traced clause and prompts for input with
the code ``|+++|.''
At each pause, 
three types of user responses are possible:

\startlist
\item{$\bullet$} If a null line is entered,
the program continues to the next pause point.

\item{$\bullet$} If an ``|=|'' character is entered,
the preceding clause is executed again.

\item{$\bullet$} Any other input is treated as a debugging statement,
and is scanned and executed.
\endlist

The pause points during interactive tracing are determined by the tracing 
option currently in effect,
as the interpreter pauses only after a traced clause.  
However, certain instructions cannot be safely (or sensibly) re-executed,
so the interpreter will not pause after executing one of these.
The ``no-pause'' instructions are |CALL|, |DO|, |ELSE|, |IF|, |THEN|, 
and |OTHERWISE|.  
The interpreter will also not pause after any clause that generated an
execution error.
\idxline{no-pause instructions}{}

Interactive tracing mode is controlled by the ``|?|'' character, 
either by itself or in combination with an alphabetic trace option.  
Any number of ``|?|'' characters may precede an option, 
and each occurrence toggles the mode currently in effect.  
For example, if the current trace option was |NORMAL|, 
then ``|TRACE ?R|'' would set the option to |RESULTS| and select interactive 
tracing mode.
A subsequent ``|TRACE ?|'' would turn off interactive tracing.

\subsection{Error Processing}%
The \AR\ interpreter provides special error processing while it executes
debugging statements.  
Errors that occur during interactive debugging are reported, 
but do not cause the program to terminate.
This special processing applies only to the statements that were entered
interactively.
Errors occurring in the program source statements are treated in the 
usual way whether or not the interpreter is in interactive tracing mode.  
\idxline{Error processing,}{during interactive tracing}

In addition to the special error processing,
the interpreter also disables the internal interrupt flags during
interactive debugging.
This is necessary to prevent an accidental transfer of control due to an 
error or uninitialized variable.  
However, if a ``|SIGNAL label|'' instruction is entered,
the transfer will take place,
and any remaining interactive input will be abandoned.
The |SIGNAL| instruction can still be used to alter the interrupt flags, 
and the new settings will take effect when the interpreter returns to 
normal processing.
\idxline{SIGNAL instruction,}{with interactive tracing}

\subsection{The External Tracing Flag}%
The \AR\ resident process maintains an \hp{external tracing flag} that 
can be used to force programs into interactive tracing mode.  
The tracing flag can be set using the |ts| command utility.  
When the flag is set, 
any program not already in interactive tracing mode will enter it immediately.
The internal trace option is set to |RESULTS| unless it is currently set 
to |INTERMEDIATES| or |SCAN|,
in which case it remains unchanged.  
Programs invoked while the external tracing flag is set will begin executing 
in interactive tracing mode.
\idxline{External Tracing flag}{}
\idxline{TS command utility}{}

The external tracing flag provides a way to regain control over programs that 
are caught in loops or are otherwise unresponsive.  
Once a program enters interactive tracing mode,
the user can step through the program statements and diagnose the problem.  
There is one caveat, though: external tracing is a global flag,
so all currently-active programs are affected by it.
The tracing flag remains set until it is cleared using the ``|te|'' command 
utility.  
Each program maintains an internal copy of the last state of the tracing flag,
and sets its tracing option to |OFF| when it observes that the tracing flag 
has been cleared.
\idxline{TE command utility}{}

\section{\chapterno-}{Interrupts}%
\AR\ maintains an internal interrupt system that can be used to detect 
and trap certain error conditions.
When an interrupt is enabled and its corresponding condition arises, 
a transfer of control to the label specific to that interrupt occurs.  
This allows a program to retain control in circumstances that might otherwise
cause the program to terminate.
The interrupt conditions can caused by either \hp{synchronous} events like
a syntax error, 
or \hp{asynchronous} events like a ``control-C'' break request. 
Note that these internal interrupts are completely separate from the 
hardware interrupt system managed by the EXEC operating system.
\idxline{Interrupts}{}
\idxline{Interrupts,}{EXEC suupported}

The interrupts supported by \AR\ are described below.
The name assigned to each is actually the label to which control will be 
tranferred.
Thus, a |SYNTAX| interrupt will transfer control to the label ``|SYNTAX:|.''
Interrupts can be enabled or disabled using the |SIGNAL| instruction.
For example,
the instruction ``|SIGNAL ON SYNTAX|'' would enable the |SYNTAX| interrupt.

\startlist
\item{$\bullet$} |BREAK\_C|.
This interrupt will trap a control-C break request generated by DOS.
If the interrupt is not enabled, 
the program terminates immediately with the error message 
``|Execution halted|'' and returns with the error code set to |2|.
\idxline{BREAK\_C Interrupt}{}

\item{$\bullet$} |BREAK\_D|.
The interrupt will detect and trap a control-D break request issued by DOS.
The break request is ignored if the interrupt is not enabled.
\idxline{BREAK\_D Interrupt}{}

\item{$\bullet$} |BREAK\_E|.
The interrupt will detect and trap a control-E break request issued by DOS.
The break request is ignored if the interrupt is not enabled.
\idxline{BREAK\_E Interrupt}{}

\item{$\bullet$} |BREAK\_F|.
The interrupt will detect and trap a control-F break request issued by DOS.
The break request is ignored if the interrupt is not enabled.
\idxline{BREAK\_F Interrupt}{}

\item{$\bullet$} |ERROR|.
This interrupt is generated by any host command that returns a non-zero code.
\vskip-\lastskip
\idxline{ERROR Interrupt}{}

\item{$\bullet$} |HALT|.  
An external halt request will be trapped if this interrupt is enabled.
Otherwise, the program terminates immediately with the error message
``|Execution halted|'' and returns with the error code set to |2|.
\vskip-\lastskip
\idxline{HALT Interrupt}{}

\item{$\bullet$} |IOERR|.
Errors detected by the I/O system will be trapped if this interrupt is
enabled.  
\vskip-\lastskip
\idxline{IOERR Interrupt}{}

\item{$\bullet$} |NOVALUE|.
An interrupt will occur if an uninitialized variable is used while this 
condition is enabled.
The usage could be within an expression, in the |UPPER| instruction,
or with the |VALUE()| built-in function.
\vskip-\lastskip
\idxline{NOVALUE Interrupt}{}

\item{$\bullet$} |SYNTAX|.
A syntax or execution error will generate this interrupt.  
Not all errors such errors can be trapped, however.
In particular, 
certain errors occurring before a program is actually executing, 
and those detected by the \AR\ external interface,
cannot be trapped by the |SYNTAX| interrupt.
\idxline{SYNTAX Interrupt}{}
\endlist

When an interrupt forces a transfer of control,
all of the currently active control ranges are dismantled,
and the interrupt that caused the transfer is disabled.
This latter action is necessary to prevent a possible recursive interrupt loop.
Only the control structures in the current environment are affected,
so an interrupt generated within a function will not affect the caller's
environment.

\subsub{Special Variables}.
Two special variables are affected when an interrupt occurs.  
The variable |SIGL| is always set to the current line number before the 
transfer of control takes place,
so that the program can determine which source line was being executed.  
When an |ERROR| or |SYNTAX| interrupt occurs,
the variable |RC| is set to the error code that caused the condition.  
For |ERROR| interrupts this value will be a command return code, 
and can usually be interpreted as an error severity level.  
The value for |SYNTAX| interrupts is always an \AR\ error code.
\idxline{SIGL Special variable,}{with interrupts}
\idxline{RC Special variable,}{with interrupts}

Interrupts are useful primarily to allow a program to take special 
error-recovery actions.  
Such actions might involve informing external programs that an error occurred, 
or simply reporting further diagnostics to help in isolating the problem.  
In the following example,
the program issues a ``message'' command to an external host called ``MyEdit''
whenever a syntax error is detected:

\begprog
/* A macro program for `MyEdit'                 */
signal on syntax        /* enable interrupt     */
  .
  . (normal processing)
  .
exit 
syntax:                 /* syntax error detected*/
   address `MyEdit'
   `message' `error' rc errortext(rc)
   exit 10
#endverb

\thatsit
