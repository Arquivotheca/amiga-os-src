%
%   This is a set of macros for the ARexx user's manual.
%
%   Here we make sure that this file is only read in once.  Each subfile
%   should start with \ifx\radeye\fmtversion\subdoc\else\input pmac \fi
%   and end with \thatsit.
%
\let\radeye=\fmtversion
%
%   The following are document style parameters, and these are what
%   should be changed if the document isn't pretty.
%
\magnification=\magstephalf
\hsize=6truein\hoffset=0.25truein
\raggedbottom
\parskip=\baselineskip
%
%   Define a couple of fonts
%
\font\bigrm=cmbx10 scaled 1577
\font\bigit=cmbxti10 scaled 1577
\font\cfont=cmbx10 scaled 1312
%
%   Some character definitions ...
%
\def\&{\char38}
\def\'{\char39}
\def\^{\char94}
\def\{{\char123}
\def\|{\char124}
\def\}{\char125}
\def\~{\char126}
{\obeyspaces\gdef {\ }}
%
%	Some counters and files
%
\newcount\secno\newcount\subsecno\newcount\tableno\newcount\figno
\newread\tocf\newwrite\tocfout
\newread\indx\newwrite\indxout
%
\def\AR{ARexx}
\def\DOC{ARexx User's Reference Manual}
\def\vb{\vskip\baselineskip}
\def\page{\vfill\eject}
\def\blankpage{\vfill{\centerline{\ }}\page}
%
%   The following macros allow you to do:
%   \begverb
%   verbatim stuff
%   #endverb
%
\def\begverb{
   \begingroup
   \def\par{\leavevmode\endgraf\nobreak}
   \catcode`\{=12\catcode`\}=12\catcode`\$=12\catcode`\&=12
   \catcode`\%=12\catcode`\~=12\catcode`\_=12\catcode`\^=12
   \catcode`\#=0
%  \catcode`\\=12
   \tt\parindent=0pt\parskip=0pt\obeyspaces\obeylines}
\def\endverb{\endgroup\noindent}
%
%   When typesetting code, just leave the parindent alone.
%   We also don't want any paragraph breaks.
%
\def\begprog{\smallbreak\begverb\parindent=20pt}
%
%   For examples, we need a similar macro.
%
\def\begexamp{\leftline{Example:}\nobreak\begverb\parindent=20pt}
\def\begexamps{\leftline{Examples:}\nobreak\begverb\parindent=20pt}
%
%   We need to allow || to encompass typewriter text.  The extra
%   set of curly braces make : be ignored to eliminate the double
%   spaces.
%
\catcode`\|=\active\def\typethis#1|{{\tt #1}{}}\let|=\typethis
%
%   A macro for highlighting new terms in the text.
%
\def\hp#1{{\it #1}}
%
%   The header and footer macros.  These are again more complex
%   because of the asynchronous nature of TeX's output mechanism.
%
\def\threeline/#1/#2/#3/{\rlap{#1}\hss #2\hss\llap{#3}}
\def\of /#1/#2/#3/{\global\def\oddfoot{\threeline/#1/#2/#3/}}
\def\ef /#1/#2/#3/{\global\def\evenfoot{\threeline/#1/#2/#3/}}
%
%   Table of contents macros.  Sorry they are so complex.
%
%   This macro specifies the amount of space each successive entry
%   is indented in the table of contents.
%
\def\tab{\hskip15pt}
%
%   This macro is needed to write out a table of contents line such
%   that the page number is expanded when shipped out, but the entry
%   number is expanded immediately.
%
\def\tocline#1{{\edef\next{\write\tocfout{\noexpand\noexpand\noexpand\a
#1/\noexpand\the\noexpand\pageno/}}\next}}
%
%   This macro is generates an index entry of the form
%   /Heading/entry/pageno
%   The page number is expanded when shipped out, but the heading and 
%   entry are expanded immediately.
%
\def\idxline#1#2{\edef\next{\write\indxout{{#1}{#2}/\the\noexpand\pageno/}}
\next}
%
%   Define a Chapter Heading macro for consistency
%
\def\cheading#1#2#3{
   \vskip 144pt
   \rightline{\cfont {#1 #2}}
   \vb
   \rightline{\cfont {#3}}
   \vb}%
%
%   This is the macro for both Chapters and Appendices.  It sets
%   the footers, prints the chapter title, and prints a message.
%   Each chapter is forced to start on an odd page.
%
\def\chapter#1#2#3{\page
   \ifodd\pageno\else\ef ////\of ////\blankpage\fi
   \global\secno=0\global\subsecno=0\global\tableno=0\global\figno=0
   \gdef\chapterno{#2}
   \ef /\folio//#1 #2/
   \of /#3//\folio/
   \tocline{{\bf {#1 #2}. #3}}
   \message{<\chapterno>}
   \cheading{#1}{#2}{#3}\noindent}
%
%   The section macro sets the headers, writes a message, and prints
%   the section title.  A \bigbreak gives a possible page break at
%   the beginning of a section.
%
\def\section#1#2{
   \bigbreak\global\advance\secno by1\global\subsecno=0
   \tocline{\tab\the\secno\ #2}
   \message{\the\secno}
   \leftline{\bf {#1}\the\secno\ #2}
   \vskip-\parskip\vskip8pt\noindent}
%
%   The subsection macro writes a message and prints the subsection title.
%   The \bigbreak gives a possible page break at the subsection.
%
\def\subsection#1{
   \bigbreak\global\advance\subsecno by1
   \tocline{\tab\tab\the\subsecno\ #1}
   \message{\the\secno.\the\subsecno}
   \leftline{\bf #1}
   \vskip-\parskip\vskip6pt\noindent}
%
%   The subsubsection macro suppresses indentation and boldens the
%   caption.  It does not generate an entry in the table of contents.
%
\def\subsub#1{\smallskip\noindent{\bf #1}}
%
%   A set of macros to handle lists would be nice.
%
\def\startlist{\smallbreak\begingroup\parskip=6pt}
\def\endlist{\par\endgroup}
%
%	Some macros for specific constructions
%
\def\usage#1{\noindent{\it Usage: #1}\hfil\break\noindent}%
\def\useor#1{\hphantom{\it Usage: }{\llap{\it or: }{\it #1}}
   \hfil\break\noindent}%
\def\errormsg#1#2#3{
   \smallbreak\noindent
   {\bf Error: |#1| ~Severity: |#2| ~Message: |#3|}
   \hfil\break\noindent}%
\def\seealso#1{
   \vskip 0pt
   \leftline{See Also: {\bf #1}}}%
%
%   This macro prints the Table of Contents.
%
\def\printcontents{
   \closeout\tocfout
   \page\pageno=-1
   \of ////
   \ef ////
   \cheading{Table of Contents}{}{\DOC}
   \def\leaderfill{\leaders\hbox to 1em{\hss.\hss}\hfill}
   \def\a##1/##2/{\line{{##1}\leaderfill ##2}}
   \input df0:working/contents 
   \page}
%
%   All done with the definitions ...	
%   Finally, we open the table of contents and index files.
%
\openout\tocfout df0:working/contents
\openout\indxout df0:working/index
%
%   Define the headline and footline
%
\headline={\hfil}
\footline={\ifodd\pageno\oddfoot\else\evenfoot\fi}
%
%   Clear the running footers ...
%
\ef ////\of ////
%
%   This macro allows \thatsit to be ignored in subfiles, but act as \bye
%   for the main file.
%
\def\subdoc{\begingroup\let\thatsit=\endgroup}
\def\thatsit{\printcontents\newbye}
\let\newbye=\bye
%
%   That's it, folks!
%
