%
%	The ARexx User's Manual
%
% Copyright � 1987 by William S. Hawes.  All Rights Reserved.
%
\ifx\radeye\fmtversion\subdoc\else\input pmac \fi

\chapter{Glossary}{}{}%

\subsub{Allocation}.
A grant of a system resource, such as memory space.
Programs designed to run in a multitasking environment generally use 
dynamic allocation to avoid tying up system resources.

\subsub{AmigaDOS}.
The higher-level part of the Amiga operating system that supports the 
filing system and input/output operations.

\subsub{Argstring}.
An ``argument string'' structure used to pass data to an \AR\ program.
The structure is passed as a pointer to the buffer area containing the string
data,
and can be treated as a pointer to a null-terminated string.

\subsub{Argument}.
A data item passed to a function, sometimes called a parameter.

\subsub{Clause}.
A group of one or more tokens forming a ``sentence'' in a language.
The clause is the smallest executable language fragment.

\subsub{Command Line Interface (CLI)}.
A program that accepts input from the user and runs programs based on the 
entered command.
The CLI generally refers to the command interpreter supplied with the Amiga,
but other command ``shells'' may be used instead.

\subsub{Concatenation}.
An operation in which two strings are joined or ``chained together.''  
\AR\ provides two concatenation operators, 
one of which joins strings directly and the other of which embeds a blank 
between the operands.

\subsub{EXEC}.
The multitasking kernel of the Amiga's operating system.
EXEC provides the task scheduling, interrupt handling, 
and message-passing primitives used to support \AR.

\subsub{Function Host}.
A program that manages a public message port for receiving function
invocation messages.
The message port may be the same one used for command messages.

\subsub{Function Library}.
A collection of functions callable from \AR\ and managed as an Amiga
shared library.  
Each function library includes an entry point to associate a function name 
with the code to be called.

\subsub{Host Address}.
The name of the public message port associated with a host application.
The host address is used as the unique identifier for the host, 
and should be unique within the system message ports list.
Within an \AR\ program the host address identifies the external host to 
which commands will be sent.

\subsub{Host Application}.
An executable program that provides a suitable command interface to receive
\AR\ commands.
Most host applications will also provide a means to invokde macro programs
from within the application.

\subsub{Interrupt}.
An event that alters the normal flow of control in a program.
Interrupts in \AR\ refer to events within the program execution and are 
distinct from the hardware-level interrupts managed by the Amiga EXEC system.

\subsub{Macro Program}.
A program that implements a complex ``macro'' operation from a series of 
``micro'' commands.

\subsub{Message Packet}.
A data structure used to pass information between tasks.
A message packet is allocated and initialized by one task and then sent to 
another task's message port.
After the recipient has processed the message, 
it ``replies'' the message to the {\it replyport} associated with the message.

\subsub{Message Port}.
A data structure used as the rendezvous point for message passing.
A message port provides the anchor for a list of message packets and 
identifies the task to be signalled when a message arrives.

\subsub{Multitasking}.
The ability to run more than one program at a time.
More precisely, multitasking permits the resources of the computer to be 
shared among many tasks without forcing any task to be aware of the others.

\subsub{Process}.
An extension to an EXEC task structure that provides the data fields 
required to use AmigaDOS functions.
All \AR\ programs run as AmigaDOS processes.

\subsub{Replyport}.
A message port designated to receive a returning message packet.
Each message packet includes a field that specifies its reply port.

\subsub{Resident Process}.
The program responsible for launching \AR\ programs and for managing
various resources used by \AR.
It is structured as a host application and opens a public message port 
named ``REXX.''

\subsub{Shared Library}.
A collection of executable code and data managed as a resource by the 
EXEC operating system.
As the name ``shared'' implies,
the code and data in a library can be used by more than one task.

\subsub{Storage Environment}.
The collection of data values forming the current state of an \AR\ program.
Storage environments are strictly nested and only one environment is current
at any time.

\subsub{Task}.
An entity consisting of executable code and a data structure managed by
the EXEC operating system.
The task is the smallest program unit that can be scheduled and run separately.

\subsub{Token}.
The elementary words or atoms of a language.
A token can be considered as a string of one or more characters forming
the smallest unit of the language.

\subsub{Typeless}.
Data items having no assumed structure or usage.  
\AR\ treats all data as typeless character strings and checks for
specific characteristics only when required by an operation.

\thatsit
