%
%	The ARexx User's Manual
%
% Copyright � 1987 by William S. Hawes.  All Rights Reserved.
%
\ifx\radeye\fmtversion\subdoc\else\input pmac \fi

\chapter{Chapter}{3}{Elements of the Language}%
This chapter introduces the rules and concepts that make up the REXX language.
The intent is not to present a formalized definition,
but rather to convey a practical understanding of how the langauge elements
``fit together'' to form programs.

\section{\chapterno-}{Format}%
\AR\ programs are composed of ASCII characters and may be created using 
any text editor.
No special formatting of the program statements is required or imposed on 
the programmer.
\idxline{Program Format}{}

\section{\chapterno-}{Tokens}%
The smallest distinct entities or ``words'' of the langauge are called 
\hp{tokens}.
A token may be a series of characters, as in the symbol |MyName|, 
or just a single character like the ``|+|'' operator.
Tokens can be categorized into \hp{comments}, \hp{symbols}, \hp{strings}, 
\hp{operators}, and \hp{special characters}.
Each of these groups are described below.
\idxline{Tokens}{}

\subsection{Comment Tokens}%
Any group of characters beginning with the sequence ``|/*|'' and ending 
with ``|*/|'' defines a \hp{comment} token.  
Comments may be placed anywhere in a program and cost little in terms 
of execution speed, 
since they are stripped (removed) when the program is first scanned by the 
interpreter.  
Comments may be ``nested'' within one another, 
but each ``|/*|'' must have a matching ``|*/|'' in the program.
\idxline{comment tokens}{}

\begexamps
/* Your basic comment          */
/* a  /* nested! */  comment   */
#endverb

\subsection{Symbol Tokens}%
Any group of the characters |a|--|z|, |A|--|Z|, |0|--|9|, and |.!?\$\_| 
defines a \hp{symbol} token.
Symbols are translated to uppercase as the program is scanned by the 
interpreter, 
so the symbol |MyName| is equivalent to |MYNAME|.  
Four types of symbols are recognized:
\idxline{symbols,}{tokens}

\startlist
\item{$\bullet$} \hp{Fixed} symbols begin with a digit (|0-9|) or a 
period (|.|). 

\item{$\bullet$} \hp{Simple} symbols do not begin with a digit,
and do not contain any periods.

\item{$\bullet$} \hp{Stem} symbols have exactly one period at the end of the
symbol name.

\item{$\bullet$} \hp{Compound} symbols include one or more periods in the
interior of the name.
\endlist
\noindent%
Stems and compound symbols have special properties that make them useful 
for building arrays and lists.

\subsub{Symbol Values}.
The value used for a fixed symbol is always the symbol name itself 
(as translated to uppercase.)  
Simple, stem, and compound symbols are called \hp{variables} and may be 
assigned a value during the course of the program execution.
A variable is \hp{uninitialized} if it has not yet been assigned a value; 
the value used for an uninitialized variable is just the variable name itself.

\begexamps
123.45       /* a fixed symbol       */
MyName       /* same as MYNAME       */
a.           /* a stem symbol        */
a.1.Index    /* a compound symbol    */
#endverb

\subsection{String Tokens}%
A group of characters beginning and ending with a quote (|'|) or double quote
(|"|) delimiter defines a \hp{string} token.  
The delimiter character itself may be included within the string by a 
double-delimiter sequence (|''| or |""|).
The number of characters in the string is called its length, 
and a string of length zero is called a \hp{null string}.  
A string is treated as a \hp{literal} in an expression; 
that is, its value is just the string itself.
\idxline{strings,}{tokens}

Strings followed immediately by an ``|X|'' or ``|B|'' character that is 
not part of a longer symbol are classifed as \hp{hex} or \hp{binary} strings,
respectively, and must be composed of hexadecimal digits (|0-9|,|A-F|) 
or binary digits (|0|,|1|). 
Blanks are permitted at byte boundaries for added readability.
Hex and binary strings are convenient for specifying non-ASCII characters
and for machine-specific information like addresses in a program.
They are converted immediately to the ``packed'' internal form.
\idxline{strings,}{hex}
\idxline{strings,}{binary}

\begexamps
"Now is the time"    /* a simple example     */
""                   /* a null string        */
'Can''t you see??'   /* Can't you see??      */
'4A 3B C0'X          /* a hex string         */
'00110111'b          /* binary for '7'       */
#endverb

\subsection{Operators}%
The characters |\~+-*/=><\&\|\^| may be combined in the sequences shown in
Table \number\chapterno .1 to form \hp{operator} tokens.
Operator sequences may include leading, trailing, and embedded blanks, 
all of which are removed when the program is scanned.
In addition to the above characters,
the \hp{blank} character is treated as a concatenation operator if it follows 
a symbol or string and is not adjacent to an operator or special character.
\idxline{operators,}{tokens}

Each operator has an associated priority that determines the order in which 
operations will be performed in an expression.
Operators with higher priorities are performed before those with lower 
priorites.

$$\vbox{\halign{\tt #\hfil&&\qquad#\hfil\cr
\multispan3\hfil Table \number\chapterno .1  Operator Sequences\hfil\cr
\bf Sequence&\bf Priority&\bf Operator Definition\cr
\~&8&Logical NOT\cr
+&8&Prefix Conversion\cr
-&8&Prefix Negation\cr
\noalign{\smallskip}
**&7&Exponentiation\cr
\noalign{\smallskip}
*&6&Multiplication\cr
/&6&Division\cr
\%&6&Integer Division\cr
//&6&Remainder\cr
\noalign{\smallskip}
+&5&Addition\cr
-&5&Subtraction\cr
\noalign{\smallskip}
\|\|&4&Concatenation\cr
\rm (blank)&4&Blank Concatenation\cr
\noalign{\smallskip}
==&3&Exact Equality\cr
\~==&3&Exact Inequality\cr
=&3&Equality\cr
\~=&3&Inequality\cr
>&3&Greater Than\cr
>={\rm, }\~<&3&Greater Than or Equal To\cr
<&3&Less Than\cr
<={\rm, }\~>&3&Less Than or Equal To\cr
\noalign{\smallskip}
\&&2&Logical AND\cr
\noalign{\smallskip}
\|&1&Logical Inclusive OR\cr
\^{\rm, }\&\&&1&Logical Exclusive OR\cr}}$$

\subsection{Special Character Tokens}%
The characters |:();,| are each treated as a separate \hp{special character}
token and have particular meanings within an \AR\ program.
Blanks adjacent to these special characters are removed, 
except for those preceding an open parenthesis or following a close 
parenthesis.
\idxline{special characters,}{tokens}

\subsub{Colon (|:|)}.
A colon, if preceded by a symbol token,
defines a \hp{label} within the program.
Labels are locations in the program to which control may be transferred 
under various conditions.
\idxline{colon,}{as token}

\subsub{Opening and Closing Parentheses (|()|)}. 
Parentheses are used in expressions to group operators and operands into 
subexpressions,
in order to override the normal operator priorities.  
An open parenthesis also serves to identify a \hp{function call} within an 
expression;
a symbol or string followed immediately by an open parenthesis defines a 
function name.
Parentheses must always be balanced within a statement.
\idxline{open parenthesis,}{as token}
\idxline{close parenthesis,}{as token}

\subsub{Semicolon (|;|)}.  
The semicolon acts as a program statement terminator.
Several statements may be placed on a single source line if separated by 
semicolons.  
\idxline{semicolon,}{as token}

\subsub{Comma (|,|)}.  
A comma token acts as the continuation character for statements that must 
be entered on several source lines.
It is also used to separate the argument expressions in a function call.
\idxline{comma,}{as token}

\section{\chapterno-}{Clauses}%
Tokens are grouped together to form \hp{clauses}, 
the smallest language unit that can be executed as a statement.  
Every clause in \AR\ can be classified as either a 
\hp{null}, \hp{label}, \hp{assignment}, \hp{instruction}, or \hp{command} 
clause.
The classification process is very simple,
since no more than two tokens are required to classify any clause.
Assignment, instruction, and command clauses are jointly termed 
\hp{statements}.
\idxline{clauses,}{classification}

\subsub{Clause Continuation}.
The end of a source line normally acts as the implicit end of a clause.  
A clause can be continued on the next source line by ending the line with a 
comma (|,|). 
The comma is then removed,
and the next line is considered as a continuation of the clause.  
There is no limit to the number of continuations that may occur.  
String and comment tokens are automatically continued if a line ends before 
the closing delimiter has been found, 
and the ``newline'' character is not considered to be part of the token.
\idxline{clauses,}{continuation of}

\subsub{Multiple Clauses}.
Several clauses can be placed on a single line by separating them with
semicolons (|;|).

\subsection{Null Clauses}%
Lines consisting only of blanks or comments are called \hp{null} clauses.
They have no function in the execution of a program, 
except to aid its readability and to increment the source line count.  
Null clauses may appear anywhere in a program.
\idxline{clauses,}{null}

\begexamp
/* perform annuity calculations     */
#endverb

\subsection{Label Clauses}% 
A symbol followed immediately by a colon defines a \hp{label} clause.
A label acts as a placemarker in the program, 
but no action occurs with the ``execution'' of a label.  
The colon is considered as an implicit clause terminator, 
so each label stands as a separate clause.  
Label clauses may appear anywhere in a program.
\idxline{clauses,}{label}

\begexamps
start:       /* begin execution     */
syntax:      /* error processing    */
#endverb

\subsection{Assignment Clauses}%
Assignments are identified by a variable symbol followed by an ``|=|'' 
operator.
In this context the operator's normal definition (an equality comparison) is 
overridden, and it becomes an assignment operator.  
The tokens to the right of the ``|=|'' are evaluated as an expression, 
and the result is \hp{assigned to} (becomes the value of) the variable symbol.
\idxline{clauses,}{assignment}

\begexamps
when = `Now is the time'
answ = 3.14 * fact(5)
#endverb

\subsection{Instruction Clauses}%
Instructions begin with certain keyword symbols, 
each of which denotes a particular action to be performed.
Instruction keywords are recognized as such only at the beginning of a clause, 
and may otherwise be used freely as symbols (although such use may become
confusing at times.)  
The \AR\ instructions are described in detail in Chapter 4.  
\idxline{clauses,}{instruction}

\begexamps
drop a b c            /* reset variables       */
say 'please'          /* a polite program      */
if j > 5 then leave;  /* several instructions  */
#endverb

\subsection{Command Clauses}%
Commands are any \AR\ expression that can't be classified as one of the 
preceding types of clauses.  
The expression is evaluated and the result is issued as a command to an 
external \hp{host}, 
which might be the native operating system or an application program.  
Commands are discussed in Chapter 5, 
and the details of the host command interface are given in Chapter 10.
\idxline{clauses,}{command}

\begexamps
'delete' 'myfile'    /* a DOS command        */
'jump' current+10    /* an editor command?   */
#endverb

\subsection{Clause Classification}%
The process by which program lines are divided into clauses and then 
classified is important in understanding the operation of an \AR\ program.  
The language interpreter splits the program source into groups of clauses 
as the program is read, 
using the end of each line as a clause separator and applying the 
continuation rule as required.  
These groups of one or more clauses are then tokenized, 
and each clause is classified into one of the above types.
Note that seemingly small syntactic differences may completely change the 
semantic content of a statement.  For example, 

\begprog
SAY 'Hello, Bill'
#endverb
is an instruction clause and will display ``|Hello, Bill|'' on the console, 
but

\begprog
''SAY 'Hello, Bill'
#endverb
is a command clause, and will issue ``|SAY Hello, Bill|'' as a command to an
external program.
The presence of the leading null string changes the classification from an 
instruction clause to a command clause.
\idxline{clauses,}{classification of}

\section{\chapterno-}{Expressions}%
Expression evaluation is an important part of \AR\ programs,
since most statements include at least one expression.
Expressions are composed of strings, symbols, operators, and parentheses.
Strings are used as literals in an expression; 
their value in an operation is just the string itself.
Fixed symbols are also literals 
(remember that symbols are always translated to uppercase,) 
but variable symbols may have an assigned value.
Operator tokens represent the predefined operations of \AR; 
each operator has an associated priority that determines the order in 
which operations will be performed.  
Parentheses may be used to alter the normal order of evaluation in the 
expression,
or to identify \hp{function} calls.  
A symbol or string followed immediately by an open parenthesis defines 
the function name, 
and the tokens between the opening and (final) closing parenthesis form the
\hp{argument list} for the function.
\idxline{expressions}{}
\idxline{functions,}{argument list}

For example, the expression \hbox{``|J 'factorial is' fact(J)|''} is
composed of a symbol |J|, a blank operator, the string |'factorial is'|,
another blank, the symbol |fact|, an open parenthesis,
the symbol |J| again, and a closing parenthesis.
|FACT| is a function name and |(J)| is its argument list,
in this case the single expression |J|.

\subsection{Symbol Resolution}%
Before the evaluation of an expression can proceed,
the interpreter must obtain a value for each symbol in the expression.
For fixed symbols the value is just the symbol name itself, 
but variable symbols must be looked up in the current symbol table.  
In the example above, 
the expression after symbol resolution would be
\hbox{``|3 'factorial is' FACT(3)|,''}
assuming that the symbol |J| had the value |3|.

Suppose that the example above had been
\hbox{``|FACT(J) 'is' J 'factorial'|.''}
Would the second occurrence of symbol |J| still resolve to |3| in this case?  
In general, function calls may have ``side effects'' that include altering 
the values of variables, 
so the value of |J| might have been changed by the call to |FACT|.  
In order to avoid ambiguities in the values assigned to symbols during the 
resolution process, 
\AR\ guarantees a strict left-to-right resolution order.
Symbol resolution proceeds irrespective of operator priority or parenthetical 
grouping; 
if a function call is found, 
the resolution is suspended while the function is evaluated.  
Note that it is possible for the same symbol to have more than one value in 
an expression.
\idxline{expressions,}{symbol resolution}

\subsection{Order of Evaluation}%
After all symbol values have been resolved,
the expression is evaluated based on operator priority and subexpression 
grouping.  
Operators of higher priority are evaluated first.  
\AR\ does not guarantee an order of evaluation among operators of equal 
priority, 
and does not employ a ``fast path'' evaluation of boolean operations.
For example, in the expression
\begprog
(1 = 2) & (FACT(3) = 6)
#endverb
the call to the |FACT| function will be made, 
although it is clear that the final result will be 0, 
since the first term of the AND operation is 0.
\idxline{operators,}{order of evaluation}

\section{\chapterno-}{Numbers and Numeric Precision}%
An important class of operands are those representing numbers.  
Numbers consist of the characters |0-9|, |.+-|, and blanks; 
an |e| or |E| may follow a number to indicate \hp{exponential notation},
in which case it must be followed by a (signed) integer.

Both string tokens and symbol tokens may be used to specify numbers. 
Since the language is typeless,
variables do not have to be declared as ``numeric'' before being used in 
an arithmetic operation.
Instead, each value string is examined when it is used to verify that 
it represents a number. 
The following examples are all valid numbers:

\begprog
33
"  12.3  "
0.321e12
' +  15.'
#endverb
Note that leading and trailing blanks are permitted, 
and that blanks may be embedded between a ``|+|'' or ``|-|'' sign and 
the number body (but not within the body.)

\subsection{Boolean Values}%
The numbers 0 and 1 are used to represent the boolean values {\bf False} and 
{\bf True}, respectively.  
The use of a value other than 0 or 1 when a boolean operand is expected 
will generate an error.
Any number equivalent to 0 or 1, for example ``|0.000|'' or ``|0.1E1|,'' 
is also acceptable as a boolean value.

\subsection{Numeric Precision}%
\AR\ allows the basic precision used for arithmetic calculations to be 
modified while a program is executing.  
The number of significant figures used in arithmetic operations is determined 
by the Numeric Digits environment variable, 
and may be modified using the |NUMERIC| instruction.

The number of decimal places used for a result depends on the operation
performed and the number of decimal places in the operands.
Unlike many languages,
\AR\ preserves trailing zeroes to indicate the precision of the result.
If the total number of digits required to express a value exceeds the current
Numeric Digits setting, 
the number is formatted in \hp{exponential notation}.
Two such formats are provided:

\startlist
\item{$\bullet$} In |SCIENTIFIC| notation, the exponent is adjusted so that a 
single digit is placed to the left of the decimal point.

\item{$\bullet$} In |ENGINEERING| notation, 
the number is scaled so that the exponent is a multiple of 3
and the digits to the left of the decimal point range from 1 to 999.
\endlist
\noindent%
The numeric precison and format can be set using the |NUMERIC| instruction.
\idxline{precision,}{numeric}
\idxline{exponential notation}{}
\idxline{scientific notation}{}
\idxline{engineering notation}{}

\section{\chapterno-}{Operators}
Operators can be grouped into four categories:
\idxline{expressions,}{operators in}
\idxline{operators,}{types of}

\startlist
\item{$\bullet$} \hp{Arithmetic} operators require one or two numeric operands,
and produce a numeric result.

\item{$\bullet$} \hp{Concatenation} operators join two strings into a 
single string.

\item{$\bullet$} \hp{Comparison} operators require two operands, 
and produce a boolean (0 or 1) result.

\item{$\bullet$} \hp{Logical} operators require one or two boolean operands, 
and produce a boolean result.
\endlist

\subsection{Arithmetic Operators}%
The arithmetic operators are listed in Table \number\chapterno .2 below.  
Note the inclusion of the integer division (|\%|) and remainder (|//|) 
operators, 
along with the usual arithmetic operations.
The result of an arithmetic operation is always formatted based on the
current Numeric Digits setting,
and will never have leading or trailing blanks.

$$\vbox{\halign{\tt #\hfil&&\qquad#\hfil\cr
\multispan3{\hfil Table \number\chapterno .2 Arithmetic Operators\hfil}\cr
\bf Sequence&\bf Priority&\bf Operation\cr
+&8&Prefix Conversion\cr
-&8&Prefix Negation\cr
\noalign{\smallskip}
**&7&Exponentiation\cr
\noalign{\smallskip}
*&6&Multiplication\cr
/&6&Division\cr
\%&6&Integer Division\cr
//&6&Remainder\cr
\noalign{\smallskip}
+&5&Addition\cr
-&5&Subtraction\cr}}$$

\subsub{Prefix Conversion (|+|)}.
This unary operator converts the operand to and internal numeric form
and formats the result based on the current Numeric Digits settings.
This causes any leading and trailing blanks to be removed,
and may result in a loss of precision.

\begexamps
'   3.12  '             ==> 3.12
1.5001                  ==> 1.500  /* If digits = 3 */
#endverb

\subsub{Prefix Negation (|-|)}.
This unary operator negates the operand.
The result is formatted based on the current Numeric Digits setting.

\begexamps
-'   3.12  '            ==> -3.12
-1.5E2                  ==> -150
#endverb%

\subsub{Exponentiation (|**|)}.
The left operand is raised to the power specified by the right operand, 
which must be an integer.  
The number of decimal places for the result is the product of the 
exponent and the number of decimal places in the base.

\begexamps
2**3                    ==> 8
3**-1                   ==> .333333333
0.5**3                  ==> 0.125
#endverb%

\subsub{Multiplication (|*|)}.
The product of two numbers is computed.
The number of decimal places for the result is the sum of the decimal
places of the operands.

\begexamps
12 * 3                  ==> 36
1.5 * 1.50              ==> 2.250
#endverb

\subsub{Division (|/|)}.
The quotient of two numbers is computed.
The number of decimal places for the result depends on the current setting 
of the numeric |DIGITS| variable; 
the number is formatted to the maximum precision required.

\begexamps
6 / 3                   ==> 2
8 / 3                   ==> 2.66666667
#endverb%

\subsub{Integer Division (|\%|)}.
The quotient of two numbers is computed, 
and the integer part of the quotient is used as the result.

\begexamps
5 % 3                   ==> 1
-8 % 3                  ==> -2
#endverb

\subsub{Remainder (|//|)}.
The result is the remainder after the two operands are divided.
The remainder for ``|a//b|'' is calculated as ``|a-(a\%b)*b|.''  
If both operands are positive integers, 
this operation yields the usual ``modulo'' result.

\begexamps
5 // 3                  ==> 2
-5 // 3                 ==> -2
5.1 // 0.2              ==> 0.1
#endverb%

\subsub{Addition (|+|)}.
The sum of two numbers is computed.
The number of decimal places for the result is the larger of the decimal
places of the operands.

\begexamps
12 + 3                  ==> 15
3.1 + 4.05              ==> 7.15
#endverb

\subsub{Subtraction (|-|)}.
The difference of two numbers is computed.
As in the case of addition, 
the number of decimal places for the result is the larger of the decimal
places of the operands.

\begexamps
12 - 3                  ==> 9
5.55 - 1.55             ==> 4.00
#endverb%

\subsection{Concatenation Operators}%
\AR\ defines two concatenation operators,
both of which require two operands.
The first, identified by the operator sequence ``|\|\||'', 
joins two strings into a single string with no intervening blank.  
The second concatenation operation is identified by the blank operator, 
and joins the two operand strings with one intervening blank.  

An implicit concatenation operator is recognized when a symbol and a string
are directly abutted in an expression.
Concatenation by abuttal uses the ``|\|\||'' operator, 
and behaves exactly as though the operator had been provided explicitly.

\begexamps
'why me,' \|\| 'Mom?'     ==> why me,Mom?
'good' 'times'          ==> good times
one'two'three           ==> ONEtwoTHREE
#endverb%

\subsection{Comparison Operators}%
Comparisons are performed in one of three modes,
and always result in a boolean value (0 or 1.)
\idxline{operators,}{comparison}

\startlist
\item{$\bullet$} \hp{Exact} comparisons proceed character-by-character, 
including any leading blanks that may be present.

\item{$\bullet$} \hp{String} comparisons ignore leading blanks,
and pad the shorter string with blanks if necessary.

\item{$\bullet$} \hp{Numeric} comparisons first convert the operands to an 
internal numeric form using the current Numeric Digits setting, 
and then perform a standard arithmetic comparison.
\endlist
\noindent%
Except for the exact equality and exact inequality operators, 
all comparison operators dynamically determine whether a string or numeric
comparison is to be performed.  
A numeric comparison is performed if both operands are valid numbers; 
otherwise, the operands are compared as strings. 

$$\vbox{\halign{\tt #\hfil&&\qquad#\hfil\cr
\multispan4{\hfil Table \number\chapterno .3 Comparison Operators\hfil}\cr
\bf Sequence&\bf Priority&\bf Operation&\bf Mode\cr
==&3&Exact Equality&Exact\cr
\~==&3&Exact Inequality&Exact\cr
=&3&Equality&String/Numeric\cr
\~=&3&Inequality&String/Numeric\cr
>&3&Greater Than&String/Numeric\cr
>={\rm },\~<&3&Greater Than or Equal&String/Numeric\cr
<&3&Less Than&String/Numeric\cr
<={\rm ,}\~>&3&Less Than or Equal&String/Numeric\cr}}$$

\subsection{Logical (Boolean) Operators}%
\AR\ defines the four logical operations NOT, AND, OR, and Exclusive OR,
all of which require boolean operands and produce a boolean result.  
Boolean operands must have values of either 0 ({\bf False}) or 1 ({\bf True}.)
An attempt to perform a logical operation on a non-boolean operand will 
generate an error.
\idxline{operators,}{boolean}

$$\vbox{\halign{\tt #\hfil&&\qquad#\hfil\cr
\multispan3{\hfil Table \number\chapterno .4 Logical Operators\hfil}\cr
\bf Sequence&\bf Priority&\bf Operation\cr
\~&8&NOT (Inversion)\cr
\&&2&AND\cr
\|&1&OR\cr
\^{\rm, }\&\&&1&Exclusive OR\cr}}$$

\section{\chapterno-}{Stems and Compound Symbols}%
Stems and compound symbols have special properties that allow for some 
interesting and unusual programming.  A compound symbol can be regarded 
as having the structure \hbox{|stem.|$n_1$|.|$n_2$|.|$n_3\ldots n_k$}
where the leading name is a stem symbol and each node $n_1\ldots n_k$ is 
a fixed or simple symbol.
Whenever a compound symbol appears in a program, 
its name is \hp{expanded} by replacing each node with its current value
as a (simple) symbol.
The value string may consist of any characters, including embedded blanks, 
and is not converted to uppercase.  
The result of the expansion is a new name that is used in place of the 
compound symbol.  
For example, if |J| has the value 3 and |K| has the value 7, 
then the compound symbol |a.j.k| will expand to |A.3.7|.
\idxline{symbols,}{stem}
\idxline{symbols,}{compound}

Stem symbols provide a way to initialize a whole class of compound symbols.  
When an assignment is made to a stem symbol, 
it assigns that value to all possible compound symbols derived from the stem.  
Thus, the value of a compound symbol depends on the prior assignments 
made to itself or its associated stem.

Compound symbols can be regarded as a form of ``associative'' or
``content-addressable'' memory.  
For example, suppose that you needed to store and retrieve a set of names 
and telephone numbers.
The conventional approach would be to set up two arrays |NAME| and |NUMBER|, 
each indexed by an integer running from one to the number of entries.  
A number would be ``looked up'' by scanning the name array until the given 
name was found, say in |NAME.12|, and then retrieving |NUMBER.12|.  
With compound symbols, 
the symbol |NAME| could hold the name to be looked-up,
and |NUMBER.NAME| would then expand to |NUMBER.Bill| (for example), 
which be the corresponding number.

Of course, compound symbols can also be used as conventional indexed arrays,
with the added convenience that only a single assignment (to the stem) is
required to initialize the entire array.

\section{\chapterno-}{The Execution Environment}%
The \AR\ interpreter provides a uniform execution environment by running
each program as a separate task (actually, as a DOS \hp{process}) 
in the Amiga's multitasking operating system.  
This allows for a flexible interface between an external host program 
and the interpreter, 
as the host can either proceed concurrently with its operations or can 
simply wait for the interpreted program to finish.
\idxline{Program execution,}{environment}

\subsection{The External Environment}%
The external environment of a program includes its task (process) structure, 
input and output streams, and current directory.  
When each \AR\ task is created, it inherits the input and output streams 
and current directory from its \hp{client}, 
the external program that invoked the \AR\ program.
The current directory is used as the starting point in a search for a program 
or data file.

\subsub{External Programs}.
The external environment usually includes one or more external programs with 
which the \AR\ program may communicate.  
Any program that supports a suitable interface can receive commands from 
\AR\ programs.
The command interface is discussed in Chapter 5.

\subsection{The Internal Environment}%
The internal environment of an \AR\ program consists of a static global
structure and one or more storage environments.  
The global data values are fixed at the time the program is invoked, 
and include the argument strings, program source code, and static data strings.
The storage environment includes the symbol table used for variable values, 
the numeric options, trace option, and host address strings.  
While the global environment is unique, 
there may be many storage environments during the course of the 
program execution.  
Each time an internal function is called a new storage environment is 
activated and initialized.  
The initial values for most fields are inherited from the previous environment,
but values may be changed afterwards without affecting the caller's 
environment.  
The new environment persists until control returns from the function.
\idxline{storage environments,}{fields in}

\subsub{Argument Strings}.
A program may receive one or more argument strings when it is first invoked. 
These arguments persist for the duration of the program and are never altered. 
The number of arguments a program receives depends in part on the mode of
invocation.  
\AR\ programs invoked as commands normally have only one argument string, 
although the ``command tokenization'' option may provide more than one.  
A program invoked as a function can have any number of arguments if called 
as an internal function, but external functions are limited to a maximum
of 15 arguments.
\idxline{arguments,}{at invocation}

The argument strings can be retrieved using either the |ARG| instruction or
the |ARG()| Built-In function.  
|ARG()| can also return the total number of arguments,
or the status (as ``exists'' or ``omitted'') of a particular argument.

\subsub{The Symbol Table}.
Every storage environment includes a \hp{symbol table} to store the value 
strings that have been assigned to variables.  
This symbol table is organized as a two-level \hp{binary tree}, 
a data structure that provides an efficient look-up mechanism.  
The primary level stores entries for simple and stem symbols, 
and the secondary level is used for compound symbols.  
All of the compound symbols associated with a particular stem are stored in 
one tree, with the root of the tree held by the entry for the stem.
\idxline{binary tree}{}
\idxline{symbol table,}{organization}

Symbols are not entered into the table until an assignment is made to the
symbol.  Once created, entries at the primary level are never removed, 
even if the symbol subsequently becomes uninitialized.  
Secondary trees are released whenever an assignment is made to the stem 
associated with the tree.

For the most part \AR\ programmers need not be concerned with the details 
of storage environments except to understand what values are saved when
a function is called.  
Applications developers who need to manipulate environment values should 
refer to the structure definitions in the INCLUDE files provided on 
the \AR\ distribution disk.

\subsection{Input and Output}%
Most computer programs require some means of communicating with the outside
world, either to accept input data or to pass along results.
The REXX language includes only a minimal specification of input and output 
(I/O) operations,
leaving the choice of additional functionality to the language implementor.
This is in keeping with the design of many computer languages.
For instance, the ``C'' langauge has no statements dedicated to I/O,
but instead relies on a standardized set of I/O functions.
\idxline{I/O,}{input stream}
\idxline{I/O,}{output stream}
\idxline{I/O,}{facilities}

\AR\ extends the I/O facilities of REXX by providing Built-In functions
to manipulate external files.  
Files are referenced by a \hp{logical name} associated with the file when 
it is first opened.  
The initial input and output streams are given the names |STDIN| and |STDOUT|.
\idxline{Built-In functions,}{in I/O}
\idxline{logical name}{}

\AR\ maintains a list of all of the files opened by a program and 
automatically closes them when the program finishes.  
There is no limit to the number of files that may be open simultaneously.

\subsection{Resource Tracking}%
\AR\ provides complete tracking for all of the dynamically-allocated 
resources that it uses to execute a program.
These resources include memory space, DOS files and related structures, 
and the message port structures supported by \AR.
The tracking system was designed to allow a program to ``bail out'' at any 
point (perhaps due to an execution error) without leaving any hanging 
resources.
\idxline{Resource tracking}{}

It is possible to go outside of the interpreter's resource tracking net
by making calls directly to the Amiga's operating system from within an
\AR\ program.
In these cases it is the programmer's responsibility to track and return
all of the allocated resources.
\AR\ provides a special interrupt facility so that a program can retain 
control after an execution error,
perform the required cleanup, and then make an orderly exit.
Chapter 7 has information on the \AR\ interrupt system.
\idxline{interrupts}{}

\thatsit
