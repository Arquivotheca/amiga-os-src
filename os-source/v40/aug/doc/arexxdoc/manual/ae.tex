%
%	The ARexx User's Manual
%
% Copyright � 1987 by William S. Hawes.  All Rights Reserved.
%
\ifx\radeye\fmtversion\subdoc\else\input pmac \fi

\chapter{Appendix}{E}{Distribution Files}%
This appendix lists the directories of the standard \AR\ distribution 
disk.
The contents of some of the directories may change from time to time,
so your disk may not show exactly the same files.
Most notably,
the |:rexx| directory will expand as more program examples are included in it.

The second section of the Appendix lists the HEADER files that define the
constants and data structures used with \AR.
All of these files are available in the :INCLUDE directory,
but are listed here for convenience in studying the structures.

\section{\chapterno-}{Directories}%
The files are listed below as they would be using the system |dir| command.
For example, \hbox{``dir df1:c opt a''} would list the contents of the |:c|
directory on disk drive 1.

\subsection{The |:C| Directory}%
This directory contains the command utilities used with \AR.
These files should be copied to your system |C:| directory when you install 
the program.
\begprog
    c (dir)
       hi                               loadlib
       rexxmast                         rx
       rxc                              rxset
       tcc                              tco
       te                               ts
#endverb

\subsection{The |:INCLUDE| Directory}%
This directory has the INCLUDE and HEADER files used for assembly language
and ``C'' programming, respectively.
These files contain the structure definitions necessary to build an interface
to \AR.
\begprog
     include (dir)
       errors.h                         rexxio.h
       rxslib.h                         storage.h
       errors.i                         rexxio.i
       rxslib.i                         storage.i
#endverb

\page
\subsection{The |:LIBS| Directory}%
These are the library files for the language interpreter and the
Support Library functions.
Both files should be copied to your system |LIBS:| directory when you
install \AR.
\begprog
     libs (dir)
       rexxsupport.library              rexxsyslib.library
#endverb

\subsection{The |:REXX| Directory}%
The |:rexx| directory contains example programs to illustrate various 
features of the language.
New files will be added from time to time,
and users are welcome to contribute files to be distributed in this way.
\begprog
     rexx (dir)
       bigif.rexx                       break.rexx
       builtin.rexx                     calc.rexx
       cmdtest.rexx                     fact.rexx
       factw.rexx                       haltme.rexx
       hosttest.rexx                    iftest.rexx
       marquis.rexx                     nesttest.rexx
       paver.rexx                       potpourri.rexx
       rslib.rexx                       select.rexx
       sigtest.rexx                     support.rexx
       test1.rexx                       timer.rexx
#endverb

\subsection{The |:TOOLS| Directory}%
These files are intended for software developers,
and include examples of interfacing to \AR.
The file |rexxtest| is of particular interest;
it calls the \AR\ interpreter directly, 
and can be run under a debugger to aid with developing new function libraries.
\begprog
     tools (dir)
       hosttest                         hosttest.asm
       loadlib.asm                      rexxtest
       rexxtest.asm                     rxoffsets.o
#endverb

\subsection{Miscellaneous Files}%
\begprog
  .info                            Install-ARexx
  README                           Start-ARexx
#endverb

\page
\section{\chapterno-}{Listings of Header Files}%
This section of the chapter consists of listings of the header files
contained in the |:include| directory.

\subsection{storage.h}
This is the main header file and contains definitions for all of the 
important data structures used by \AR.

\font\tt=cmtt10 at 10.0truept
\begprog
/* === rexx/storage.h ==================================================
 *
 * Copyright (c) 1986, 1987 by William S. Hawes (All Rights Reserved)
 *
 * =====================================================================
 * Header file to define ARexx data structures.
 */

\#ifndef REXX_STORAGE_H
\#define REXX_STORAGE_H

\#ifndef EXEC_TYPES_H
\#include "exec/types.h"
\#endif
\#ifndef EXEC_NODES_H
\#include "exec/nodes.h"
\#endif
\#ifndef EXEC_LISTS_H
\#include "exec/lists.h"
\#endif
\#ifndef EXEC_PORTS_H
\#include "exec/ports.h"
\#endif
\#ifndef EXEC_LIBRARIES_H
\#include "exec/libraries.h"
\#endif

/* The NexxStr structure is used to maintain the internal strings in REXX.
 * It includes the buffer area for the string and associated attributes.
 * This is actually a variable-length structure; it is allocated for a
 * specific length string, and the length is never modified thereafter
 * (since it's used for recycling).
 */
#endverb
%
\page
\leftline{\bf storage.h (cont.)}\vskip-\lastskip
\begprog
struct NexxStr {
   LONG     ns_Ivalue;                 /* integer value                 */
   UWORD    ns_Length;                 /* length in bytes (excl null)   */
   UBYTE    ns_Flags;                  /* attribute flags               */
   UBYTE    ns_Hash;                   /* hash code                     */
   BYTE     ns_Buff[8];                /* buffer area for strings       */
   };                                  /* size: 16 bytes (minimum)      */

\#define NXADDLEN 9                     /* offset plus null byte         */
\#define IVALUE(nsPtr) (nsPtr->ns_Ivalue)

/* String attribute flag bit definitions                                */
\#define NSB_KEEP     0                 /* permanent string?             */
\#define NSB_STRING   1                 /* string form valid?            */
\#define NSB_NOTNUM   2                 /* non-numeric?                  */
\#define NSB_NUMBER   3                 /* a valid number?               */
\#define NSB_BINARY   4                 /* integer value saved?          */
\#define NSB_FLOAT    5                 /* floating point format?        */
\#define NSB_EXT      6                 /* an external string?           */
\#define NSB_SOURCE   7                 /* part of the program source?   */

/* The flag form of the string attributes                               */
\#define NSF_KEEP     (1 << NSB_KEEP  )
\#define NSF_STRING   (1 << NSB_STRING)
\#define NSF_NOTNUM   (1 << NSB_NOTNUM)
\#define NSF_NUMBER   (1 << NSB_NUMBER)
\#define NSF_BINARY   (1 << NSB_BINARY)
\#define NSF_FLOAT    (1 << NSB_FLOAT )
\#define NSF_EXT      (1 << NSB_EXT   )
\#define NSF_SOURCE   (1 << NSB_SOURCE)

* Combinations of flags 
\#define NSF_INTNUM   (NSF_NUMBER \| NSF_BINARY \| NSF_STRING)
\#define NSF_DPNUM    (NSF_NUMBER \| NSF_FLOAT)
\#define NSF_ALPHA    (NSF_NOTNUM \| NSF_STRING)
\#define NSF_OWNED    (NSF_SOURCE \| NSF_EXT \| NSF_KEEP)
\#define KEEPSTR      (NSF_STRING \| NSF_SOURCE \| NSF_NOTNUM)
\#define KEEPNUM      (NSF_STRING \| NSF_SOURCE \| NSF_NUMBER \| NSF_BINARY)
#endverb
%
\page
\leftline{\bf storage.h (cont.)}\vskip-\lastskip
\begprog
/* The RexxArg structure is identical to the NexxStr structure, but
 * is allocated from system memory rather than from internal storage.
 * This structure is used for passing arguments to external programs.
 * It is usually passed as an "argstring", a pointer to the string buffer.
 */

struct RexxArg {
   LONG     ra_Size;                   /* total allocated length        */
   UWORD    ra_Length;                 /* length of string              */
   UBYTE    ra_Flags;                  /* attribute flags               */
   UBYTE    ra_Hash;                   /* hash code                     */
   BYTE     ra_Buff[8];                /* buffer area                   */
   };                                  /* size: 16 bytes (minimum)      */

/* The RexxMsg structure is used for all communications with Rexx programs.
 * It is an EXEC message with a parameter block appended.
 */
struct RexxMsg {
   struct Message rm_Node;             /* EXEC message structure        */
   APTR     rm_TaskBlock;              /* pointer to global structure   */
   APTR     rm_LibBase;                /* library base                  */
   LONG     rm_Action;                 /* command (action) code         */
   LONG     rm_Result1;                /* primary result (return code)  */
   LONG     rm_Result2;                /* secondary result              */
   STRPTR   rm_Args[16];               /* argument block (ARG0-ARG15)   */
   struct MsgPort *rm_PassPort;        /* forwarding port               */
   STRPTR   rm_CommAddr;               /* host address (port name)      */
   STRPTR   rm_FileExt;                /* file extension                */
   LONG     rm_Stdin;                  /* input stream (filehandle)     */
   LONG     rm_Stdout;                 /* output stream (filehandle)    */
   LONG     rm_avail;                  /* future expansion              */
   };                                  /* size: 128 bytes               */
/* Field definitions                                                    */
\#define ARG0(rmp) (rmp->rm_Args[0])    /* start of argblock             */
\#define ARG1(rmp) (rmp->rm_Args[1])    /* first argument                */
\#define ARG2(rmp) (rmp->rm_Args[2])    /* second argument               */

\#define MAXRMARG  15                   /* maximum arguments             */

/* Command (action) codes for message packets                           */
\#define RXCOMM    $01000000            /* a command-level invocation    */
\#define RXFUNC    $02000000            /* a function call               */
\#define RXCLOSE   $03000000            /* close the port                */
\#define RXQUERY   $04000000            /* query for information         */
\#define RXADDFH   $07000000            /* add a function host           */
#endverb
%
\page
\leftline{\bf storage.h (cont.)}\vskip-\lastskip
\begprog
\#define RXADDLIB  $08000000            /* add a function library        */
\#define RXREMLIB  $09000000            /* remove a function library     */
\#define RXADDCON  $0A000000            /* add/update a ClipList string  */
\#define RXREMCON  $0B000000            /* remove a ClipList string      */
\#define RXTCOPN   $0C000000            /* open the trace console        */
\#define RXTCCLS   $0D000000            /* close the trace console       */

/* Command modifier flag bits                                           */
\#define RXFB_NOIO    16                /* suppress I/O inheritance?     */
\#define RXFB_RESULT  17                /* result string expected?       */
\#define RXFB_STRING  18                /* program is a "string file"?   */
\#define RXFB_TOKEN   19                /* tokenize the command line?    */
\#define RXFB_NONRET  20                /* a "no-return" message?        */

/* Modifier flags                                                       */
\#define RXFF_RESULT  (1 << RXFB_RESULT)
\#define RXFF_STRING  (1 << RXFB_STRING)
\#define RXFF_TOKEN   (1 << RXFB_TOKEN )
\#define RXFF_NONRET  (1 << RXFB_NONRET)
\#define RXCODEMASK   $FF000000
\#define RXARGMASK    $0000000F

/* The RexxRsrc structure is used to manage global resources.
 * The name string for each node is created as a RexxArg structure,
 * and the total size of the node is saved in the "rr_Size" field.
 * Functions are provided to allocate and release resource nodes.
 * If special deletion operations are required, an offset and base can
 * be provided in "rr_Func" and "rr_Base", respectively.  This function
 * will be called with the base in register A6 and the node in A0.
 */
struct RexxRsrc {
   struct Node rr_Node;
   WORD     rr_Func;                   /* "auto-delete" offset          */
   APTR     rr_Base;                   /* "auto-delete" base            */
   LONG     rr_Size;                   /* total size of node            */
   LONG     rr_Arg1;                   /* available ...                 */
   LONG     rr_Arg2;                   /* available ...                 */
   };                                  /* size: 32 bytes                */
/* Resource node types                                                  */
\#define RRT_ANY      0                 /* any node type ...             */
\#define RRT_LIB      1                 /* a function library            */
\#define RRT_PORT     2                 /* a public port                 */
\#define RRT_FILE     3                 /* a file IoBuff                 */
\#define RRT_HOST     4                 /* a function host               */
\#define RRT_CLIP     5                 /* a Clip List node              */
#endverb
%
\page
\leftline{\bf storage.h (cont.)}\vskip-\lastskip
\begprog
/* The RexxTask structure holds the fields used by REXX to communicate with
 * external processes, including the client task.  It includes the global
 * data structure (and the base environment).  The structure is passed to
 * the newly-created task in its "wake-up" message.
 */

\#define GLOBALSZ  200                  /* total size of GlobalData      */
struct RexxTask {
   BYTE     rt_Global[GLOBALSZ];       /* global data structure         */
   struct MsgPort rt_MsgPort;          /* global message port           */
   UBYTE    rt_Flags;                  /* task flag bits                */
   BYTE     rt_SigBit;                 /* signal bit                    */

   APTR     rt_ClientID;               /* the client's task ID
   APTR     rt_MsgPkt;                 /* the packet being processed
   APTR     rt_TaskID;                 /* our task ID
   APTR     rt_RexxPort;               /* the REXX public port

   APTR     rt_ErrTrap;                /* Error trap address
   APTR     rt_StackPtr;               /* stack pointer for traps

   struct List rt_Header1;             /* Environment list              */
   struct List rt_Header2;             /* Memory freelist               */
   struct List rt_Header3;             /* Memory allocation list        */
   struct List rt_Header4;             /* Files list                    */
   struct List rt_Header5;             /* Message Ports List            */
   };

/* Definitions for RexxTask flag bits                                   */
\#define RTFB_TRACE   0                 /* external trace flag           */
\#define RTFB_HALT    1                 /* external halt flag            */
\#define RTFB_SUSP    2                 /* suspend task?                 */
\#define RTFB_TCUSE   3                 /* trace console in use?         */
\#define RTFB_WAIT    6                 /* waiting for reply?            */
\#define RTFB_CLOSE   7                 /* task completed?               */

/* Definitions for memory allocation constants                          */
\#define MEMQUANT  16                   /* quantum of memory space       */
\#define MEMMASK   $FFFFFFF0            /* mask for rounding the size    */

\#define MEMQUICK  (1 << 0 )            /* EXEC flags: MEMF_PUBLIC       */
\#define MEMCLEAR  (1 << 16)            /* EXEC flags: MEMF_CLEAR        */
#endverb
%
\page
\leftline{\bf storage.h (cont.)}\vskip-\lastskip
\begprog
/* The SrcNode is a temporary structure used to hold values destined for a
 * segment array.  It is also used to maintain the memory freelist.
 */

struct SrcNode {
   struct SrcNode *sn_Succ;            /* next node                     */
   struct SrcNode *sn_Pred;
   APTR     sn_Ptr;                    /* pointer value                 */
   LONG     sn_Size;                   /* size of object                */
   };                                  /* size: 16 bytes                */
\#endif
#endverb

\page
\subsection{rxslib.h}
This file defines the library base for the \AR\ Systems Library.

\begprog
/* === rexx/rxslib.h ===================================================
 *
 * Copyright (c) 1986, 1987 by William S. Hawes (All Rights Reserved)
 *
 * =====================================================================
 * The header file for the REXX Systems Library
 */

\#ifndef REXX_RXSLIB_H
\#define REXX_RXSLIB_H

\#ifndef REXX_STORAGE_H
\#include "rexx/storage.h"
\#endif

/* Some macro definitions                                             */

\#define RXSNAME  "rexxsyslib.library"
\#define RXSID    "rexxsyslib 1.0 (23 AUG 87)"
\#define RXSDIR   "REXX"
\#define RXSTNAME "ARexx"

/* The REXX systems library structure.  This should be considered as  */
/* semi-private and read-only, except for documented exceptions.      */

struct RxsLib {
   struct Library rl_Node;             /* EXEC library node           */
   UBYTE    rl_Flags;                  /* global flags                */
   UBYTE    rl_pad;
   APTR     rl_SysBase;                /* EXEC library base           */
   APTR     rl_DOSBase;                /* DOS library base            */
   APTR     rl_IeeeDPBase;             /* IEEE DP math library base   */
   LONG     rl_SegList;                /* library seglist             */
   LONG     rl_MaxAlloc;               /* maximum memory allocation   */
   LONG     rl_Chunk;                  /* allocation quantum          */
   LONG     rl_MaxNest;                /* maximum expression nesting  */
   struct NexxStr *rl_NULL;            /* static string: NULL         */
   struct NexxStr *rl_FALSE;           /* static string: FALSE        */
   struct NexxStr *rl_TRUE;            /* static string: TRUE         */
   struct NexxStr *rl_REXX;            /* static string: REXX         */
   struct NexxStr *rl_COMMAND;         /* static string: COMMAND
   struct NexxStr *rl_STDIN;           /* static string: STDIN
   struct NexxStr *rl_STDOUT;          /* static string: STDOUT
   struct NexxStr *rl_STDERR;          /* static string: STDERR
#endverb
%
\page
\leftline{\bf rxslib.h (cont.)}\vskip-\lastskip
\begprog
   STRPTR    rl_Version;               /* version/configuration string*/
   STRPTR    rl_TaskName;              /* name string for tasks       */
   LONG      rl_TaskPri;               /* starting priority           */
   LONG      rl_TaskSeg;               /* startup seglist             */
   LONG      rl_StackSize;             /* stack size                  */
   STRPTR    rl_RexxDir;               /* REXX directory              */
   STRPTR    rl_CTABLE;                /* character attribute table   */
   struct NexxStr *rl_Notice;          /* copyright notice            */

   struct MsgPort rl_RexxPort;         /* REXX public port            */
   UWORD     rl_ReadLock;              /* lock count                  */
   LONG      rl_TraceFH;               /* global trace console        */
   struct List rl_TaskList;            /* REXX task list              */
   WORD      rl_NumTask;               /* task count                  */
   struct List rl_LibList;             /* Library List header         */
   WORD      rl_NumLib;                /* library count               */
   struct List rl_ClipList;            /* ClipList header             */
   WORD      rl_NumClip;               /* clip node count             */
   struct List rl_MsgList;             /* pending messages            */
   WORD      rl_NumMsg;                /* pending count               */
   };

/* Global flag bit definitions for RexxMaster                         */
\#define RLFB_TRACE RTFB_TRACE          /* interactive tracing?        */
\#define RLFB_HALT  RTFB_HALT           /* halt execution?             */
\#define RLFB_SUSP  RTFB_SUSP           /* suspend execution?          */
\#define RLFB_TCUSE RTFB_TCUSE          /* trace console in use?       */
\#define RLFB_TCOPN 4                   /* trace console open?         */
\#define RLFB_STOP  6                   /* deny further invocations    */
\#define RLFB_CLOSE 7                   /* close the master            */

\#define RLFMASK    0x07                /* passed flags                */

         ; Initialization constants

\#define RXSVERS    2                   /* main version                */
\#define RXSREV     1                   /* revision                    */
\#define RXSALLOC   0x800000            /* maximum allocation          */
\#define RXSCHUNK   1024                /* allocation quantum          */
\#define RXSNEST    32                  /* expression nesting limit    */
\#define RXSTPRI    0                   /* task priority               */
\#define RXSSTACK   4096                /* stack size                  */
\#define RXSLISTH   4                   /* number of list headers      */
#endverb
%
\page
\leftline{\bf rxslib.h (cont.)}\vskip-\lastskip
\begprog
/* Character attribute flag bits used in REXX.  Defined only for      */
/* ASCII characters (range 0-127).                                    */

\#define CTB_SPACE   0                  /* white space characters      */
\#define CTB_DIGIT   1                  /* decimal digits 0-9          */
\#define CTB_ALPHA   2                  /* alphabetic characters       */
\#define CTB_REXXSYM 3                  /* REXX symbol characters      */
\#define CTB_REXXOPR 4                  /* REXX operator characters    */
\#define CTB_REXXSPC 5                  /* REXX special symbols        */
\#define CTB_UPPER   6                  /* UPPERCASE alphabetic        */
\#define CTB_LOWER   7                  /* lowercase alphabetic        */

/* Attribute flags                                                    */
\#define CTF_SPACE   (1 << CTB_SPACE)
\#define CTF_DIGIT   (1 << CTB_DIGIT)
\#define CTF_ALPHA   (1 << CTB_ALPHA)
\#define CTF_REXXSYM (1 << CTB_REXXSYM)
\#define CTF_REXXOPR (1 << CTB_REXXOPR)
\#define CTF_REXXSPC (1 << CTB_REXXSPC)
\#define CTF_UPPER   (1 << CTB_UPPER)
\#define CTF_LOWER   (1 << CTB_LOWER)

\#endif
#endverb
%
\page
\subsection{rexxio.h}
This file defines the data structures used for buffered I/O.  
\AR\ uses linked lists of {\bf IoBuff} structures to keep track of the files
it opens.
Each {\bf IoBuff} node is allocated as an ``auto-delete'' structure and
can be closed and released by a call to either CloseF() or RemRsrcNode().
An entire list of files can be closed with a call to RemRsrcList().

\begprog
/* === rexx/rexxio.h ====================================================
 *
 * Copyright (c) 1986, 1987 by William S. Hawes (All Rights Reserved)
 *
 * ======================================================================
 * Header file for ARexx Input/Output related structures
 */

\#ifndef REXX_REXXIO_H
\#define REXX_REXXIO_H

\#ifndef REXX_STORAGE_H
\#include "rexx/storage.h"
\#endif

\#define RXBUFFSZ  204                  /* buffer length                 */

/* The IoBuff is a resource node used to maintain the File List.  Nodes are
 * allocated and linked into the list whenever a file is opened.
 */

struct IoBuff {
   struct RexxRsrc iobNode;            /* structure for files/strings   */
   APTR     iobRpt;                    /* read/write pointer            */
   LONG     iobRct;                    /* character count               */
   LONG     iobDFH;                    /* DOS filehandle                */
   APTR     iobLock;                   /* DOS lock                      */
   LONG     iobBct;                    /* buffer length                 */
   BYTE     iobArea[RXBUFFSZ];         /* buffer area                   */
   };                                  /* size: 256 bytes               */

/* Access mode definitions                                              */
\#define RXIO_EXIST   -1                /* an external filehandle        */
\#define RXIO_STRF    0                 /* a "string file"               */
\#define RXIO_READ    1                 /* read-only access              */
\#define RXIO_WRITE   2                 /* write mode                    */
\#define RXIO_APPEND  3                 /* append mode (existing file)   */
#endverb
%
\page
\leftline{\bf rexxio.h (cont.)}\vskip-\lastskip
\begprog
/* Offset anchors for SeekF()                                           */
\#define RXIO_BEGIN   -1                /* relative to start             */
\#define RXIO_CURR    0                 /* relative to current position  */
\#define RXIO_END     1                 /* relative to end               */

/* The Library List contains just plain resource nodes.                 */

\#define LLOFFSET(rrp) (rrp->rr_Arg1)   /* "Query" offset                */
\#define LLVERS(rrp)   (rrp->rr_Arg2)   /* library version               */

/* The RexxClipNode structure is used to maintain the Clip List.  The
 * value string is stored as an argstring in the rr_Arg1 field.
 */

\#define CLVALUE(rrp) ((STRPTR) rrp->rr_Arg1)

/* A message port structure, maintained as a resource node.
 * The ReplyList holds packets that have been received but haven't been
 * replied.
 */

struct RexxMsgPort {
   struct RexxRsrc rmp_Node;           /* linkage node                  */
   struct MsgPort  rmp_Port;           /* the message port              */
   struct List     rmp_ReplyList;      /* messages awaiting reply       */
   };

/* DOS Device types                                                     */
\#define DT_DEV    0                    /* a device                      */
\#define DT_DIR    1                    /* an ASSIGNed directory         */
\#define DT_VOL    2                    /* a volume                      */

/* Private DOS packet types                                             */
\#define ACTION_STACK 2002              /* stack a line                  */
\#define ACTION_QUEUE 2003              /* queue a line                  */
\#endif
#endverb
%
\page
\subsection{errors.h}
This file contains the definitions for all of the error messages issued
by the \AR\ interpreter.

\begprog
/* == errors.h =========================================================
 *
 * Copyright (c) 1987 by William S. Hawes (All Rights Reserved)
 * 
 * =====================================================================
 * Definitions for ARexx error codes
 */

\#define ERRC_MSG  0                    /*  error code offset           */
\#define ERR10_001 (ERRC_MSG+1)         /*  program not found           */
\#define ERR10_002 (ERRC_MSG+2)         /*  execution halted            */
\#define ERR10_003 (ERRC_MSG+3)         /*  no memory available         */
\#define ERR10_004 (ERRC_MSG+4)         /*  invalid character in program*/
\#define ERR10_005 (ERRC_MSG+5)         /*  unmatched quote             */
\#define ERR10_006 (ERRC_MSG+6)         /*  unterminated comment        */
\#define ERR10_007 (ERRC_MSG+7)         /*  clause too long             */
\#define ERR10_008 (ERRC_MSG+8)         /*  unrecognized token          */
\#define ERR10_009 (ERRC_MSG+9)         /*  symbol or string too long   */

\#define ERR10_010 (ERRC_MSG+10)        /*  invalid message packet      */
\#define ERR10_011 (ERRC_MSG+11)        /*  command string error        */
\#define ERR10_012 (ERRC_MSG+12)        /*  error return from function  */
\#define ERR10_013 (ERRC_MSG+13)        /*  host environment not found  */
\#define ERR10_014 (ERRC_MSG+14)        /*  required library not found  */
\#define ERR10_015 (ERRC_MSG+15)        /*  function not found          */
\#define ERR10_016 (ERRC_MSG+16)        /*  no return value             */
\#define ERR10_017 (ERRC_MSG+17)        /*  wrong number of arguments   */
\#define ERR10_018 (ERRC_MSG+18)        /*  invalid argument to function*/
\#define ERR10_019 (ERRC_MSG+19)        /*  invalid PROCEDURE           */

\#define ERR10_020 (ERRC_MSG+20)        /*  unexpected THEN/ELSE        */
\#define ERR10_021 (ERRC_MSG+21)        /*  unexpected WHEN/OTHERWISE   */
\#define ERR10_022 (ERRC_MSG+22)        /*  unexpected LEAVE or ITERATE */
\#define ERR10_023 (ERRC_MSG+23)        /*  invalid statement in SELECT */
\#define ERR10_024 (ERRC_MSG+24)        /*  missing THEN clauses        */
\#define ERR10_025 (ERRC_MSG+25)        /*  missing OTHERWISE           */
\#define ERR10_026 (ERRC_MSG+26)        /*  missing or unexpected END   */
\#define ERR10_027 (ERRC_MSG+27)        /*  symbol mismatch on END      */
\#define ERR10_028 (ERRC_MSG+28)        /*  invalid DO syntax           */
\#define ERR10_029 (ERRC_MSG+29)        /*  incomplete DO/IF/SELECT     */
#endverb
%
\page
\leftline{\bf errors.h (cont.)}\vskip-\lastskip
\begprog
\#define ERR10_030 (ERRC_MSG+30)        /*  label not found             */
\#define ERR10_031 (ERRC_MSG+31)        /*  symbol expected             */
\#define ERR10_032 (ERRC_MSG+32)        /*  string or symbol expected   */
\#define ERR10_033 (ERRC_MSG+33)        /*  invalid sub-keyword         */
\#define ERR10_034 (ERRC_MSG+34)        /*  required keyword missing    */
\#define ERR10_035 (ERRC_MSG+35)        /*  extraneous characters       */
\#define ERR10_036 (ERRC_MSG+36)        /*  sub-keyword conflict        */
\#define ERR10_037 (ERRC_MSG+37)        /*  invalid template            */
\#define ERR10_038 (ERRC_MSG+38)        /*  invalid TRACE request       */
\#define ERR10_039 (ERRC_MSG+39)        /*  uninitialized variable      */

\#define ERR10_040 (ERRC_MSG+40)        /*  invalid variable name       */
\#define ERR10_041 (ERRC_MSG+41)        /*  invalid expression          */
\#define ERR10_042 (ERRC_MSG+42)        /*  unbalanced parentheses      */
\#define ERR10_043 (ERRC_MSG+43)        /*  nesting level exceeded      */
\#define ERR10_044 (ERRC_MSG+44)        /*  invalid expression result   */
\#define ERR10_045 (ERRC_MSG+45)        /*  expression required         */
\#define ERR10_046 (ERRC_MSG+46)        /*  boolean value not 0 or 1    */
\#define ERR10_047 (ERRC_MSG+47)        /*  arithmetic conversion error */
\#define ERR10_048 (ERRC_MSG+48)        /*  invalid operand             */

/* Return Codes for general use ...                                    */
\#define RC_FAIL   -1                   /*  something's wrong           */
\#define RC_OK     0                    /*  success                     */
\#define RC_WARN   5                    /*  warning only                */
\#define RC_ERROR  10                   /*  something's wrong           */
\#define RC_FATAL  20                   /*  complete or severe failure  */
#endverb

\thatsit
