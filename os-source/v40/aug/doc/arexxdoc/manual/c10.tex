%
%	The ARexx User's Manual
%
% Copyright � 1987 by William S. Hawes.  All Rights Reserved.
%
\ifx\radeye\fmtversion\subdoc\else\input pmac \fi

\chapter{Chapter}{10}{Interfacing to \AR}%
This chapter discusses the issues involved in designing and implementing
an interface between \AR\ and an external applications program.
The material presented here is directed to software developers,
so a high degree of familiarity with programming the Amiga in either ``C'' 
or assembly-language is assumed.

\AR\ can interact with external programs in several ways.  
The \hp{command interface} is used to communicate with an external program
running as a separate task in the Amiga's multitasking environment.
The interaction takes place by passing messages between public message ports,
and is in many ways similar to the interaction of a program with Intuition, 
the Amiga's window and menu manager.  
The command interface provides both a means of sharing data and a method 
of controlling an applications program.
\idxline{command interface}{}

\hp{Function libraries} provide a mechanism for calling external code 
as part of an \AR\ program's task context.  
The linkages for such calls are established dynamically at run time rather 
than when the program is linked,
so each function library must include an entry point to match function names 
with the address of the function to be called.
\idxline{function libraries}{}

\hp{Function hosts} are external tasks that manage a public message port
for communicating with \AR\ or other programs.
Both function hosts and function libraries are managed by the Library List,
which provides a prioritized search mechanism for resolving function names.
Function hosts may be used as a gateway into a network to provide a
remote procedure call facility.
\AR\ imposes no constraints on the internal operations of a function host,
except to require that message packets be returned with an appropriate code.
\idxline{function hosts}{}

The \hp{resident process} acts as the hub for communications between \AR\
and external entities.
It opens and manages a public message port named ``REXX,''
and provides a number of support services.
Note that the resident process is itself a ``host application'' whose 
function it is to launch \AR\ programs and maintain global resources.  
The activation structures for all \AR\ programs are linked into a list 
maintained by the resident process, 
and in principle their complete internal states are accessible to external
programs.
\idxline{resident process}{}

The \AR\ interpreter is structured as an Amiga shared library and includes
entry points specifically designed to help implement an interface to \AR.
Functions are available to create and delete message packets,
argument strings, and other resources.
Software developers are urged to use these library routines whenever possible,
as they provide ``safe'' access to the internal structures.
The \AR\ Systems Library functions are documented in Appendix C.
The distribution disk contains the INCLUDE files required to work with
the library and data structures.  
\idxline{shared library}{}

\section{\chapterno-}{Basic Structures}%
Most developers will need to work with only two of the data structures
used by \AR.
The {\bf RexxArg} structure is used for all of the strings manipulated by
the interpreter.
It is usually passed as an \hp{argstring}, 
a pointer offset from the structure base that may be treated like an 
ordinary string pointer.
The {\bf RexxMsg} structure is an extension of an EXEC {\bf Message},
and is the message packet used for all communications with external programs.
\idxline{RexxArg structure}{}
\idxline{RexxMsg structure}{}

\subsub{Argstrings}.
All \AR\ strings are maintained as {\bf RexxArg} structures, 
which are diagrammed in Table \chapterno .1 below.
Note that this actually a variable-length structure allocated for each
specific string length.
String parameters are sent in the form of \hp{argstrings},
a pointer to the string buffer area of the {\bf RexxArg} structure.
The string in the structure is always given a trailing null byte,
so that external programs can treat argstrings like a pointer to a 
null-terminated string.
Additional data about the string (its length, hash code, and attributes) 
are available at negative offsets from the argstring pointer.
\idxline{argstring}{}
\idxline{RexxArg structure}{}

\vskip 6pt\centerline{Table 10.1  The {\bf RexxArg} Structure}
\begverb
               STRUCTURE RexxArg,0
               LONG     ra_Size        ; allocated length
               UWORD    ra_Length      ; length of string
               UBYTE    ra_Flags       ; attribute flags
               UBYTE    ra_Hash        ; hash code
               STRUCT   ra_Buff,8      ; buffer (argstring points here)
#endverb%

Library functions are available to create and delete argstrings,
and for converting integers into argstring format.
The function |CreateArgstring()| allocates a structure and copies a 
string into it,
and returns an argstring pointer to the structure.
The function |DeleteArgstring()| can be used to release an argstring
when it is no longer needed.

\subsub{Message Packets}.
All communications between \AR\ and external programs are mediated with
message packets, 
whose structure is diagrammed in Table \number\chapterno .2 below.
Functions are provided in the \AR\ Systems Library to create, initialize, 
and delete these message packets.
Each packet sent from \AR\ to an external program is marked with a 
special pointer in its name field.  
This can be used to distinguish the message packets from those belonging
to other programs,
in case a message port is being shared.

Message packets are created using the |CreateRexxMsg()| function,
and can be released using the |DeleteRexxMsg()| function.
Note that the message packets passed by \AR\ to a host application
(as a command, for instance)
are identical to the packets the host would use to invoke an \AR\ program.
This commonality of design means that only one set of functions is needed
to create and delete message packets,
and that external programs can use the same routines that the interpreter uses
to handle the packets. 
\idxline{CreateRexxMsg() function}{}
\idxline{DeleteRexxMsg() function}{}

\subsub{Resource Nodes}.
A somewhat higher-level data structure called a ``resource node'' 
(a {\bf RexxRsrc} structure) is used extensively within \AR\ to maintain 
resource lists.
These nodes are variable-length structures that include the total allocated 
length as a field within the node,
and that also provide for an ``auto-delete'' function.
This latter capability allows the address of a clean-up function to be 
associated with the node so that an entire (possibly inhomogeneous) 
list of resource nodes can be deallocated with a single function call.
\idxline{RexxRsrc structure}{}

\vskip 6pt\centerline{Table 10.2  The {\bf RexxMsg} Structures}
\begverb
               STRUCTURE RexxMsg,MN_SIZE
               APTR     rm_TaskBlock   ; global pointer
               APTR     rm_LibBase     ; library pointer
               LONG     rm_Action      ; command code
               LONG     rm_Result1     ; primary result
               LONG     rm_Result2     ; secondary result
               STRUCT   rm_Args,16*4   ; arguments (ARG0-ARG15)
                                     ; the extension area
               APTR     rm_PassPort    ; forwarding port
               APTR     rm_CommAddr    ; host address
               APTR     rm_FileExt     ; file extension
               LONG     rm_Stdin       ; input stream
               LONG     rm_Stdout      ; output stream
               LONG     rm_avail       ; reserved
               LABEL    rm_SIZEOF      ; 128 bytes
#endverb

\section{\chapterno-}{Designing a Command Interface}%
The minimal command interface between \AR\ and an applications program 
requires only a public message port and a routine to process the commands 
received.
For most host applications this will require little extra machinery, 
as the program will probably already have several message ports for key
and menu events, timer messages, and so on.  
Processing the command strings should be relatively straightforward for 
command-oriented applications.
Hosts that are entirely menu-driven will require somewhat more additional
programming,
unless commands are supported only as simulated menu events.
The specific choice of which commands to support is always left up to
the applications designer, 
as \AR\ imposes no restrictions on the structure of the commands that 
can be issued.
\idxline{command interface,}{design of}

The basic sequence of events in the command interface begins when the 
host sends a command invocation message to the \AR\ resident process.  
This is usually in response to a direct input from the user, 
such as a command that was not recognized as one of the primitives 
supported by the host.
When the resident process receives the message packet, 
it spawns a new DOS process the run the macro program.
The command line is parsed to extract the command token (the first word), 
and the interpreter searches for a file that matches the command name.

Once a macro program file has been found,
it is executed by the interpreter and (usually) results in one or more
commands being issued back to the host application's public port.  
The macro program waits while each command is processed by the host, 
and takes appropriate actions if the return code indicates that an error 
occurred.  
Eventually the macro program finishes and returns the invocation message 
packet back to the host.

Error handling is an important consideration in the interface design.
Macro programs must receive meaningful return codes so that processing 
actions can be altered when errors occur.  
Normally, the host application should not return a message packet until the 
command has been processed and its error status is known.  
Hosts that support two streams of commands (from the user and from the 
command interface) will need a flag to indicate the source of 
each command.
Errors in user commands might normally be reported on the screen, 
but errors in \AR\ commands must be reported by setting the result field 
in the message packet.
\idxline{command interface,}{error handling}

Return codes should generally be chosen to follow the model of an 
\hp{error severity level},
with small integers representing relatively harmless conditions and larger
values indicating progressively more severe errors.
This will allow a characteristic failure level to be established within a
macro program,
so that insignificant errors can be ignored.
The choice of the specific return code values is left to the applications 
designer.

\subsection{Receiving Command Messages}%
Each host application must open a public message port to support the command
interface.
When a macro program issues a command to the host,
a message packet containing the command is sent to this public port.
The structure of these message packets is shown in Table \number\chapterno .2.  
The |rm\_Action| field will be set to |RXCOMM|,
and the |ARG0| parameter slot will contain the command as an argstring pointer.
Parameter slots |ARG1|--|ARG15| are not used for command messages.  
Two other fields are potentially of interest:  
the |rm\_RexxTask| field contains a pointer to the global data structure 
for the program that issued the command, 
and the |rm\_LibBase| field has the \AR\ library base address.
The fields in the extension area may also be of interest to the host program;
these are described later on.
Except for setting the result fields |rm\_Result1| and |rm\_Result2|, 
the host application should not alter the message packet.

\subsection{Result Fields}%
When the host program finishes processing the command, 
it must set the primary result field |rm\_Result1| to an
\hp{error severity level} or zero if no errors occurred.  
This is the field which will be assigned to the special variable |RC| in 
the macro program.
The secondary result field |rm\_Result2| should be set to zero unless a
result string (as described below) is being returned.
The packet can then be returned to the sender using the EXEC function 
|ReplyMsg()|.
\idxline{Result fields,}{setting values}
\idxline{ReplyMsg()}{}

In some cases a macro program may request a result string by setting the
|RXFB\_RESULT| modifier bit in the command code.  
If possible, 
the host application should then return the result as an argstring pointer
in the secondary result field |rm\_Result2|.
A result string should only be returned if \hp{explicitly requested} 
and if \hp{no errors occurred} during the call (|rm\_Result1| set to zero.)
Failure to observe these rules will result either in memory loss or in
corruption of the system free-memory list.

\subsection{Multiple Host Processes}%
Many applications programs support concurrent activities on several sets of 
data. 
For example, most text editors allow several files to be edited at once.
A command issued from a particular instance of the editor might invoke an
\AR\ macro program, 
so clearly any commands issued from that macro would have to be directed to
the correct instance of the editor.
\AR\ provides for this by allowing the applications program to declare 
an initial host address when a program is invoked.  
A separate message port would be opened for each instance of the host 
application, 
and this port would be named as the initial host address for all invocations 
from that instance.  
In the example above, 
if the editor opened two ports named ``MyEdit1'' and ``MyEdit2,'' 
then programs invoked by the ``MyEdit1'' instance would send commands back 
to the ``MyEdit1'' port.

\subsub{Multiple Message Ports}.
Host applications are not limited to having a single message port for
commands.
If several different kinds of commands are to be received, 
it might be appropriate to set up more than one port.  
Macro programs would then use the |ADDRESS| instruction to direct commands 
to the appropriate port.  
The different ports could be used simultaneously, 
since \AR\ programs execute as separate tasks. 
\idxline{ADDRESS instruction}{}

\section{\chapterno-}{Invoking \AR\ Programs}%
\AR\ programs are invoked by sending a message packet to the resident process.  
Programs may be invoked as either a command or as a function.
The command mode of invocation is generally simpler,
as it requires setting only a few fields in the message packet.

\subsection{Message Packets}%
The structure of the message packet supported by \AR\ is shown in 
Table \number\chapterno .2.
This structure provides fields for passing arguments and for specifying 
overrides to various internal defaults.
The packets are cleared (set to 0) when allocated, 
and the client-supplied fields are never altered by \AR.
Message packets can be reused after being returned, 
and generally only one is required.  

\subsub{Command (Action) Code}.  
The |rm\_Action| field of the message packet determines the mode of invocation.
It can be set to either |RXCOMM| or |RXFUNC| for command or function mode, 
respectively.  
Several modifier flags can be used with the command code; 
these are described later in this chapter.
\idxline{Action code,}{in message packet}

\subsub{Argument Strings}.  
Command strings, function names,
and argument strings must be supplied as argstrings.
Strings can be conveniently packaged into argstrings using the 
|CreateArgstring()| library function, 
which takes a string pointer and a length as its arguments.
Argstrings point to a null-terminated string and may be treated like 
an ordinary string pointer in most cases.
In principle, a host application could build the argstrings directly,
but since the strings must remain unchanged for the duration of an \AR\
program, 
the host might need to maintain many such structures.  

The argstring pointer returned by |CreateArgstring()| is installed 
in the appropriate parameter slot of the message packet:  
|ARG0| for the command string or function name, 
and |ARG1|--|ARG15| for argument strings. 
Argstrings can be recycled after a packet has returned by calling the 
|DeleteArgstring()| function.

\subsub{Sending the Packet}.  
Once the required fields have been filled in, 
the host application can send the packet to the ``REXX'' public port
using the EXEC function |PutMsg()|.
The address of the ``REXX'' port can be obtained by a call to the |FindPort()| 
function,
but this address should not be cached internally,
since the port could close at any time.
To be absolutely safe,
the calls to |FindPort()| and |PutMsg()| should be bracketed by calls to
the EXEC routines |Forbid()| and |Permit()|.
This will exclude the slight possibility that the message port could close
in the few microseconds before the message packet is actually sent to the 
port address.
\idxline{PutMsg() function}{}
\idxline{Forbid() function}{}
\idxline{Permit() function}{}

After sending the packet the host can return to its normal processing, 
since the macro program will execute as a separate task.
In most cases it will be advisable to ``lock-out'' further user commands 
while the \AR\ program is running, 
to preserve the integrity of any shared data structures that may be 
accessed externally.

\subsection{Command Invocations}%
In the command mode of invocation the host supplies a command string 
consisting of a name token followed by an argument string.
\AR\ parses the string to extract the command name, 
which is usually the name of a program file.  
The default action is to use the remainder of the command string as the 
(single) argument string for the program.  
This may be overridden by requesting \hp{command tokenization},
which is done by setting the |RXFB\_TOKEN| modifier flag in the action code
of the message packet.  
In this case the entire command string will be parsed, 
and the program may have many argument strings.  
(There is no limit to the number of arguments that may be derived from the 
command string, 
since they don't have to fit into the parameter slots of the message packet.)
\idxline{command invocation}{}

The parsing process uses ``white space'' (blanks, tabs, etc.) as the token
separators, and has a several special features.

\subsub{Quoting Convention}.  
Either single (|'|) or double (|"|) quotes may be used to surround items that 
include ``white space'' and would otherwise be separated during parsing.  
Single quotes may appear within a double-quote-delimited token, 
and vice versa;
however, double-delimiter sequences are not accepted.  
The quotes are removed from the parsed token.  
An ``implicit'' quote at the end of the string is also recognized.
If the command string ends before the closing delimiter has been found, 
the null byte is accepted as the final delimiter.  
For example,

\begprog
look.rexx "Now is the time" "can't you see
#endverb%
is a command with the two argument strings ``|Now is the time|'' and 
``|can't you see|'' (but without the quotes.)
\idxline{quoting convention,}{for commands}

\subsub{String Files}.
If the command name (the first token of the string) is quoted, 
it is assumed to be a ``string file'' --- an \AR\ program in a string, 
rather than the name of a disk file.  
This is a convenient way to run very brief programs, 
although programs of any length may be stored this way.
If command tokenization has not been specified, 
the remainder of the string is not scanned and no quote characters are removed.
In this case the quoting convention is useful only for indicating 
``string file'' programs.
The entire command string can be declared as a ``string file'' by setting
the |RXFB\_STRING| modifier flag of the action code.  
When this flag is set,
no parsing at all is applied to the command.
\idxline{string file}{}

\subsub{Result Strings}.
Command invocations do not usually request a result string, 
but can do so by setting the |RXFB\_RESULT| modifier flag.  
The host application must be prepared to recycle the returned result string 
once it is no longer needed. 

\subsection{Function Invocations}%
In a function invocation the host application supplies a function name string
and from 0 to 15 argument strings.  
The name string is used to locate an external program file and may include
directory specifiers and a file extension.  
The actual argument count (not including the name string) must be placed
in the low-order byte of the command code.

This mode of invocation is normally used when a result string is expected
and the argument strings are conveniently available.
Note that a result does not have to be requested, however.

\subsub{Result Strings}.
Function invocations request a result string by setting the |RXFB\_RESULT|
modifier flag bit.
If no errors occurred and a result string was requested, 
the secondary result field in the returned packet will be a pointer to the 
result string.
However, if the program exited without supplying a result, 
the secondary field will be zero.

\subsub{String Files}.
The function name argument may specify a ``string file'' rather than the 
name of a filing system object.  
This is indicated by setting the |RXFB\_STRING| modifier flag.

\subsection{Search Order}%
The search for a program file matching a command or function name is normally 
a two-step process.
For each directory to be checked,
a search is made first with the current file extension appended to the
name string.
If this search fails,
the second search uses the unmodified name string.
The first step is skipped if the command or function name includes an
explicit file extension.

The default file extension is ``|.rexx|,''
but this can be changed by supplying a file extension string in an
extended message packet.  
Host applications will usually specify a file extension,
since it provides a convenient way to distinguish the macro programs that
are specific to that application.
Refer to the section on Extension Fields for further details.

The search path for a program depends on the way the program name was 
specified.
If an explicit device or directory specification precedes the program name,
only that directory will be searched.
For example,
the command-level invocation of \hbox{``|rx df0:s/test|''} will search
only the |df0:s| directory for a file named |test.rexx| or |test|.
If the program name does not include a path,
the search path begins with the current directory and proceeds to the
system |REXX:| directory.  
To further the above example,
invoking the program as \hbox{``|rx test 1 2 3|''} would search for the 
files |test.rexx|, |test|, |REXX:test.rexx|, and |REXX:test|, in that order.
\idxline{search path}{}

If an \AR\ program cannot be found, one alternative action may be taken.  
If the |rm\_PassPort| field of an extended packet was supplied, 
the message packet is passed along to that port, 
which might be the next process in a search chain.
Otherwise the message is returned with a ``Program not found'' error 
indication (error code 1.)

\subsection{Extension Fields}%
The {\bf RexxMsg} structure includes several ``extension fields'' that can 
be used to override various defaults when a program is invoked.
These extension fields can be filled in selectively, 
and only the non-zero values will override the corresponding default.  
\AR\ never modifies the extension area.

Host applications should supply values for the file extension and host 
address fields of the message packet.
The file extension affects which program files will match a given command name,
and allows macro programs specific to the host to be given distinctive names.
The host address must refer to a public message port,
and will usually indicate the host's own port.
Any appropriate (but usually short) strings can be chosen for these values.
Often, the name of the applications program itself can be used as its
host address and file extension.

\subsub{PassPort}.
Th |rm\_PassPort| field allows the search for a program to be ``passed along''
to another message port after checking for an \AR\ program.
If the command or function name doesn't resolve to an \AR\ program, 
the message packet is forwarded to the message port specified as the PassPort.
This allows applications to maintain control over the search order for 
external program files.

Note that the |rm\_PassPort| field must be the actual \hp{address} of a 
message port, 
rather than a name string.
The PassPort therefore does not have to be a public port,
but the port should be a secured resource,
since the message is sent directly to this address without checking to see
whether it is a valid message port.

\subsub{Host Address}.
The |rm\_CommAddr| field overrides the default initial host address, 
which is ``REXX.''  
The host address is the name of the message port to which commands will be 
directed,
and is supplied as a pointer to a null-terminated string.
Applications that support multiple instances of user data will usually create 
a separate message port for each instance.
The name of this port would then be supplied as the host address for any
commands issued from that instance.

\subsub{File Extension}.
The |rm\_FileExt| field is used to override the default file extension for 
\AR\ programs, which is ``REXX.''  
Host applications can use the file extension to distinguish the names of the
macro programs specific to that application.  
It is supplied as a pointer to a null-terminated string.

\subsub{Input and Output Streams}.
The default input and output streams for an \AR\ program are inherited from
the host application's process structure,
if the host is a process rather than just a task.
One or both of these streams may be overridden by supplying an appropriate 
value in the |rm\_Stdin| or |rm\_Stdout| fields.
The values supplied must be valid DOS filehandles, 
and must not be closed while the program is executing.
The streams are installed directly into the program's process structure,
replacing the prior values.
\idxline{input stream}{}
\idxline{output stream}{}

The output stream is also used as the default tracing stream for the program. 
If interactive tracing is to be used in a program, 
the output stream should refer to a console device, 
since it will be used for input as well.

In the event that an \AR\ program is invoked by an EXEC task, 
rather than by an DOS process,
the extension field streams are the only way that the launched program
can be given default I/O streams.

\subsection{Interpreting the Result Fields}%
The message packet that invoked an \AR\ program is returned to the client 
when the program finishes.
The two result fields will contain error codes or possibly a result string.  
The interpretation of the result fields depends partly on the mode of 
invocation.
If the primary result field |rm\_Result1| is zero,
the program executed normally and the secondary field |rm\_Result2| will 
contain a pointer to a result string,
assuming that one was requested (and available.)  
\idxline{result fields,}{interpretation of}

If the primary result is non-zero,
it represents either an error severity level or else the return code from 
a command invocation.  
The two cases can be distinguished by examining the secondary result.  
If the secondary field is also non-zero, 
an error occurred and the secondary field is an \AR\ error code.  
If the secondary result is zero,
then the primary result is the return code passed by an ``|EXIT rc|'' or
``|RETURN rc|'' instruction in the program.
The application program can use this return code either as an error 
indication or to initiate some particular processing action.

Result strings are always returned as an argstring and become the property
(that is, responsibility) of the host.  
When the string is no longer needed,
it can be released using the |DeleteArgstring()| function.

Errors occurring in macro programs should usually be reported to the user.
Explanatory messages are available for all \AR\ error codes, 
and can be obtained by calling the \AR\ Systems Library function |ErrorMsg()|.

\section{\chapterno-}{Communicating with the Resident Process}%
All communications with the resident process are handled by passing message 
packets, 
which were previously diagrammed in Table \number\chapterno .2.  
The packet has a command field that describes the action to be performed and 
parameter fields that are specific to the command.  
Message packets are processed as they are received, 
and are then either returned to the sender or passed along to another process 
(in the case of a program invocation.)
The packet includes two result fields that are used to return error codes or 
result strings.  
The parameter fields of the message packet may contain either (long) integer
values or pointers to argument strings.  
String arguments are assumed to be argstring pointers unless otherwise 
specified.

\subsection{Command (Action) Codes}%
The command codes that are currently implemented in the resident process 
are described below.  
Commands are listed by their mnemonic codes,
followed by the valid modifier flags.
The final code value is always the logical OR of the code value and all of
the modifier flags selected.
The command code is installed in the |rm\_Action| field of the message packet.

\usage{|RXADDCON [RXFB\_NONRET]|}
This code specifies an entry to be added to the Clip List.  
Parameter slot |ARG0| points to the name string,
slot |ARG1| points to the value string, 
and slot |ARG2| contains the length of the value string.
The name and value arguments do not need to be argstrings, 
but can be just pointers to storage areas.
The name should be a null-terminated string, 
but the value can contain arbitrary data including nulls.
\idxline{RXADDCON action code}{}
\idxline{Clip List}{adding entries}

\usage{|RXADDFH [RXFB\_NONRET]|}
This action code specifies a function host to be added to the Library List.
Parameter slot |ARG0| points to the (null-terminated) host name string, 
and slot |ARG1| holds the search priority for the node.
The search priority should be an integer between 100 and -100 inclusive;
the remaining priority ranges are reserved for future extensions.
If a node already exists with the same name,
the packet is returned with a warning level error code.
Note that no test is made at this time as to whether the host port exists.
\idxline{RXADDFH action code}{}
\idxline{Library List}{adding entries}

\usage{|RXADDLIB [RXFB\_NONRET]|}
This code specifies an entry to be added to the Library List.
Parameter slot |ARG0| points to a null-terminated name string referring
either to a function library or a function host.
Slot |ARG1| is the priority for the node and should be an integer between 
100 and -100 inclusive;
the remaining priority ranges are reserved for future extensions.
Slot |ARG2| contains the entry point offset and slot |ARG3| is the library
version number.
If a node already exists with the same name,
the packet is returned with a warning level error code.
Otherwise,
a new entry is added and the library or host becomes available to \AR\ 
programs.
Note that no test is made at this time as to whether the library exists
and can be opened.
\idxline{RXADDLIB action code}{}
\idxline{Library List}{adding library}
\idxline{Library List}{adding host}

\usage{|RXCOMM [RXFB\_TOKEN] [RXFB\_STRING] [RXFB\_RESULT] [RXFB\_NOIO]|}
Specifies a command-mode invocation of an \AR\ program.
Parameter slot |ARG0| must contain an argstring pointer to the 
command string.
The |RXFB\_TOKEN| flag specifies that the command line is to be tokenized 
before being passed to the invoked program.
The |RXFB\_STRING| flag bit indicates that the command string is a 
``string file.''
Command invocations do not normally return result strings,
but the |RXFB\_RESULT| flag can be set if the caller is prepared to handle
the cleanup associated with a returned string.
The |RXFB\_NOIO| modifier suppresses the inheritance of the host's input
and output streams.
\idxline{RXCOMM action code}{}

\usage{|RXFUNC [RXFB\_RESULT] [RXFB\_STRING] [RXFB\_NOIO]| argcount}
This command code specifies a function invocation.  
Parameter slot |ARG0| contains a pointer to the function name string, 
and slots |ARG1| through |ARG15| point to the argument strings, 
all of which must be passed as argstrings.
The lower byte of the command code is the argument count; 
this count excludes the function name string itself.  
Function calls normally set the |RXFB\_RESULT| flag,
but this is not mandatory.
The |RXFB\_STRING| modifier indicates that the function name string is 
actually a ``string file''.
The |RXFB\_NOIO| modifier suppresses the inheritance of the host's input
and output streams.
\idxline{RXFUNC action code}{}

\usage{|RXREMCON [RXFB\_NONRET]|}
This code requests that an entry be removed from the Clip List.
Parameter slot |ARG0| points to the null-terminated name to be removed.
The Clip List is searched for a node matching the supplied name, 
and if a match is found the list node is removed and recycled.
If no match is found the packet is returned with a warning error code.
\idxline{Clip List}{removing entry}
\idxline{RXREMCON action code}{}

\usage{|RXREMLIB [RXFB\_NONRET]|}
This command removes a Library List entry.
Parameter slot |ARG0| points to the null-terminated string specifying the
library to be removed.
The Library List is searched for a node matching the library name, 
and if a match is found the node is removed and released.
If no match is found the packet is returned with a warning error code.
The library node will not be removed if the library is currently being used
by an \AR\ program.
\idxline{Library List}{removing entry}
\idxline{RXREMLIB action code}{}

\usage{|RXTCCLS [RXFB\_NONRET]|}
This code requests that the global tracing console be closed.  
The console window will be closed immediately unless one or more 
\AR\ programs are waiting for input from the console.
In this event,
the window will be closed as soon as the active programs are no longer 
using it. 
\idxline{RXTCCLS action code}{}

\usage{|RXTCOPN [RXFB\_NONRET]|}
This command requests that the global tracing console be opened.  
Once the console is open,
all active \AR\ programs will divert their tracing output to this console.  
Tracing input (for interactive debugging) will also be diverted to the new 
console.
Only one console can be opened; 
subsequent |RXTCOPN| requests will be returned with a warning error message.
\idxline{RXTCOPN action code}{}

\subsection{Modifier Flags}%
Command codes may include modifier flags to select various processing options.
Modifier flags are specific to certain commands,
and are ignored otherwise.  

\subsub{|RXFB\_NOIO|}.  
This modifier is used with the |RXCOMM| and |RXFUNC| command codes to suppress 
the automatic inheritance of the host's input and output streams.
\idxline{RXFB\_NOIO modifier}{}

\subsub{|RXFB\_NONRET|}.  
Specifies that the message packet is to be recycled by the resident process 
rather than being returned to the sender.  
This implies that the sender doesn't care about whether the requested action 
succeeded, 
since the returned packet provides the only means of acknowledgement.
Message packets are released using the library function |DeleteRexxMsg()|.
\idxline{RXFB\_NONRET modifier}{}

\subsub{|RXFB\_RESULT|}.  
This modifer is valid with the |RXCOMM| and |RXFUNC| commands, 
and requests that the called program return a result string.  
If the program |EXIT|s (or |RETURN|s) with an expression,
the expression result is returned to the caller as an argstring.
It is then the caller's responsibility to release the argstring when it is no 
longer needed;
this can be done using the library function |DeleteArgstring()|.
\idxline{RXFB\_RESULT modifier}{}

\subsub{|RXFB\_STRING|}.  
This modifer is valid with the |RXCOMM| and |RXFUNC| command codes.  
It indicates that the command or function argument (in slot ARG0) is a 
``string file'' rather than a file name.
\idxline{RXFB\_STRING modifier}{}

\subsub{|RXFB\_TOKEN|}.  
This flag is used with the |RXCOMM| code to request that the command 
string be completely tokenized before being passed to the invoked program.  
Programs invoked as commands normally have only a single argument string.  
The tokenization process uses ``white space'' to separate the tokens,
except within quoted strings.  
Quoted strings can use either single or double quotes,
and the end of the command string (a null character) is considered as an 
implicit closing quote.
\idxline{RXFB\_TOKEN modifier}{}

\subsection{Result Fields}
The resident process uses the standard command-level conventions for the
primary return code installed in |rm\_Result1|.
Minor or warning errors are indicated by a value of 5,
and more serious errors are returned as values of 10 or 20.
The secondary result field |rm\_Result2| will either be zero or an \AR\
error code if applicable.

Note that |RXCOMM| and |RXFUNC| messages are returned directly by the invoked
macro program,
rather than by the resident process.

\section{\chapterno-}{External Function Libraries}%
\AR\ supports external function libraries as a mechanism for user-defined 
extensions to the language.  
Function libraries may be written and maintained by users or applications 
developers.

\subsection{Design Considerations}%
There are several different purposes for which a function library might be 
designed.  
In the simplest case,
a library could be used to extend the string manipulation or mathematical
capabilities of the language by defining new functions.  
Such a library could be entirely self-contained or might call other 
system libraries to perform specific operations.

Another alternative would be to build a library that interacts closely with 
an external applications program.
This could allow specific operations in the host application to be be 
performed as function calls rather than as commands.
There are several advantages to this approach,
as it avoids the need to parse command strings and does not require 
the multiple task context changes associated with message-passing.  
The library might include entry points for specific operations as well as 
functions to support special processing required by the applications program.

Function libraries can also serve as bridges to other system or applications 
libraries.  
For example, if a program needed to call the functions in a graphics library, 
a bridge library could be built to match the function names in the program
with the appropriate entry point in the graphics library.  
A related possibility would be to use \AR\ as a test driver for a program
under development.  
Once the query table and parameter passing mechanisms for the function 
library have been built, 
new routines under development could be tested by just adding a table entry.
Since building test programs is often very time-consuming,
the flexibility and interactive debugging capabilities of \AR\ make it
an attractive alternative to compiled languages like ``C.''

Regardless of the intended application,
all function libraries share a common structure.
The initial design follows that of the standard EXEC shared library, 
with the three required entry points |Open|, |Close|, and |Expunge|,
plus a reserved slot.  
The library must also have a ``query'' entry point, 
which serves to match the name supplied by \AR\ with the intended function.  
Typically, this will consist of a table of function names and a routine 
to search for the specified one.

\subsub{Reentrancy}.
Functions libraries should be designed to be fully reentrant,
since any number of \AR\ programs may be running at any time.
If this is not feasible due to other design constraints,
the query function should include a lockout mechanism to prevent multiple
calls to the library routines.
\idxline{reentrant,}{requirement for}

\subsection{Calling Convention}%
The library's query function will be called from the interpreter's context
with the address of a message packet in register A0 and the library base 
in A6.
The structure of the message packet is the same as that in Table 
\number\chapterno .2,
but note that although a message packet is used to carry the arguments,
it is not queued at a message port and does not need to be unlinked.  
The name of the function to be called is carried in the |ARG0| parameter slot.
The query function must search for this function name and, 
if the name cannot be found,
must return an error code of 1 (``Program not found'') in register D0.  
The library will then be closed and the search continued in the next 
function library.
The query function should not modify any fields within the message packet, 
as it must be passed along to the next library until the function is located.
\idxline{function libraries,}{calling convention}

\subsection{Parameter Conversion}%
Once the requested function has been found,
the query function may need to transform the parameters passed by \AR\ into
the form expected by the function.
Whether the parameter strings need to be converted depends on how they are 
to be used.
In some cases it may be sufficient just to forward a pointer to the
message packet to the called function,
while in other cases the query function may need to load parameters
into registers or to perform conversion operations.  
The parameters in |ARG1|--|ARG15| are always passed as argstrings, 
and may be treated like a pointer to a null-terminated string.  
Further attributes are stored at negative offsets from the argstring pointer,
and may be helpful in working with the string.
\idxline{function libraries,}{parameter conversion}

Numeric quantities are passed as strings of ASCII characters and will need 
to be converted to integer or floating-point format if arithmetic calculations
are to be performed.
The \AR\ Systems Library includes a limited set of functions to do parameter
conversions.

The actual parameter count can be obtained from the low-order byte of the 
|rm\_Action| field in the message packet.
The count never includes the function name itself (in |ARG0|),
but does include arguments specified as ``defaults.''
Such arguments will have a zero value in the corresponding parameter slot.

Note that the parameter block of the message packet,
containing the fields |ARG0|--|ARG15|,
is structured like the argument array expected by the |main(argc,argv)| 
function of a ``C'' program.
This suggests a simple way that a function library could provide a bridge 
to a series  of ``C'' programs.
The query function would need only to determine the address of the called
function,
and then push the parameter block address and argument count onto the
program stack.

\subsection{Returned Values}%
Each library function must return an error code and a value string.
The error code is returned in register DO,
and should be 0 if no errors occurred.  
The value string must be returned as an argstring pointer in register A1,
unless D0 indicates that an error occurred during the call.
The mechanisms for creating the proper return values can be made part of the
query function,
so that all functions in the library share a common return path.
\idxline{function libraries,}{returned values}

\section{\chapterno-}{Direct Manipulation of Data Structures}%
All of the data structures maintained by the resident process are built
into the \AR\ Systems Library base and are therefore accessible to 
external programs.
The Task List in the {\bf RexxBase} structure links the global data structures 
for all currently active \AR\ programs.
This linkage uses the node contained in the message port of the {\bf RexxTask}
structure, 
rather than at the head of the structure.  
The {\bf RexxTask} structure is the global data structure and initial storage 
environment for the \AR\ program, 
and all descendant storage environments are linked into the Environment List.
The linkage of internal data structures is such that the complete internal 
state of all \AR\ programs can be reached starting from the library base 
pointer.
\idxline{RexxTask structure}{}

Two library functions, |LockRexxBase()| and |UnlockRexxBase()|,
are provided to mediate access to the global structures.
The structure base should be locked before reading any of the data items
or traversing any of the lists.
The present version of these functions provides only a global lock, 
but future extensions will allow individual resouces to be locked.
\idxline{LockRexxBase()}{}
\idxline{UnlockRexxBase()}{}

In general it should not be necessary to manipulate directly any of these 
data structures.
Functions have been provided in the \AR\ Systems Library to perform all of 
the operations required to interface external program to the \AR\ system.
It is therefore recommended that applications developers avoid using any
of the internal structures except as provided through the library functions.

\thatsit
