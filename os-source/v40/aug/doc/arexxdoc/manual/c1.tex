%
%	The ARexx User's Manual
%
% Copyright � 1987 by William S. Hawes.  All Rights Reserved.
%
\ifx\radeye\fmtversion\subdoc\else\input pmac \fi

\chapter{Chapter}{1}{What is \AR?}%
\AR\ is a high-level language useful for prototyping, software integration, 
and general programming tasks.  
It is an implementation of the REXX language described by M. F. Cowlishaw in
{\it The REXX Language: A Practical Approach to Programming}
(Prentice-Hall, 1985), and follows the language definition closely.
\AR\ is particularly well suited as a command language.
Command programs, sometimes called ``scripts'' or ``macros'', 
are widely used to extend the predefined commands of an operating system 
or to customize an applications program.

As a programming language, 
\AR\ can be useful to a wide cross section of users.  
For the novice programmer, 
\AR\ is an easy-to-learn yet powerful language that serves as a good
introduction to programming techniques.  
Its source-level debugging facilities will help take some of the mystery 
out of how programs work (or don't work, as is more frequently the case.)

For the more sophisticated user, 
\AR\ provides the means to build fully integrated software packages, 
combining different applications programs into an environment tailored 
to their needs.  
A common command language among applications that support \AR\ will bring 
uniformity to procedural interfaces, 
much as the Amiga's Intuition provides uniformity in the graphical interface.

Finally, for the software developer, 
\AR\ offers a straightforward way to build fully programmable applications 
programs.
Developers can concentrate their efforts on making the basic operations 
of their programs fast and efficient, 
and let the end user add the frills and custom features.

\section{\chapterno-}{Language Features}%
Some of the important features of the language are:
\idxline{Language features}{}

\subsub{Typeless Data}.
Data are treated as typeless character strings.  
Variables do not have to be declared before being used, 
and all operations dynamically check the validity of the operands.  
\idxline{typeless}{}

\subsub{Command Interface}.
\AR\ programs can issue commands to external programs that provide a
suitable command interface.
Any software package that implements the command interface is then fully 
programmable using \AR, 
and can be extended and customized by the end user.
\idxline{command interface}{}

\subsub{Tracing and Debugging}.
\AR\ includes source-level debugging facilities that allow the programmer
to see the step-by-step actions of a program as it runs,
thereby reducing the time required to develop and test programs.  
An internal interrupt system permits special handling of errors that would 
otherwise cause the program to terminate.
\idxline{tracing}{}
\idxline{interrupts}{}

\subsub{Interpreted Execution}.
\AR\ programs are run by an interpreter, 
so separate compilation and linking steps are not required.
This makes it especially useful for prototyping and as a learning tool.

\subsub{Function Libraries}.
External function libraries can be used to extend the capabilities of the 
language or as bridges to other programs.
Libraries also allow \AR\ programs to be used as ``test drivers'' for 
software development and testing.
\idxline{function library}{as bridge}
\idxline{function library}{as test driver}

\subsub{Automatic Resource Management}.
Internal memory allocation related to the creation and destruction of strings 
and other data structures is handled automatically.

\section{\chapterno-}{\AR\ on the Amiga}%
\AR\ was designed to run on the Amiga, 
and makes use of many of the features of its multitasking operating system.  
\AR\ programs run as separate tasks and may communicate with each other 
or with external programs.
The interpreter follows the design guidelines expected of well-behaved 
programs in a multitasking environment:  
specifically, it uses as little memory as possible and is careful to return 
resources to the operating system when they are no longer needed.  
Memory requirements were minimized by implementing the entire \AR\ system 
as a shared library,
so that only one copy of the program code must be loaded.

\section{\chapterno-}{Further Information}%
The aforementioned book by M. F. Cowlishaw is highly recommended to those
interested in further information about REXX.
It presents an interesting discussion of the design and development of the 
language.

\thatsit
