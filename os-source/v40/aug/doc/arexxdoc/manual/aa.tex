%
%	The ARexx User's Manual
%
% Copyright � 1987 by William S. Hawes.  All Rights Reserved.
%
\ifx\radeye\fmtversion\subdoc\else\input pmac \fi

\chapter{Appendix}{A}{Error Messages}%
When the \AR\ interpreter detects an error in a program, 
it returns an error code to indicate the nature of the problem.  
Errors are normally handled by displaying the error code, 
the source line number where the error occurred, 
and a brief message explaining the error condition.  
Unless the |SYNTAX| interrupt has been previously enabled 
(using the |SIGNAL| instruction), 
the program then terminates and control returns to the caller.
Most syntax and execution errors can be trapped by the |SYNTAX| interrupt, 
allowing the user to retain control and perform whatever special error 
processing is required.
Certain errors are generated outside of the context of an \AR\ program, 
and therefore cannot be trapped by this mechanism.  
Refer to chapter 7 for further information on error trapping and processing.
\vskip-\lastskip
\idxline{SYNTAX interrupt,}{error processing}%
\idxline{SIGNAL instruction}{}%

Associated with each error code is a \hp{severity level} that is reported 
to the calling program as the primary result code.
The error code itself is returned as the secondary result.  
The subsequent propagation or reporting of these codes is of course 
dependent on the external (calling) program.
\idxline{severity level,}{with error code}

The following pages list all of the currently-defined error codes, 
along with the associated severity level and message string.

\errormsg{1}{5}{Program not found}%
The named program could not be found, or was not an \AR\ program.  
\AR\ programs are expected to start with a ``|/*|'' sequence.  
This error is detected by the external interface and cannot be trapped by 
the |SYNTAX| interrupt.

\errormsg{2}{10}{Execution halted}%
A control-C break or an external halt request was received and the program
terminated.
This error will be trapped if the |HALT| interrupt has been enabled.
\idxline{HALT interrupt}{}

\errormsg{3}{20}{Insufficient memory}%
The interpreter was unable to allocate enough memory for an operation.  
Since memory space is required for all parsing and execution operations,
this error cannot usually be trapped by the |SYNTAX| interrupt.

\errormsg{4}{10}{Invalid character}%
A non-ASCII character was found in the program.  
Control codes and other non-ASCII characters may be used in a program 
by defining them as hex or binary strings.  
This is a scan phase error and cannot be trapped by the |SYNTAX| interrupt.

\errormsg{5}{10}{Unmatched quote}%
A closing single or double quote was missing.  
Check that each string is properly delimited.  
This is a scan phase error and cannot be trapped by the SYNTAX interrupt.

\errormsg{6}{10}{Unterminated comment}%
The closing ``|*/|'' for a comment field was not found.  
Remember that comments may be nested, 
so each ``|/*|'' must be matched by a ``|*/|.''  
This is a scan phase error and cannot be trapped by the |SYNTAX| interrupt.

\errormsg{7}{10}{Clause too long}%
A clause was too long for the internal buffer used as temporary storage.  
The source line in question should be broken into smaller parts.  
This is a scan phase error and cannot be trapped by the |SYNTAX| interrupt.

\errormsg{8}{10}{Invalid token}%
An unrecognized lexical token was found, 
or a clause could not be properly classified.
This is a scan phase error and cannot be trapped by the |SYNTAX| interrupt.

\errormsg{9}{10}{Symbol or string too long}%
An attempt was made to create a string longer than the maximum supported 
by the interpreter.  
The implementation limits for internal structures are given in Appendix B.

\errormsg{10}{10}{Invalid message packet}%
An invalid action code was found in a message packet sent to the \AR\
resident process.  
The packet was returned without being processed.  
This error is detected by the external interface and cannot be trapped by
the |SYNTAX| interrupt.

\errormsg{11}{10}{Command string error}%
A command string could not be processed.  
This error is detected by the external interface and cannot be trapped by 
the |SYNTAX| interrupt.

\errormsg{12}{10}{Error return from function}%
An external function returned a non-zero error code.  
Check that the correct parameters were supplied to the function.

\errormsg{13}{10}{Host environment not found}%
The message port corresponding to a host address string could not be found.
Check that the required external host is active.

\errormsg{14}{10}{Requested library not found}%
An attempt was made to open a function library included in the Library List, 
but the library could not be opened.  
Check that the correct name and version of the library were specified 
when the library was added to the resource list. 
\idxline{Library List}{}

\errormsg{15}{10}{Function not found}%
A function was called that could not be found in any of the 
currently accessible libraries,
and could not be located as an external program.
Check that the appropriate function libraries have been added to the 
Libraries List.

\errormsg{16}{10}{Function did not return value}%
A function was called which failed to return a result string, 
but did not otherwise report an error.
Check that the function was programmed correctly, 
or invoke it using the |CALL| instruction.

\errormsg{17}{10}{Wrong number of arguments}%
A call was made to a function which expected more (or fewer) arguments.  
This error will be generated if a Built-In or external function is called 
with more arguments than can be accomodated in the message packet used for 
external communications.

\errormsg{18}{10}{Invalid argument to function}%
An inappropriate argument was supplied to a function, 
or a required argument was missing.  
Check the parameter requirements specified for the function.

\errormsg{19}{10}{Invalid PROCEDURE}%
A |PROCEDURE| instruction was issued in an invalid context.  
Either no internal functions were active, 
or a |PROCEDURE| had already been issued in the current storage environment.
\idxline{PROCEDURE instruction}{}

\errormsg{20}{10}{Unexpected THEN or WHEN}%
A |WHEN| or |THEN| instruction was executed outside of a valid context.
The |WHEN| instruction is valid only within a |SELECT| range, 
and |THEN| must be the next instruction following an |IF| or |WHEN|.
\idxline{THEN instruction}{}
\idxline{WHEN instruction}{}

\errormsg{21}{10}{Unexpected ELSE or OTHERWISE}%
An |ELSE| or |OTHERWISE| was found outside of a valid context.  
The |OTHERWISE| instruction is valid only within a |SELECT| range.
|ELSE| is valid only following the |THEN| branch of an |IF| range.  
\idxline{ELSE instruction}{}
\idxline{OTHERWISE instruction}{}

\errormsg{22}{10}{Unexpected BREAK, LEAVE, or ITERATE}%
The |BREAK| instruction is valid only within a |DO| range or inside an 
|INTERPRET|ed string.
The |LEAVE| and |ITERATE| instructions are valid only within an \hp{iterative} 
|DO| range.

\errormsg{23}{10}{Invalid statement in SELECT}%
A invalid statement was encountered within a |SELECT| range.  
Only |WHEN|, |THEN|, and |OTHERWISE| statements are valid within a 
|SELECT| range, 
except for the conditional statements following |THEN| or |OTHERWISE| clauses.

\errormsg{24}{10}{Missing or multiple THEN}%
An expected |THEN| clause was not found,
or another |THEN| was found after one had already been executed.
\idxline{THEN instruction,}{missing}

\errormsg{25}{10}{Missing OTHERWISE}%
None of the |WHEN| clauses in a |SELECT| succeeded, 
but no |OTHERWISE| clause was supplied.
\idxline{OTHERWISE instruction,}{missing}

\errormsg{26}{10}{Missing or unexpected END}%
The program source ended before an |END| was found for a |DO| or |SELECT| 
instruction, 
or an |END| was encountered outside of a |DO| or |SELECT| range.

\errormsg{27}{10}{Symbol mismatch}%
The symbol specified on an |END|, |ITERATE|, or |LEAVE| instruction did not
match the index variable for the associated |DO| range.  
Check that the active loops have been nested properly.

\errormsg{28}{10}{Invalid DO syntax}%
An invalid |DO| instruction was executed.  
An initializer expression must be given if a |TO| or |BY| expression is
specified,
and a |FOR| expression must yield a non-negative integer result.

\errormsg{29}{10}{Incomplete IF or SELECT}%
An |IF| or |SELECT| range ended before all of the required statements were 
found.
Check whether the conditional statement following a |THEN|, |ELSE|, 
or |OTHERWISE| clause was omitted.

\errormsg{30}{10}{Label not found}%
A label specified by a |SIGNAL| instruction, 
or implicitly referenced by an enabled interrupt, 
could not be found in the program source.  
Labels defined dynamically by an |INTERPRET| instruction or by interactive 
input are not included in the search.
\idxline{Label,}{error if missing}

\errormsg{31}{10}{Symbol expected}%
A non-symbol token was found where only a symbol token is valid.  
The |DROP|, |END|, |LEAVE|, |ITERATE|, and |UPPER| instructions may only be
followed by a symbol token, 
and will generate this error if anything else is supplied.  
This message will also be issued if a required symbol is missing.
\idxline{Symbol token}{}

\errormsg{32}{10}{Symbol or string expected}%
An invalid token was found in a context where only a symbol or string is
valid.

\errormsg{33}{10}{Invalid keyword}%
A symbol token in an instruction clause was identified as a keyword, 
but was invalid in the specific context.

\errormsg{34}{10}{Required keyword missing}%
An instruction clause required a specific keyword token to be present, 
but it was not supplied.  
For example, 
this message will be issued if a |SIGNAL ON| instruction is not followed by 
one of the interrupt keywords (e.g. |SYNTAX|.)

\errormsg{35}{10}{Extraneous characters}%
A seemingly valid statement was executed, 
but extra characters were found at the end of the clause.

\errormsg{36}{10}{Keyword conflict}%
Two mutually exclusive keywords were included in an instruction clause, 
or a keyword was included twice in the same instruction.

\errormsg{37}{10}{Invalid template}%
The template provided with an |ARG|, |PARSE|, or |PULL| instruction 
was not properly constructed.  
Refer to Chapter 8 for a description of template structure and processing.
\idxline{ARG instruction}{}
\idxline{PARSE instruction}{}
\idxline{PULL instruction}{}

\errormsg{38}{10}{Invalid TRACE request}%
The alphabetic keyword supplied with a |TRACE| instruction or as the argument 
to the |TRACE()| Built-In function was not valid.
Refer to Chapter 7 for the valid |TRACE| options.

\errormsg{39}{10}{Uninitialized variable}%
An attempt was made to use an uninitialized variable while the |NOVALUE|
interrupt was enabled.

\errormsg{40}{10}{Invalid variable name}%
An attempt was made to assign a value to a fixed symbol.

\errormsg{41}{10}{Invalid expression}%
An error was detected during the evaluation an expression.  
Check that each operator has the correct number of operands, 
and that no extraneous tokens appear in the expression.  
This error will be detected only in expressions that are actually evaluated.
No checking is performed on expressions in clauses that are being skipped.

\errormsg{42}{10}{Unbalanced parentheses}%
An expression was found with an unequal number of opening and closing
parentheses.

\errormsg{43}{43}{Nesting limit exceeded}%
The number of subexpressions in an expression was greater than the maximum
allowed.
The expression should be simplified by breaking it into two or more
intermediate expressions.
\idxline{nesting,}{subexpression limit}

\errormsg{44}{10}{Invalid expression result}%
The result of an expression was not valid within its context.  
For example, this message will be issued if an increment or limit expression 
in a |DO| instruction yields a non-numeric result.

\errormsg{45}{10}{Expression required}%
An expression was omitted in a context where one is required.
For example, 
the |SIGNAL| instruction, if not followed by the keywords |ON| or |OFF|, 
must be followed by an expression.

\errormsg{46}{10}{Boolean value not 0 or 1}%
An expression result was expected to yield a boolean result, 
but evaluated to something other than 0 or 1.
\idxline{Boolean Value}{}

\errormsg{47}{10}{Arithmetic conversion error}%
A non-numeric operand was used in a operation requiring numeric operands. 
This message will also be generated by an invalid hex or binary string.

\errormsg{48}{10}{Invalid operand}%
An operand was not valid for the intended operation.  
This message will be generated if an attempt is made to divide by 0, 
or if a fractional exponent is used in an exponentiation operation.

\thatsit
