%
%	The ARexx User's Manual
%
% Copyright � 1987 by William S. Hawes.  All Rights Reserved.
%
\ifx\radeye\fmtversion\subdoc\else\input pmac \fi
%
%	Now for the Introduction
%
\chapter{Introduction}{}{}%
Welcome to \AR, an implementation of the REXX language for the Amiga computer.
\AR\ is a powerful programming tool, 
but one which by virtue of its clean syntax and sparse vocabulary is also 
easy to learn and easy to use.

\section{}{Organization of this Document}%
This document will attempt to fill the roles of User's Manual,
Language Reference, and Programmer's Guide.
The chapters that follow have been organized to provide a gentle introduction 
to the language.

\startlist
\item{$\bullet$} Chapter 1, {\it What is \AR?},
gives an overview of the \AR\ language and its implementation on the Amiga.

\item{$\bullet$} Chapter 2, {\it Getting Acquainted},
tells how to install \AR\ on your Amiga and presents several example 
programs to illustrate the features of the language.

\item{$\bullet$} Chapter 3, {\it Elements of the Language},
introduces the language structure and syntax.

\item{$\bullet$} Chapter 4, {\it Instructions},
describes the action statements of \AR.

\item{$\bullet$} Chapter 5, {\it Commands},
describes the program statements used to communicate with external programs.

\item{$\bullet$} Chapter 6, {\it Functions},
explains how functions are called and documents the Built-In Function library.

\item{$\bullet$} Chapter 7, {\it Tracing and Interrupts},
describes the source-level debugging features useful for developing and 
testing programs.

\item{$\bullet$} Chapter 8, {\it Parsing and Templates},
describes the instructions used to extract words or fields from strings.

\item{$\bullet$} Chapter 9, {\it The Resident Process},
describes the capabilities of the global communications and resources manager.

\item{$\bullet$} Chapter 10, {\it Interfacing to \AR}, 
describes how to design and implement an interface between \AR\ and an
external program.

\item{$\bullet$} Appendix A, {\it Error Messages},
lists the error messages issued by the interpreter.

\item{$\bullet$} Appendix B, {\it Limits and Compatibility},
discusses the compatibility of \AR\ with the language standard.

\item{$\bullet$} Appendix C, {\it The \AR\ Systems Library},
documents the functions in the \AR\ systems library.

\item{$\bullet$} Appendix D, {\it The Support Library},
documents the library of Amiga-specific functions.

\item{$\bullet$} Appendix E, {\it Distribution Files},
lists the files on the distribution disk.
\endlist
\noindent%
Finally, a Glossary and an Index are provided.

\subsection{Using this Manual}%
If you are new to the REXX language, or perhaps to programming itself,
you should review chapters 1 through 4 and then play with \AR\ by running
some of the sample programs given in chapter 2.
Further examples are available in the |:rexx| directory of the distribution
disk.

If you are already familiar with REXX you may wish to skip directly 
to chapter 5, 
which begins to present some of the system-dependent features of this 
implementation.
A summary of the compatibility of \AR\ with the language definition is 
contained in Appendix B.

\subsection{Typographic Conventions}%
Describing a language is sometimes difficult because of the multiple and
changing contexts involved.
To help clarify the presentation here,
a simple typographic convention has been adopted throughout the document.
All of the terms and words specific to the REXX language, 
as well as the program examples and computer input and output, 
have been set |in typewriter font like this|.
This should help to distinguish the language keywords and examples from the
surrounding text.

\section{}{Future Directions}%
\AR, like most software products, 
will probably evolve somewhat over the next few years as new features 
are added, old bugs are removed, and market imperatives become more apparent.
While the core language will probably undergo few modifications,
many capabilities will be added to the function libraries supported by \AR.
Your comments and suggestions for improvements to \AR\ are most welcome.

The author sincerely hopes that other software developers will consider 
using \AR\ with their products.
The advantages of having a rich variety of software products sharing a 
common user interface and a common procedural interface cannot be overstated.
This is the underlying promise of the Amiga's multitasking capability,
and that which most sets it apart from other inexpensive computers.

\subsub{Example Programs}.
One of the best ways to learn a computer language is to study examples
written by more experienced programmers.
The \AR\ distribution disk includes a few example programs in the |:rexx|
directory,
and more programs will be added in future releases.

If you have written a REXX language program (for any computer) that you
think would be of interest to a more general audience,
please send it to the author for consideration.
Programs should be of interest either in terms of their specific 
functionality or as an example of programming technique.
Each program submitted should include an author credit and a few lines
of commentary on its intended function.

\thatsit
