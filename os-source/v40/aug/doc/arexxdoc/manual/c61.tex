%
%	The ARexx User's Manual
%
% Copyright � 1987 by William S. Hawes.  All Rights Reserved.
%
\ifx\radeye\fmtversion\subdoc\else\input pmac \fi
%
% First section of a long chapter!
%
\chapter{Chapter}{6}{Functions}%
The basic concept of a function is a program or group of statements that
will be executed whenever the function name appears in a certain context.
Functions are an important building block of most computer languages in that 
they allow \hp{modular programming} ---
the ability to build a large program from a series of smaller,
more easily developed modules.
In \AR\ a function may be defined as part of (internal to) a program,
as part of a library, or as a separate external program.

\section{\chapterno-}{Syntax and Search Order}%
Function calls in an expression are defined syntactically as a symbol or 
string followed immediately by an open parenthesis.  
The symbol or string (taken as a literal) specifies the \hp{function name}, 
and the open parenthesis begins the \hp{argument list}.
Between the opening and eventual closing parentheses are zero or more
argument expressions, separated by commas,
that supply the data being passed to the function.
For example,

\begprog
CENTER('title',20)
ADDRESS()
'AllocMem'(256*4,1)
#endverb%
are all valid function calls.
Each argument expression is evaluated in turn and the resulting strings are
passed as the argument list to the function. 
There is no limit to the number of arguments that may be passed to an internal
function,
but calls to Built-In or external functions are limited to a maximum of 
15 arguments.  
Note that each argument expression, 
while often just a single literal value, 
can include arithmetic or string operations or even other function calls.  
Argument expressions are evaluated from left to right.
\idxline{Search order,}{for function calls}

Functions can also be invoked using the |CALL| instruction.
The syntax of this form is slightly different,
and is described in Chapter 4.
The |CALL| instruction can be used to invoke a function that may not return 
a value.

\subsection{Search Order}%
Function linkages in \AR\ are established dynamically at the time of the 
function call.
A specific search order is followed until a function matching the 
name symbol or string is found.
If the specified function cannot be located,
an error is generated and the expression evaluation is terminated.
The full search order is:

\startlist
\item{1.} {\it Internal Functions}.
The program source is examined for a label that matches the function name.  
If a match is found,
a new storage environment is created and control is transferred to the label.

\item{2.} {\it Built-In Functions}.
The Built-In function library is searched for the specified name.  
All of these functions are defined by uppercase names, 
and the library has been specially organized to make the search as 
efficient as possible.

\item{3.} {\it Function Libraries and Function Hosts}.
The available function libraries and function hosts are maintained in a
prioritized list,
which is searched starting at the highest priority until the requested 
function is found or the end of the list is reached.
Each function library is opened and called at a special entry point to 
determine whether it contains a function matching the given name.
Function hosts are called using a message-passing protocol similar to that
used for commands, 
and may be used as gateways for remote procedure calls to other machines
in a network.

\item{4.} {\it External \AR\ Programs}.
The final search step is to check for an external \AR\ program file by
sending an invocation message to the \AR\ resident process.
The search always begins in the current directory, 
and follows the same search path as the original \AR\ program invocation.  
The name matching process is not case-sensitive.
\endlist
\noindent%
Note that the function name-matching procedure may be case-sensitive for 
some of the search steps but not for others.  
The matching procedure used in a function library or function host
is left to the discretion of the applications designer.
Functions defined with mixed-case names must be called using a string token, 
since symbol names are always translated to uppercase.

The full search order is followed whenever the function name is defined 
by a symbol token.
However, the search for internal functions is bypassed if the name is
specified by a string token.  
This allows internal functions to usurp the names of external functions, 
as in the following example:

\begprog
CENTER:                         /* internal "CENTER" */
arg string,length               /* get arguments     */
length = min(length,60)         /* compute length    */
return 'CENTER'(string,length)
#endverb%
Here the Built-In function |CENTER()| has been replaced by an internal 
function of the same name, 
which calls the original function after modifying the length argument.  

\subsection{Internal Functions}%
The interpreter creates a new storage environment when an internal function
is called,
so that the previous (caller's) environment is preserved.
The new environment inherits the values from its predecessor,
but subsequent changes to the environment variables do not affect the 
previous environment.  
The specific values that are preserved are:

\startlist
\item{$\bullet$} The current and previous host addresses,

\item{$\bullet$} The |NUMERIC DIGITS|, |FUZZ|, and |FORM| settings,

\item{$\bullet$} The trace option, inhibit flag, and interactive flag,

\item{$\bullet$} The state of the interrupt flags defined by the |SIGNAL| 
instruction, and

\item{$\bullet$} The current prompt string as set by the |OPTIONS PROMPT| 
instruction.
\endlist
\noindent%
The new environment does not automatically get a new symbol table, 
so initially all of the variables in the previous environment are 
available to the called function.
The |PROCEDURE| instruction can be used to create a new symbol table 
and thereby protect the caller's symbol values.

Execution of the internal function proceeds until a |RETURN| instruction 
is executed.
At this point the new environment is dismantled and control resumes 
at the point of the function call.
The expression supplied with the |RETURN| instruction is evaluated and 
passed back to the caller as the function result.

\subsection{Built-In Functions}%
\AR\ provides a substantial library of predefined functions as part of the
language system.
These functions are always available and have been optimized to work with 
the internal data structures. 
In general the Built-In functions execute much faster than an equivalent 
interpreted function,
so their usage is strongly recommended. 

The Built-In Function Library is not user-extensible,
but additional functions will be included in later releases.

\subsection{External Function Libraries}%
External function libraries provide a mechanism with which users and 
applications developers can extend the functionality of \AR.  
A function library is a collection of one or more functions together 
with a ``query'' entry point that serves to match a name string with 
the appropriate function.
External function libraries are supported as standard Amiga shared libraries,
and may be either memory or disk-resident.
Disk-resident libraries are loaded and opened as needed.

The \AR\ resident process maintains a list, called the \hp{Library List}, 
of the currently available function libraries and function hosts.
Applications programs can add or remove function libraries as required.
The Library List is maintained as a priority-sorted queue,
and entries can be added at an appropriate priority to control the function 
name resolution.
Libraries with higher priorities are searched first;
within a given priority level,
those libraries added first are searched first.

During the search process the \AR\ interpreter opens each library and calls 
its ``query'' entry point.
The query function must then check to see whether the requested function name 
is in the library.
If not, it returns a ``function not found'' error code and the search 
continues with the next library in the list.  
Function libraries are always closed after being checked so that 
the operating system can reclaim the memory space if required.
Once the requested function has been found,
it is called with the arguments passed by the interpreter,
and must return an error code and a result string.

The \AR\ language system includes an external function library in a file 
called ``|rexxsupport.library|.''  
It contains a number of Amiga-specific functions and is described in 
Appendix~D.
Chapter 10 provides information on designing and implementing function 
libraries.

\subsection{Function Hosts}%
Function hosts are called by sending a function invocation message packet
to the public message port identified by the host's name.
No constraints are imposed on the internal design of the host except that
it must eventually return the invocation message with an appropriate
return code and result string.
The function call may result in a new program being loaded and run,
or might even be sent to a network handler as a remote procedure call.

The available function hosts, along with the function libraries,
are contained in the Library List maintained by the resident process.
This list provides a general mechanism for resolving function names in a 
priority-controlled manner.

The \AR\ resident process is an example of a function host.
It is added to the Library List at a nominal priority of -60 when the
resident process is started,
using the same name (``|REXX|'') that is used for command invocations.
When it receives a function invocation packet, 
it searches for an external file matching the function name,
just as it would for a command invocation of the same name.  
In particular,
the search begins with the current directory and proceeds to the system
|REXX:| directory.
Two names are used in the search:  
the function name with the current file extension appended, 
and the name by itself.  
The name matching process is not case-sensitive, 
but is affected by the presence of explicit directory specifications 
or file extensions in the name string.  
The rules governing the search for external programs are covered in Chapter 9.

External programs are always run as a separate process in the Amiga's 
multitasking system.
The calling program ``sleeps'' until the called function finishes and the 
message packet returns.  
The result string and error code are returned in the packet.

\thatsit
