%
%	The ARexx User's Manual
%
% Copyright � 1987 by William S. Hawes.  All Rights Reserved.
%
\ifx\radeye\fmtversion\subdoc\else\input pmac \fi

\chapter{Chapter}{5}{Commands}%
The REXX language is unusual in that an entire syntactic class of program 
statements are reserved for \hp{commands},
statements that have meaning not within the language itself but rather to an 
external program.
When a command clause is found in a program,
it is evaluated as an expression and then sent through the 
\hp{command interface} to an explicit or implicit \hp{host application},
an external program that has announced its ability to receive commands.
The host application then processes the command and returns a result code
that indicates whether the command was performed successfully.
In this manner every host program becomes fully programmable,
and with even a limited set of predefined operations can be customized by
the end user.
\idxline{command interface}{}
\idxline{host application}{}

This chapter discusses the \AR\ command interface and examines some of
the ways in which commands can be used to build programs for an external
program.
Such programs are often called ``macro programs'' because they implement 
a complex (``macro'') action from a series of simpler ``micro'' commands.

Chapter 10 has detailed information on the data structures required to
implement a command interface for an applications program.

\section{\chapterno-}{Command Clauses}%
Syntactically, a command clause is just an expression that can't be 
classified as another type of clause.
The actual structure of the command is dictated by the external host
to which it is intended,
but in most cases will follow the model of a name or letter followed by
parameter data.
Command names can be given as either a symbol or a string.  
However, it is generally safer to use a string for the name, 
since it can't be assigned a value or be mistaken for an instruction keyword.  
For example, the following might be commands for a text editor:
\idxline{command clauses}{}

\begprog		
JUMP current+10         /* advance to next      */
'insert' newstring      /* blast it in          */
'TOP'                   /* back to the top      */
#endverb%
\noindent
Since command clauses are expressions, 
they are fully evaluated before being sent to the host.  
Any part of the final command string can be computed within the program, 
so virtually any sort of command structure can be created.

The interpretation of the received commands depends entirely on the host
application.
In the simplest case the command strings will correspond exactly to commands
that could be entered directly by a user.
For instance,
positional control (up/down) commands for a text editor would probably have
identical interpretations whether issued by the user or from a program.
Other commands may be valid only when issued from a macro program;
a command to simulate a menu operation would probably not be entered by
the user.

\section{\chapterno-}{The Host Address}%
The destination for a command is determined by the current {\it host address},
which is the name of the \hp{public message port} managed by an external
program.
\AR\ maintains two implicit host addresses, 
a ``current'' and a ``previous'' value, 
as part of the program's storage environment.
These values can be changed at any time using the |ADDRESS| instruction 
(or its synonym, |SHELL|,)
and the current host address can be inspected with the |ADDRESS()| Built-In 
function.
The default host address string is ``REXX'',
but this can be overridden when a program is invoked.
In particular, most host applications will supply the name of their 
public port when they invoke a macro program, 
so that the macro can automatically issue commands back to the host.
\idxline{host address}{}
\idxline{public message port,}{as host address}
\idxline{host address,}{with ADDRESS instruction}
\idxline{host address,}{inspecting} 

One special host address is recognized:  the string |COMMAND| indicates
that the command should be issued directly to the underlying DOS.  
All other host addresses are assumed to refer to a public message port.
An attempt to send a command to a non-existent message port will generate 
the syntax error ``Host environment not found.''
\idxline{host address,}{COMMAND}

Single commands can be sent to a specific host without disturbing the host
address settings.  
This is done using the |ADDRESS| instruction,
as the following example illustrates:

\begprog
ADDRESS MYEDIT 'jump top'
#endverb%
This example would send the command ``|jump top|'' to an external host 
named ``|MYEDIT|.'' 

It is important to note that you cannot send commands to a host application 
without knowing the name of its public message port.
Writing macro programs to communicate with two or more hosts may require 
some clever programming to determine whether both hosts are active and 
what their respective host addresses are.

\section{\chapterno-}{The Command Interface}%
\AR\ implements its command interface using the message-passing facilities
provided by the EXEC operating system.
Each host application must provide a public message port,
the name of which is referred to as the \hp{host address}.
\AR\ programs issue commands by placing the command string in a message 
packet and sending the packet to the host's message port.  
The program ``sleeps'' while the host processes the command, 
and awakens when the message packet returns.  
The entire process can be regarded as a dialogue between the host application 
and a macro program:
the host initiates the dialogue by invoking the macro,
and the macro program replies with one or more command strings.  
The commands that can be sent are not limited to simple text strings, 
but might be address pointers or even bit-mapped images.
\idxline{host address,}{in command interface}

After it finishes processing a command,
the host ``replies'' the message packet with a \hp{return code} that 
indicates the status of the command.  
This return code is placed in the \AR\ special variable |RC| so that it 
can be examined by the program.  
A value of zero is assumed to mean that no errors occurred,
while positive values usually indicate progressively more severe error 
conditions.
The return code allows the macro program to determine whether the command 
succeeded and to take action if it failed, 
so it is important for each applications program to document the meanings 
of the return codes for its commands.
\idxline{return code}{}

\section{\chapterno-}{Using Commands in Macro Programs}%
\AR\ can be used to write programs for any host application that includes a
suitable command interface.
Some applications programs are designed with an embedded macro language, 
and may include many predefined macro commands.  
With a well-designed macro language interface the user will be usually 
be unaware of whether a given action is implemented as a primitive operation 
or as a macro program.
\idxline{macro programs,}{using commands}

The starting point in designing a macro program is to examine the commands
that would be required to perform it manually.
The documentation for the host application program should then describe
the possible return codes for each command;
these codes can be used to determine whether the operation performed by 
the command was successful.  
Check also for ``shortcut'' commands that may be available only to macro
programs; 
some applications programs may include very powerful functions that were
implemented specifically for use in macro programs.

\section{\chapterno-}{Using \AR\ with Command Shells}%
Although \AR\ was designed to work most effectively with programs that
support its specific command interface,
it can be used with any ``command shell'' program that uses standard 
I/O mechanisms to obtain its input stream.
There are several ways to use \AR\ to prepare a stream of commands for such
program.

One obvious technique is to create an actual command file on the ``RAM:''
disk and then pass it directly to the command shell.
For example, 
you could open a new CLI window to run a standard ``execute'' script using
the following short program:
\begprog
/* Launch a new CLI  */
address command
conwindow = "CON:0/0/640/100/NewOne"

/* create a command file on the fly */
call open out,"ram:$$temp",write
call writeln out,'echo "this is a test"'
call close out

/* open the new CLI window */
'newcli' conwindow "ram:$$temp"
exit
\endverb
\noindent%
Since no disk accesses are required,
this method is actually fairly fast, if not very elegant.

Another alternative is to use the command stacking facility provided by the
|PUSH| and |QUEUE| instructions.
These instructions allow an \AR\ program to stack an arbitrary stream of
commands and data for the command shell or other program to read.
Any set of commands that could be ``typed ahead'' at a command prompt can
be prepared in this fashion.
After the \AR\ program exits,
the next program that uses the input stream will read the prepared commands 
and can process them in the normal fashion.

\section{\chapterno-}{Command Inhibition}%
Sometimes it is necessary to write and test macro programs that issue
potentially destructive commands.
For instance,
a program to find and delete unneeded files would be difficult to test safely,
since it might accidentally delete the wrong files and would require a
continual source of new files for testing.

To simplify the development and testing of such programs,
\AR\ provides a special tracing mode called \hp{command inhibition} that
suppresses host commands.
While in command inhibition mode,
command processing proceeds normally except that the command is not 
actually issued and the variable |RC| is set to 0.
This allows the program logic to be verified before any commands are actually
sent to the external program.
Chapter 7 has further information on this facility.
\idxline{command inhibition,}{in testing}

\thatsit
