%
%	The ARexx User's Manual
%
% Copyright � 1987 by William S. Hawes.  All Rights Reserved.
%
\ifx\radeye\fmtversion\subdoc\else\input pmac\fi

\chapter{Appendix}{D}{The \AR\ Support Library}%
The \AR\ language system is distributed with an external function library 
that provides a number of Amiga-specific functions.
It is a standard Amiga shared library named |rexxsupport.library| and should 
reside in the system |LIBS:| directory.
Unlike the Systems Library described in the previous Appendix,
the support library functions are callable from with \AR\ programs.

The support library was designed to supplement the generic Built-In functions
with functions specific to the Amiga.
This library will be expanded in future releases,
and users are encouraged to submit suggestions for additional functions.

The Support Library must be added to the global Library List before it can be 
accessed by \AR\ programs.
This can be done using the Built-In function |ADDLIB()| or by direct
communication with the resident process.
The library name must be specified as |rexxsupport.library|, 
the query function offset is -30, and the version number is 0.
The search priority can be set to 0 or whatever value is appropriate.

\subsection{ALLOCMEM()}%
\usage{|ALLOCMEM(|length,[attribute]|)|}
Allocates a block of memory of the specified length from the system 
free-memory pool and returns its address as a 4-byte string.
The optional \hp{attribute} parameter must be a standard EXEC memory 
allocation flag,
supplied as a 4-byte string.
The default attribute is for ``PUBLIC'' memory (not cleared).  

This function should be used whenever memory is allocated for use by external 
programs.
It is the user's responsibility to release the memory space when it is
no longer needed.
\idxline{ALLOCMEM(),}{support library function}
\seealso{|FREEMEM()|}

\begexamp
   say c2x(allocmem(1000))      ==> 00050000
#endverb

\subsection{CLOSEPORT()}%
\usage{|CLOSEPORT(|name|)|}
Closes the message port specified by the \hp{name} argument, 
which must have been allocated by a call to |OPENPORT()| within the current
\AR\ program.
Any messages received but not yet |REPLY|ed are automatically returned with
the return code set to 10.
\idxline{CLOSEPORT(),}{support library function}
\seealso{|OPENPORT()|}

\begexamp
call closeport myport
#endverb

\page
\subsection{FREEMEM()}%
\usage{|FREEMEM(|address,length|)|}
Releases a block of memory of the given length to the system freelist.  
The \hp{address} parameter is a four-byte string, 
typically obtained by a prior call to |ALLOCMEM()|.  
|FREEMEM()| cannot be used to release memory allocated using |GETSPACE()|,
the \AR\ internal memory allocator.
The returned value is a boolean success flag.
\idxline{FREEMEM(),}{support library function}

\seealso{|ALLOCMEM()|}

\begexamp
say freemem('00042000'x,32)     ==> 1
#endverb

\subsection{GETARG()}%
\usage{|GETARG(|packet,[n]|)|}
Extracts a command, function name, or argument string from a message packet.
The {\it packet} argument must be a 4-byte address obtained from a prior 
call to |GETPKT()|.
The optional {\it n} argument specifies the slot containing the string 
to be extracted, 
and must be less than or equal to the actual argument count for the packet.
Commands and function names are always in slot 0;
function packets may have argument strings in slots 1--15.
\idxline{GETARG(),}{support library function}

\begexamps
command  = getarg(packet)
function = getarg(packet,0)  /* name string  */
arg1     = getarg(packet,1)  /* 1st argument */
#endverb

\subsection{GETPKT()}%
\usage{|GETPKT(|name|)|}
Checks the message port specified by the \hp{name} argument to see whether
any messages are available.
The named message port must have been opened by a prior call to |OPENPORT()| 
within the current \AR\ program.
The returned value is the 4-byte address of the first message packet,
or |'0000 0000'x| if no packets were available.

The function returns immediately whether or not a packet is enqueued at the
message port.
Programs should never be designed to ``busy--loop'' on a message port.
If there is no useful work to be done until the next message packet arrives,
the program should call |WAITPKT()| and allow other tasks to proceed.
\idxline{GETPKT(),}{support library function}
\seealso{|WAITPKT()|}

\begexamp
packet = getpkt('MyPort')
#endverb

\subsection{OPENPORT()}%
\usage{|OPENPORT(|name|)|}
Creates a public message port with the given name.
The returned value is the 4-byte address of the Port Resource structure 
or |'0000 0000'x| if the port could not be opened or initialized.  
An initialization failure will occur if another port of the same name already 
exists, 
or if a signal bit couldn't be allocated.

The message port is allocated as a Port Resource node and is linked into
the program's global data structure.
Ports are automatically closed when the program exits,
and any pending messages are returned to the sender.
\idxline{OPENPORT(),}{support library function}

\seealso{|CLOSEPORT()|}

\begexamp
myport = openport("MyPort")
#endverb

\subsection{REPLY()}%
\usage{|REPLY(|packet,rc|)|}
Returns a message packet to the sender, 
with the primary result field set to the value given by the \hp{rc} argument.
The secondary result is cleared.  
The \hp{packet} argument must be supplied as a 4-byte address,
and the \hp{rc} argument must be a whole number.  
\idxline{REPLY(),}{support library function}

\begexamp
call reply packet,10            /* error return */
#endverb

\subsection{SHOWDIR()}%
\usage{|SHOWDIR(|directory,[|'All'| |\|| |'File'| |\|| |'Dir'|]|)|}
Returns the contents of the specified directory as a string of names 
separated by blanks.  
The second parameter is an option keyword that selects whether all entries, 
only files, or only subdirectories will be included.
\idxline{SHOWDIR(),}{support library function}

\begexamp
say showdir("df1:c")            ==> rx ts te hi tco tcc
#endverb

\subsection{SHOWLIST()}%
\usage{|SHOWLIST(||\{||'D' \| ~'L' \| ~'P' \| ~'R' \| ~'W'||\}|,[name]|)|}
The first argument is an option keyword to select a system list; 
the options currently supported are |Devices|, |Libraries|, |Ports|,
|Ready|, and |Waiting|.
If only the first parameter is supplied, 
the function scans the selected list and returns the node names in a 
string separated by blanks.
If the \hp{name} parameter is supplied, 
the boolean return indicates whether the specified list contains a node 
of that name.
The name matching is case-sensitive.

The list is scanned with task switching forbidden so as to provide an
accurate snapshot of the list at that time.
\idxline{SHOWLIST(),}{support library function}

\begexamp
say showlist('P')               ==> REXX MyCon
say showlist('P','REXX')        ==> 1
#endverb%

\page
\subsection{STATEF()}%
\usage{|STATEF(|filename|)|}
Returns a string containing information about an external file.
The string is formatted as\hfil
\centerline{``\hp{|\{DIR| |\|| |FILE\}| length blocks protection comment}.''}
The \hp{length} token gives the file length in bytes,
and the \hp{block} token specifies the file length in blocks.
\idxline{STATEF(),}{support library function}

\begexamp
say statef("libs:rexxsupport.library") 
/* would give "FILE 1880 4 RWED  "      */ 
#endverb%

\subsection{WAITPKT()}%
\usage{|WAITPKT(|name|)|}
Waits for a message to be received at the specified (named) port, 
which must have been opened by a call to |OPENPORT()| within the current
\AR\ program.
The returned boolean value indicates whether a message packet is available
at the port.
Normally the returned value will be 1 ({\bf True}),
since the function waits until an event occurs at the message port.

The packet must then be removed by a call to |GETPKT()|, 
and should be returned eventually using the |REPLY()| function.
Any message packets received but not returned when an \AR\ program exits
are automatically |REPLY|ed with the return code set to 10.
\idxline{WAITPKT(),}{support library function}

\begexamp
call waitpkt 'MyPort'   /* wait awhile */
#endverb

\thatsit
